\documentclass[12pt,a4paper]{article}
\usepackage[utf8]{inputenc}
\usepackage[spanish]{babel}
\usepackage[margin=2.5cm]{geometry}
\usepackage{graphicx}
\usepackage{hyperref}
\hypersetup{
    colorlinks=true,
    linkcolor=black,
    citecolor=black,
    urlcolor=black
}

\usepackage{fancyhdr}
\usepackage{booktabs}
\usepackage{array}
\usepackage{enumitem}
\usepackage{xcolor}
\usepackage{titlesec}
\usepackage{listings}
\usepackage{float}
\usepackage{longtable}
\usepackage{tabularx}
\usepackage{pdflscape}
\usepackage{amsmath}
\usepackage{amsfonts}
\usepackage{setspace}
\usepackage{multirow}
\usepackage{colortbl}
\usepackage{microtype}
\usepackage{placeins}
% Configuración de espaciado
\setlength{\parindent}{0pt}
\setlength{\parskip}{0.8em}
\onehalfspacing

% Colores institucionales UNSCH
\definecolor{unschblue}{RGB}{0,0,0}
\definecolor{unschgold}{RGB}{184,134,11}
\definecolor{tablehead}{RGB}{230,240,250}
\definecolor{tablerow}{RGB}{245,248,252}

% Configuración de encabezados y pies de página
\pagestyle{fancy}
\fancyhf{}
\fancyhead[L]{\small\textit{SIGHC - Sistema Integral de Gestión de Historias Clínicas}}
\fancyfoot[C]{\thepage}
\renewcommand{\headrulewidth}{0.5pt}
\renewcommand{\footrulewidth}{0.3pt}
\renewcommand{\thesection}{\Roman{section}}

% Formato de títulos profesional
\titleformat{\section}
{\normalfont\LARGE\bfseries\color{unschblue}}
{\thesection}{1em}{}[\titlerule]
\titleformat{\subsection}
{\normalfont\Large\bfseries\color{unschblue}}
{\thesubsection}{1em}{}
\titleformat{\subsubsection}
{\normalfont\large\bfseries}
{\thesubsubsection}{1em}{}

% Configuración listings SQL profesional
\lstdefinestyle{sqlstyle}{
    backgroundcolor=\color{gray!8},
    commentstyle=\color{gray!60}\itshape,
    keywordstyle=\color{unschblue}\bfseries,
    stringstyle=\color{red!70!black},
    basicstyle=\ttfamily\small,
    breaklines=true,
    captionpos=b,
    numbers=left,
    numberstyle=\tiny\color{gray!50},
    numbersep=8pt,
    frame=single,
    framesep=8pt,
    tabsize=2,
    showstringspaces=false,
    xleftmargin=15pt,
    framexleftmargin=12pt,
    language=SQL,
    morekeywords={IDENTITY,NVARCHAR,VARCHAR,DATETIME2,DECIMAL,UNIQUE,CONSTRAINT,REFERENCES,DEFAULT,SYSDATETIME,CHECK}
}
% Formato de títulos en NEGRO
\titleformat{\section}
{\normalfont\LARGE\bfseries\color{black}}
{\thesection}{1em}{}[\titlerule]

\titleformat{\subsection}
{\normalfont\Large\bfseries\color{black}}
{\thesubsection}{1em}{}

\titleformat{\subsubsection}
{\normalfont\large\bfseries\color{black}}
{\thesubsubsection}{1em}{}
% APLICAR EL ESTILO GLOBALMENTE
\lstset{style=sqlstyle}

% Comando para tablas profesionales
\newcommand{\tablaheader}[1]{\rowcolor{tablehead}\textbf{#1}}

% =====================================================
% CONFIGURACIÓN DE ENUMERACIONES ESTILO APA 7ma EDICIÓN
% Sistema: I., 1.1., 1.1.1., 1.1.1.1.
% =====================================================

% Configurar numeración de secciones con números romanos
\renewcommand{\thesection}{\Roman{section}}          % I, II, III ...
\renewcommand{\thesubsection}{\arabic{section}.\arabic{subsection}}      % 1.1, 1.2 ...
\renewcommand{\thesubsubsection}{\arabic{section}.\arabic{subsection}.\arabic{subsubsection}} % 1.1.1


% Configuración para listas enumerate (dentro del texto)
\setlist[enumerate,1]{
    label=\Roman*.,
    ref=\Roman*,
    leftmargin=2.54cm,
    itemsep=6pt,
    parsep=0pt,
    topsep=6pt,
    font=\bfseries
}

\setlist[enumerate,2]{
    label=\arabic{enumi}.\arabic*.,
    ref=\arabic{enumi}.\arabic*,
    leftmargin=3.17cm,
    itemsep=3pt,
    parsep=0pt,
    topsep=3pt,
    font=\bfseries
}

\setlist[enumerate,3]{
    label=\arabic{enumi}.\arabic{enumii}.\arabic*.,
    ref=\arabic{enumi}.\arabic{enumii}.\arabic*,
    leftmargin=3.81cm,
    itemsep=0pt,
    parsep=0pt,
    topsep=3pt,
    font=\normalfont
}

\setlist[enumerate,4]{
    label=\arabic{enumi}.\arabic{enumii}.\arabic{enumiii}.\arabic*.,
    ref=\arabic{enumi}.\arabic{enumii}.\arabic{enumiii}.\arabic*,
    leftmargin=4.45cm,
    itemsep=0pt,
    parsep=0pt,
    topsep=3pt,
    font=\normalfont
}


\newcounter{requisitotabla}

\newcommand{\requisitotabla}[2]{%
    \refstepcounter{requisitotabla}%
    \noindent\textbf{Tabla \therequisitotabla}\\[2mm]
    \textit{#1: #2}\\[4mm]
}


\begin{document}
% =====================================================
% PORTADA PROFESIONAL - UNSCH
% =====================================================
\begin{titlepage}
    \centering

    {\LARGE \textbf{UNIVERSIDAD NACIONAL SAN CRISTÓBAL DE HUAMANGA}} \\
    \vspace{0.3cm}
    {\large Facultad de Ingeniería de Minas, Geología y Civil} \\
    {\large Escuela Profesional de Ingeniería de Sistemas} \\

    \vspace{0.6cm}

    \begin{figure}[h!]
        \centering
        \includegraphics[width=0.25\linewidth]{820958ea-0a0c-454d-bb6a-a5644a478fa9.jpg}
    \end{figure}

    \vspace{0.4cm}
    {\LARGE \textbf{Sistema Integral de Gestión de Historias Clínicas (SIGHC)}} \\
    \vspace{0.2cm}
    

    \vspace{0.6cm}
    {\Large Asignatura: Construcción y Evolución de Software} \\

    \vspace{0.8cm}
    {\large \textbf{Presentado por:}} \\[0.2cm]


\begin{tabular}{l l}
    \textbf{Nombres y Apellidos} & \textbf{Código} \\
    \hline
    De la Cruz Flores, Anngy Gina &27222513\\
    Gutierrez Gutierrez, Keyla Jhazym &27220300\\
    Montero Gutiérrez Brandon Fernando&27222102\\
    Ovalle Luyo, Steve Smith & 27220112 \\
    Paipay Vega, Eduardo Sebastián & 27222136 \\

  \end{tabular}


    \vspace{0.8cm}
    {\large \textbf{Docente a cargo:}} \\
    Ing.{ RICHARD ZAPATA CASAVERDE} \\

    \vspace{0.6cm}
    {\large Ayacucho, Perú} \\
    {\large Diciembre 2025}
\end{titlepage}

% =====================================================
% ÍNDICES DEL DOCUMENTO
% =====================================================

\tableofcontents
\newpage

\listoffigures
\newpage

\listoftables
\newpage




\section*{Resumen Ejecutivo}

El presente documento describe el diseño del \textbf{Sistema Integral de Gestión de Historias Clínicas (SIGHC)}, una solución informática orientada a la digitalización, centralización y gestión segura de la información clínica del Hospital Regional de Ayacucho. El sistema ha sido concebido como una plataforma tecnológica que integra los principales procesos clínicos y administrativos de la institución, permitiendo la gestión de pacientes, citas médicas, consultas, diagnósticos bajo el estándar CIE--10, tratamientos, inventario de medicamentos, control de accesos y auditoría de operaciones, conforme a los lineamientos definidos en la especificación técnica del sistema.

El objetivo principal del SIGHC es proporcionar una plataforma centralizada, confiable y segura que facilite la gestión eficiente de la información clínica y administrativa, apoyando al personal médico y administrativo en la toma de decisiones oportunas y mejorando la continuidad de la atención. Asimismo, el sistema busca asegurar el cumplimiento de la normativa vigente en materia de gestión de historias clínicas y protección de datos personales establecida por el Ministerio de Salud, mediante mecanismos de control, trazabilidad y seguridad de la información.

La digitalización de las historias clínicas constituye un elemento clave en la modernización del sector salud, ya que permite un acceso rápido, seguro y controlado a la información médica del paciente desde las distintas áreas del hospital. Este enfoque reduce significativamente los riesgos asociados a la pérdida, deterioro o duplicidad de registros físicos, fortalece la trazabilidad de las atenciones médicas y mejora la coordinación entre los servicios clínicos, contribuyendo a una atención más oportuna, segura y de mayor calidad.

El sistema SIGHC se desarrolla sobre una arquitectura cliente--servidor y utiliza una base de datos relacional como núcleo de almacenamiento de la información clínica y administrativa. Para su diseño se consideran tecnologías robustas y ampliamente adoptadas en el ámbito institucional, tales como sistemas de gestión de bases de datos relacionales, lenguajes de programación para el desarrollo del backend, herramientas de control de versiones y entornos de desarrollo especializados. Asimismo, el sistema incorpora mecanismos de seguridad como autenticación segura, cifrado de información y control de accesos basado en roles, garantizando la confidencialidad, integridad y trazabilidad de los datos clínicos.

% =====================================================
% 1. INTRODUCCIÓN
% =====================================================
\newpage
\section{Introducción}

En la actualidad, la transformación digital en el sector salud constituye un factor determinante para garantizar la calidad, eficiencia y seguridad en la atención al paciente. Los hospitales públicos enfrentan el desafío de gestionar grandes volúmenes de información clínica de manera oportuna y confiable, lo que exige la adopción de sistemas de información robustos que permitan el acceso inmediato, organizado y seguro a los datos médicos, en concordancia con las normativas vigentes y los estándares internacionales de salud.

En este contexto, la gestión de historias clínicas en los hospitales públicos continúa realizándose, en muchos casos, mediante procesos manuales o semi--automatizados, caracterizados por el uso de documentos físicos y sistemas informáticos fragmentados que no se encuentran integrados entre sí. Esta situación genera dependencia de archivos en papel, dispersión de la información y redundancias en los registros, dificultando el acceso oportuno a los antecedentes médicos del paciente y limitando la coordinación entre áreas como admisión, consultorios, farmacia y archivo clínico. Como consecuencia, se producen demoras en la atención, incremento de los tiempos de espera y restricciones en la generación de reportes e indicadores para la toma de decisiones clínicas y administrativas.

El manejo tradicional de las historias clínicas conlleva además problemáticas críticas, tales como la pérdida, deterioro o duplicidad de documentos, errores en el registro de datos y una limitada trazabilidad de las modificaciones realizadas sobre la información médica. La ausencia de mecanismos formales de auditoría y de control de accesos dificulta identificar quién accede o modifica los registros clínicos, incrementando los riesgos asociados a la seguridad y confidencialidad de los datos sensibles del paciente. Estas limitaciones afectan directamente la continuidad asistencial y exponen a las instituciones de salud a posibles incumplimientos de la normativa sobre protección de datos personales.

Frente a este escenario, la implementación de un sistema integral de gestión clínica se presenta como una solución estratégica para modernizar los procesos médico--administrativos. Un sistema como el Sistema Integral de Gestión de Historias Clínicas (SIGHC) permite centralizar la información de pacientes, citas, consultas, diagnósticos basados en el estándar CIE--10, tratamientos y medicamentos en una base de datos relacional normalizada, garantizando la integridad referencial y eliminando la redundancia de datos. Esta centralización, combinada con el uso de vistas optimizadas, procedimientos almacenados y mecanismos de auditoría, posibilita una reducción significativa en los tiempos de búsqueda de historias clínicas y una disminución de los errores asociados a la duplicidad de información, mejorando la continuidad asistencial y la consistencia de los datos clínicos.






% =====================================================
% 3. OBJETIVOS
% =====================================================

\section{Objetivos}

\subsection{Objetivo General}
Diseñar e implementar el \textbf{Sistema Integral de Gestión de Historias Clínicas (SIGHC)}, una plataforma centralizada de base de datos que permita digitalizar y automatizar el flujo de información médica y administrativa del Hospital Regional de Ayacucho, garantizando la integridad, disponibilidad y confidencialidad de los datos.

\subsection{Objetivos Específicos}
\begin{itemize}[leftmargin=1.5em, label=\color{unschblue}\textbullet]
    \item \textbf{Gestionar el ciclo de vida de la Historia Clínica:} Centralizar los antecedentes, triaje, diagnósticos y tratamientos en un expediente electrónico único y accesible.
    \item \textbf{Automatizar la gestión de Citas Médicas:} Implementar una agenda electrónica que permita la programación, reprogramación y cancelación de citas, vinculando pacientes con médicos disponibles.
    \item \textbf{Estandarizar los Diagnósticos:} Integrar el catálogo internacional CIE-10 para el registro preciso y normalizado de enfermedades y diagnósticos.
    \item \textbf{Controlar el Inventario y Farmacia:} Gestionar el stock de medicamentos en tiempo real y vincular la emisión de recetas electrónicas con el despacho en farmacia.
    \item \textbf{Implementar Auditoría y Seguridad:} Desarrollar mecanismos de control de acceso basados en roles y registrar todas las transacciones (logs) para garantizar la trazabilidad de los datos según la normativa vigente.
\end{itemize}

% =====================================================
% 5. ALCANCES Y LIMITACIONES
% =====================================================
\newpage
\section{Alcances y Limitaciones}

\subsection{Alcances}
El sistema SIGHC abarcará los siguientes módulos funcionales:

\begin{itemize}[label=\checkmark, color=unschblue]
    \item \textbf{Módulo de Admisión y Citas:} Registro de pacientes y programación de agenda médica.
    \item \textbf{Módulo de Historia Clínica Electrónica:} Registro de triaje, consulta externa, diagnósticos (CIE-10) y evolución médica.
    \item \textbf{Módulo de Farmacia:} Gestión del catálogo de medicamentos, stock y despacho de recetas electrónicas.
    \item \textbf{Módulo de Seguridad:} Gestión de usuarios (Médicos, Admisión, Farmacia, Admin), roles y auditoría de datos.
    \item \textbf{Reportes Operativos:} Generación de reportes básicos de atenciones y recetas.
\end{itemize}

\subsection{Limitaciones}
\begin{itemize}[leftmargin=1.5em]
    \item \textbf{Alcance Académico:} El proyecto se limita al diseño y prototipado funcional de la base de datos y backend para el curso, sin incluir la implementación física en servidores del hospital en esta etapa.
    \item \textbf{Facturación y Seguros:} No se incluirá el módulo de facturación electrónica ni integración con aseguradoras (SIS/EsSalud) en esta primera versión.
    \item \textbf{Imágenes Médicas:} No se contempla el almacenamiento de imágenes pesadas (Rayos X, Tomografías) en la base de datos (sistema PACS), solo referencias textuales.
    \item \textbf{Infraestructura:} La implementación depende de la infraestructura tecnológica (computadoras y red) que posea el hospital, la cual no es parte del presupuesto de este proyecto.
\end{itemize}

\subsubsection{Módulos Principales del Sistema}
\begin{itemize}[leftmargin=2em]
    \item \textbf{M01:} Gestión de Pacientes e Historias Clínicas
    \item \textbf{M02:} Gestión de Citas Médicas y Agenda
    \item \textbf{M03:} Consultas y Diagnósticos CIE-10
    \item \textbf{M04:} Tratamientos y Prescripciones Médicas
    \item \textbf{M05:} Gestión de Personal Médico
    \item \textbf{M06:} Inventario de Medicamentos
    \item \textbf{M07:} Usuarios, Roles y Seguridad (RBAC)
    \item \textbf{M08:} Auditoría y Trazabilidad
    \item \textbf{M09:} Reportes y Business Intelligence
    \item \textbf{M10:} Respaldo y Recuperación
\end{itemize}

\subsection{Referencias Normativas}

\begin{itemize}[leftmargin=2em]
    \item IEEE Std 830-1998: IEEE Recommended Practice for Software Requirements Specifications
    \item ISO/IEC 25010:2011 - Systems and Software Quality Requirements
    \item Microsoft SQL Server 2019 Documentation - Best Practices
    \item Ley N° 29733 - Ley de Protección de Datos Personales (Perú)
    \item NTS N° 139-MINSA/2018 - Norma Técnica de Historia Clínica
    \item Clasificación Internacional de Enfermedades (CIE-10)
    \item ISO 27001:2013 - Information Security Management
    \item OWASP Top 10 2021 - Web Application Security Risks
\end{itemize}


%
%DESCRIPCION DEL PROBLEMA
%
\newpage
\section{Descripción del Problema}

En el Hospital Regional de Ayacucho, la gestión tradicional de las historias clínicas se ha basado históricamente en el uso de documentos físicos y registros parcialmente digitalizados, lo que ha generado múltiples deficiencias operativas, administrativas y de seguridad. La ausencia de una plataforma integral y centralizada limita la eficiencia del personal de salud y afecta directamente la calidad, oportunidad y continuidad de la atención médica brindada a los pacientes.

\subsection{Uso de historias clínicas físicas}

El uso predominante de historias clínicas físicas implica la dependencia de archivos en papel para el registro, consulta y actualización de la información médica del paciente. Este enfoque dificulta el acceso oportuno a los antecedentes clínicos, especialmente en contextos de alta demanda asistencial, y limita la disponibilidad simultánea de la información para distintos servicios del hospital. Asimismo, el manejo físico incrementa el desgaste, deterioro y extravío de documentos, afectando la integridad histórica de los registros clínicos.

\subsection{Pérdida o duplicidad de información médica}

La falta de un sistema centralizado y de mecanismos automáticos de validación ha generado escenarios recurrentes de pérdida, duplicidad e inconsistencia de datos clínicos. Registros duplicados de pacientes, diagnósticos incompletos y tratamientos no consolidados dificultan la continuidad asistencial y elevan el riesgo de errores médicos. Según el análisis técnico del sistema propuesto, estos problemas han contribuido a una alta tasa de errores por duplicidad de datos, situación que el SIGHC busca reducir en más del 95\% mediante control de unicidad, normalización de datos y validaciones automáticas.

\subsection{Retrasos en la atención al paciente}

Los procesos manuales y semi--automatizados incrementan significativamente los tiempos de búsqueda y recuperación de historias clínicas, generando retrasos en la programación de citas, la atención médica y la toma de decisiones clínicas. En el contexto actual, la localización de una historia clínica puede tomar varios minutos, afectando la productividad del personal y aumentando los tiempos de espera del paciente. El diagnóstico técnico evidencia que estos retrasos impactan negativamente en el cumplimiento de citas y en la eficiencia operativa del hospital.

\subsection{Falta de trazabilidad y auditoría}

El modelo tradicional carece de mecanismos formales de auditoría y trazabilidad que permitan identificar de manera precisa quién accede, modifica o consulta la información clínica. Esta ausencia de registros detallados impide reconstruir el historial de cambios sobre datos críticos como diagnósticos, tratamientos o antecedentes médicos, dificultando la supervisión, el control interno y la rendición de cuentas. Además, limita la capacidad del hospital para realizar auditorías clínicas y administrativas conforme a los requerimientos normativos vigentes.

\subsection{Riesgos en la seguridad de datos sensibles}

La gestión manual y la inexistencia de controles tecnológicos robustos exponen la información médica a riesgos significativos de acceso no autorizado, alteración o pérdida. La carencia de control de accesos basado en roles, cifrado de datos, registros de acceso y políticas formales de respaldo incrementa la probabilidad de incumplimiento de la Ley N.\textsuperscript{o}~29733 de Protección de Datos Personales y de la NTS N.\textsuperscript{o}~139--MINSA sobre historias clínicas. Estos riesgos comprometen la confidencialidad, integridad y disponibilidad de la información clínica, haciendo imprescindible la adopción de un sistema integral con seguridad y auditoría de nivel hospitalario.





% =====================================================
% 4. JUSTIFICACIÓN
% =====================================================
\newpage
\section{Justificación}

\begin{description}[style=multiline, leftmargin=4.5cm, font=\bfseries\color{unschblue}]
    \item[Justificación Tecnológica] El proyecto aplicará conceptos avanzados de modelado de bases de datos, garantizando la integridad referencial y la optimización de consultas para manejar grandes volúmenes de información histórica.
    
    \item[Justificación Operativa] La automatización reducirá drásticamente el tiempo administrativo dedicado al llenado de formularios manuales y búsqueda de papeles, permitiendo que el personal médico dedique más tiempo a la atención directa del paciente.
    
    \item[Justificación Social] Los beneficiarios directos son los pacientes de la región de Ayacucho, quienes recibirán una atención más rápida y segura, con un historial clínico legible y disponible para cualquier especialidad médica dentro del hospital.
    
    \item[Justificación Legal] El sistema se alinea con las normativas de protección de datos personales y modernización del estado, cumpliendo con estándares de auditoría exigidos por las entidades de control.
\end{description}

\newpage
%
%Público Objetivo y Beneficiarios
%
\section{Público Objetivo y Beneficiarios}

El Sistema Integral de Gestión de Historias Clínicas (SIGHC) está orientado a diversos grupos de interés que participan directa o indirectamente en los procesos clínicos y administrativos del hospital público para el cual ha sido diseñado. El sistema busca atender las necesidades operativas, asistenciales y de control definidas en la especificación técnica del SIGHC.

\subsection{Hospitales y centros de salud}

El principal público objetivo del sistema SIGHC son los hospitales y centros de salud públicos que requieren modernizar la gestión de sus historias clínicas y optimizar sus procesos médico--administrativos. En particular, el sistema está diseñado para el Hospital Regional de Ayacucho, considerando su estructura organizacional, volumen de pacientes y flujos de atención. La implementación del SIGHC permite a estas instituciones centralizar la información clínica, mejorar la eficiencia operativa, fortalecer la seguridad de los datos y disponer de información confiable para la toma de decisiones institucionales.

\subsection{Personal médico y administrativo}

El personal médico y administrativo constituye uno de los principales beneficiarios del sistema. Médicos, personal de enfermería, personal de admisión, farmacia, auditoría y administración utilizan el SIGHC para registrar, consultar y gestionar la información clínica de manera segura y oportuna. El sistema reduce los tiempos de búsqueda de historias clínicas, facilita el registro estructurado de consultas y diagnósticos CIE--10, optimiza la programación de citas médicas y mejora la coordinación entre las distintas áreas del hospital, incrementando la productividad y reduciendo errores operativos.

\subsection{Pacientes}

Los pacientes se benefician de manera indirecta mediante una atención médica más rápida, segura y continua. Al contar con historias clínicas electrónicas centralizadas, el personal de salud puede acceder oportunamente a los antecedentes médicos, diagnósticos y tratamientos, lo que contribuye a una mejor toma de decisiones clínicas y a la reducción de tiempos de espera. Asimismo, el uso de información clínica confiable disminuye el riesgo de errores médicos y mejora la calidad global del servicio de salud brindado.

\subsection{Entidades reguladoras del sector salud}

Las entidades reguladoras del sector salud, como el Ministerio de Salud y los órganos de supervisión y auditoría, se benefician del SIGHC al disponer de información clínica estructurada, trazable y auditable. El sistema facilita el cumplimiento de la normativa vigente en materia de gestión de historias clínicas y protección de datos personales, permitiendo la generación de reportes, auditorías clínicas y administrativas, y fortaleciendo los mecanismos de control y supervisión institucional.


%%
%% METODOLOGIA Y CICLO DE VIDA DEL SOFTWARE
%%
\newpage
\section{Metodología y Ciclo de Vida}

El desarrollo del Sistema Integral de Gestión de Historias Clínicas (SIGHC) se enmarca dentro de una metodología estructurada que permite asegurar la correcta identificación de requerimientos, el diseño consistente de la solución y la documentación técnica necesaria para su futura implementación. Considerando que el proyecto se centra principalmente en el diseño lógico, funcional y de base de datos del sistema, se adopta un enfoque metodológico híbrido, combinando elementos de SCRUM y RUP.

\subsection{Metodología Seleccionada}

Para el proyecto SIGHC se emplea una metodología híbrida basada en \textbf{SCRUM} y \textbf{Rational Unified Process (RUP)}, aprovechando las fortalezas de ambos enfoques. SCRUM se utiliza como marco de referencia para la gestión iterativa de los requerimientos, permitiendo organizar el trabajo en incrementos y priorizar funcionalidades críticas del sistema, tales como la gestión de historias clínicas, citas médicas, diagnósticos y seguridad de accesos.

Por su parte, RUP aporta una estructura formal para el análisis y diseño del sistema, lo cual resulta fundamental en un proyecto que requiere alta rigurosidad técnica, trazabilidad de requerimientos, definición clara de casos de uso, modelado de la arquitectura y diseño detallado de la base de datos relacional. Este enfoque es coherente con el contenido del PDF del SIGHC, el cual enfatiza la especificación técnica, la normalización de datos, el control de integridad y los mecanismos de auditoría.

La combinación de SCRUM y RUP permite mantener flexibilidad en la definición de funcionalidades, sin sacrificar la documentación y el control necesarios en sistemas críticos del sector salud.

\subsection{Fases del Ciclo de Vida}

El ciclo de vida del software del proyecto SIGHC se organiza en las siguientes fases:

\begin{itemize}
    \item \textbf{Análisis:} En esta fase se identifican y documentan los requerimientos funcionales y no funcionales del sistema, las reglas de negocio, los actores involucrados y los casos de uso. Asimismo, se analizan los procesos actuales de gestión clínica y las problemáticas asociadas al manejo manual o semi--automatizado de la información médica.

    \item \textbf{Diseño:} Comprende el diseño lógico y funcional del sistema, incluyendo la arquitectura cliente--servidor, el modelo entidad--relación, la normalización de la base de datos hasta la Tercera Forma Normal (3NF), la definición de tablas, claves foráneas, restricciones, vistas, procedimientos almacenados y disparadores de auditoría. Esta fase constituye el núcleo del proyecto descrito en el PDF del SIGHC.

    \item \textbf{Desarrollo:} Corresponde a la implementación de los componentes definidos en la fase de diseño, principalmente la construcción de la base de datos, scripts SQL, procedimientos almacenados, triggers y estructuras de seguridad. En el contexto del presente proyecto, esta fase se aborda a nivel de especificación técnica y scripts de referencia, sin llegar a un despliegue productivo.

    \item \textbf{Pruebas:} Incluye la validación de los requerimientos definidos, la verificación de la integridad de los datos, el correcto funcionamiento de las reglas de negocio, la seguridad de accesos y la consistencia de la información clínica. Se consideran pruebas funcionales y de seguridad a nivel lógico.

    \item \textbf{Implementación:} Esta fase corresponde al despliegue del sistema en un entorno productivo, configuración de servidores y capacitación de usuarios. No forma parte del alcance inmediato del proyecto, pero se define como una etapa futura sustentada en el diseño realizado.

    \item \textbf{Mantenimiento:} Considera las actividades de soporte, corrección de errores, mejoras evolutivas y adaptación del sistema a cambios normativos o requerimientos institucionales, una vez que el sistema sea implementado en producción.
\end{itemize}


% =====================================================
% 2. CATÁLOGO DE MÓDULOS (TABLE + TABULAR, dividido en 2 páginas)
% =====================================================
\newpage
\section{Catálogo de Módulos del Sistema}
\textbf{Tabla 1}\\[2mm]\textit{Descripción de Módulos}

% ===================== TABLA 1 =====================
\begin{table}[H]
\centering
\label{tab:modulos_part1}

\begin{tabular}{|p{1.2cm}|p{3.5cm}|p{9.5cm}|}
\hline
\tablaheader{ID} & \tablaheader{Módulo} & \tablaheader{Descripción Funcional} \\
\hle

M01 & Gestión de Pacientes & Administración integral del ciclo de vida del paciente: registro con historia clínica única, actualización de datos demográficos, gestión de antecedentes personales y familiares, búsqueda avanzada con múltiples criterios, prevención de duplicados por DNI y gestión de estados (activo, inactivo, fallecido). \\
\hline

M02 & Gestión de Citas & Sistema completo de programación de citas médicas: agenda electrónica por médico y especialidad, validación de disponibilidad horaria, reprogramación y cancelación con motivo, confirmación automática por SMS/Email, lista de espera y consulta del historial de citas. \\
\hline

M03 & Consultas Médicas & Registro electrónico de consultas: diagnóstico con codificación CIE-10, signos vitales (presión arterial, temperatura, frecuencia cardíaca, peso, talla e IMC), evolución clínica, examen físico, motivo de consulta, plan de trabajo y registro de interconsultas. \\
\hline

M04 & Tratamientos & Prescripción electrónica: registro de tratamientos farmacológicos y no farmacológicos, dosis e indicaciones, duración del tratamiento, vía de administración, verificación de interacciones medicamentosas, alertas por alergias e impresión de recetas médicas. \\
\hline

M05 & Personal Médico & Gestión de personal asistencial: registro de médicos con CMP y RNE, especialidades y subespecialidades, horarios de atención por consultorio, turnos rotativos, vacaciones y permisos, además de estadísticas de productividad. \\
\hline

\end{tabular}
\end{table}

% ===================== TABLA 2 =====================
\begin{table}[H]
\centering
\label{tab:modulos_part2}

\begin{tabular}{|p{1.2cm}|p{3.5cm}|p{9.5cm}|}
\hline
\tablaheader{ID} & \tablaheader{Módulo} & \tablaheader{Descripción Funcional} \\
\hline

M06 & Inventario Médico & Control de stock de medicamentos e insumos: registro de productos con lote y fecha de vencimiento, movimientos de entrada y salida, kardex valorizado, alertas de stock mínimo, caducidad próxima, consumo por paciente y transferencias entre almacenes. \\
\hline

M07 & Seguridad & Control de acceso mediante RBAC: autenticación segura con hash bcrypt, gestión de usuarios y permisos granulares, perfiles jerárquicos (administrador, médico, enfermería, recepción, farmacia), política de cambio obligatorio de contraseña y sesiones con timeout. \\
\hline

M08 & Auditoría & Trazabilidad completa: registro inmutable de operaciones críticas mediante triggers, auditoría de modificaciones en historias clínicas, log de accesos por usuario, bitácora de cambios con timestamp y alineamiento con normativas del MINSA. \\
\hline

M09 & Reportes y BI & Inteligencia de negocio: dashboards con KPIs en tiempo real, reportes por especialidad, diagnósticos más frecuentes, productividad médica, indicadores de gestión hospitalaria, exportación a Excel/PDF y gráficos interactivos. \\
\hline

M10 & Respaldo & Plan de continuidad del negocio: backups automáticos diarios y semanales, política de retención de 90 días, respaldos diferenciales e incrementales, restauración point-in-time, pruebas de recuperación trimestrales, RPO ≤ 15 min y RTO ≤ 4 h. \\

\hline
\end{tabular}
\end{table}

\newpage

\subsection{Catalago de Requerimientos}
\textbf{Tabla 2}\\[2mm]\textit{Requisitos Funcionales – M01 Gestión de Pacientes}
\begin{table}[H]
\centering
\label{tab:rf_m01}
\begin{tabular}{|p{1.2cm}|p{3.5cm}|p{9.5cm}|}
\hline
\tablaheader{ID RF} & \tablaheader{Módulo} & \tablaheader{Descripción del Requisito Funcional} \\
\hline
RF01 & M01 & Registrar pacientes generando automáticamente un número único de historia clínica, validando la unicidad del DNI. \\
\hline
RF02 & M01 & Modificar datos demográficos del paciente manteniendo historial de cambios mediante auditoría. \\
\hline
RF03 & M01 & Consultar la historia clínica completa del paciente con acceso controlado por roles. \\
\hline
RF04 & M01 & Buscar pacientes mediante múltiples criterios como DNI, nombres, apellidos y número de historia clínica. \\
\hline
\end{tabular}
\end{table}

\textbf{Tabla 3}\\[2mm]\textit{Requisitos Funcionales – M02 Gestión de Citas Médicas}

\begin{table}[H]
\centering
\label{tab:rf_m02}
\begin{tabular}{|p{1.2cm}|p{3.5cm}|p{9.5cm}|}
\hline
\tablaheader{ID RF} & \tablaheader{Módulo} & \tablaheader{Descripción del Requisito Funcional} \\
\hline
RF05 & M02 & Programar citas médicas validando disponibilidad horaria del médico y especialidad. \\
\hline
RF06 & M02 & Reprogramar o cancelar citas registrando el motivo de la operación. \\
\hline
RF07 & M02 & Consultar agenda médica diaria, semanal y mensual por médico y especialidad. \\
\hline
RF08 & M02 & Registrar la asistencia o inasistencia del paciente a la cita médica. \\
\hline
\end{tabular}
\end{table}

\textbf{Tabla 4}\\[2mm]\textit{Requisitos Funcionales – M03 Consultas Médicas}

\begin{table}[H]
\centering
\label{tab:rf_m03}
\begin{tabular}{|p{1.2cm}|p{3.5cm}|p{9.5cm}|}
\hline
\tablaheader{ID RF} & \tablaheader{Módulo} & \tablaheader{Descripción del Requisito Funcional} \\
\hline
RF09 & M03 & Registrar consultas médicas con signos vitales, examen físico y evolución clínica. \\
\hline
RF10 & M03 & Registrar diagnósticos médicos utilizando el estándar CIE-10. \\
\hline
RF11 & M03 & Modificar diagnósticos registrando la justificación médica y auditoría del cambio. \\
\hline
RF12 & M03 & Consultar historial de consultas médicas por paciente. \\
\hline
\end{tabular}
\end{table}

\newpage
\textbf{Tabla 5}\\[2mm]\textit{Requisitos Funcionales – M04 Tratamientos}

\begin{table}[H]
\centering
\label{tab:rf_m04}
\begin{tabular}{|p{1.2cm}|p{3.5cm}|p{9.5cm}|}
\hline
\tablaheader{ID RF} & \tablaheader{Módulo} & \tablaheader{Descripción del Requisito Funcional} \\
\hline
RF13 & M04 & Prescribir tratamientos farmacológicos y no farmacológicos asociados a diagnósticos. \\
\hline
RF14 & M04 & Consultar tratamientos activos del paciente. \\
\hline
RF15 & M04 & Generar recetas médicas electrónicas en formato PDF. \\
\hline
RF16 & M04 & Detectar alergias e interacciones medicamentosas. \\
\hline
\end{tabular}
\end{table}

\textbf{Tabla 6}\\[2mm]\textit{Requisitos Funcionales – M05 Personal Médico}

\begin{table}[H]
\centering
\label{tab:rf_m05}
\begin{tabular}{|p{1.2cm}|p{3.5cm}|p{9.5cm}|}
\hline
\tablaheader{ID RF} & \tablaheader{Módulo} & \tablaheader{Descripción del Requisito Funcional} \\
\hline
RF17 & M05 & Registrar médicos con CMP, RNE y especialidades. \\
\hline
RF18 & M05 & Asignar horarios de atención médica por consultorio. \\
\hline
RF19 & M05 & Gestionar turnos, guardias y permisos médicos. \\
\hline
RF20 & M05 & Consultar estadísticas de productividad médica. \\
\hline
\end{tabular}
\end{table}

\textbf{Tabla 7}\\[2mm]\textit{Requisitos Funcionales – M06 Inventario de Medicamentos}

\begin{table}[H]
\centering
\label{tab:rf_m06}
\begin{tabular}{|p{1.2cm}|p{3.5cm}|p{9.5cm}|}
\hline
\tablaheader{ID RF} & \tablaheader{Módulo} & \tablaheader{Descripción del Requisito Funcional} \\
\hline
RF21 & M06 & Registrar medicamentos en el catálogo institucional con lote, fecha de vencimiento y stock inicial. \\
\hline
RF22 & M06 & Controlar movimientos de entrada y salida de medicamentos mediante kardex valorizado. \\
\hline
RF23 & M06 & Generar alertas automáticas por stock mínimo y medicamentos próximos a vencer. \\
\hline
RF24 & M06 & Consultar el inventario de medicamentos en tiempo real por almacén y categoría. \\
\hline
\end{tabular}
\end{table}

\newpage
\textbf{Tabla 8}\\[2mm]\textit{Requisitos Funcionales – M07 Seguridad y Control de Acceso}

\begin{table}[H]
\centering
\label{tab:rf_m07}
\begin{tabular}{|p{1.2cm}|p{3.5cm}|p{9.5cm}|}
\hline
\tablaheader{ID RF} & \tablaheader{Módulo} & \tablaheader{Descripción del Requisito Funcional} \\
\hline
RF25 & M07 & Autenticar usuarios mediante credenciales seguras y control de sesiones. \\
\hline
RF26 & M07 & Gestionar usuarios del sistema asignando roles y permisos según RBAC. \\
\hline
RF27 & M07 & Restringir el acceso a la información clínica según el rol del usuario. \\
\hline
RF28 & M07 & Registrar intentos fallidos de acceso y bloqueos automáticos por seguridad. \\
\hline
\end{tabular}
\end{table}

\textbf{Tabla 9}\\[2mm]\textit{Requisitos Funcionales – M08 Auditoría y Trazabilidad}

\begin{table}[H]
\centering
\label{tab:rf_m08}
\begin{tabular}{|p{1.2cm}|p{3.5cm}|p{9.5cm}|}
\hline
\tablaheader{ID RF} & \tablaheader{Módulo} & \tablaheader{Descripción del Requisito Funcional} \\
\hline
RF29 & M08 & Registrar automáticamente todas las operaciones críticas realizadas en el sistema. \\
\hline
RF30 & M08 & Auditar modificaciones realizadas sobre historias clínicas y diagnósticos médicos. \\
\hline
RF31 & M08 & Consultar bitácoras de auditoría filtradas por usuario, fecha y tipo de acción. \\
\hline
RF32 & M08 & Garantizar la inmutabilidad de los registros de auditoría para fines legales. \\
\hline
\end{tabular}
\end{table}


\textbf{Tabla 10}\\[2mm]\textit{Requisitos Funcionales – M09 Reportes y Business Intelligence}

\begin{table}[H]
\centering
\label{tab:rf_m09}
\begin{tabular}{|p{1.2cm}|p{3.5cm}|p{9.5cm}|}
\hline
\tablaheader{ID RF} & \tablaheader{Módulo} & \tablaheader{Descripción del Requisito Funcional} \\
\hline
RF33 & M09 & Generar reportes clínicos y administrativos con indicadores clave de desempeño. \\
\hline
RF34 & M09 & Visualizar estadísticas de morbilidad por diagnóstico y especialidad médica. \\
\hline
RF35 & M09 & Exportar reportes en formatos PDF y Excel. \\
\hline
RF36 & M09 & Consultar dashboards interactivos para apoyo a la toma de decisiones gerenciales. \\
\hline
RF37 & M09 & Filtrar reportes por rangos de fechas, servicios y personal médico. \\
\hline
\end{tabular}
\end{table}

\newpage

\textbf{Tabla 11}\\[2mm]\textit{Requisitos Funcionales – M10 Respaldo y Recuperación}

\begin{table}[H]
\centering
\label{tab:rf_m10}
\begin{tabular}{|p{1.2cm}|p{3.5cm}|p{9.5cm}|}
\hline
\tablaheader{ID RF} & \tablaheader{Módulo} & \tablaheader{Descripción del Requisito Funcional} \\
\hline
RF38 & M10 & Ejecutar respaldos automáticos diarios y semanales de la base de datos. \\
\hline
RF39 & M10 & Restaurar la base de datos ante fallos o pérdida de información. \\
\hline
RF40 & M10 & Gestionar políticas de retención de respaldos históricos. \\
\hline
RF41 & M10 & Registrar eventos de respaldo y restauración para auditoría. \\
\hline
RF42 & M10 & Validar la integridad de los respaldos mediante pruebas periódicas de recuperación. \\
\hline
\end{tabular}
\end{table}


% =====================================================
% 3. REQUERIMIENTOS FUNCIONALES
% =====================================================
\newpage

\subsection{Requerimientos Funcionales}
Los requerimientos funcionales (RF) describen las capacidades que el sistema debe proporcionar. Se han identificado \textbf{42 requerimientos funcionales} organizados por módulo.

\subsubsection{Módulo M01: Gestión de Pacientes}

\textbf{Tabla 12}\\[2mm]\textit{RF-001: Registrar Nuevo Paciente}



\begin{table}[H]
\small
\begin{tabular}{|p{3cm}|p{11cm}|}
\hline
\tablaheader{Atributo} & \tablaheader{Descripción} \\
\hline
\textbf{Prioridad} & Crítica \\
\hline
\textbf{Descripción} & El sistema debe permitir registrar nuevos pacientes con generación automática de número de historia clínica única alfanumérica de 15 caracteres. \\
\hline
\textbf{Entradas} & DNI (8 dígitos), Nombres, Apellidos, Fecha nacimiento, Sexo (M/F), Dirección, Teléfono, Email, Grupo sanguíneo, Antecedentes médicos, Alergias. \\
\hline
\textbf{Procesamiento} & 1. Validar formato DNI con dígito verificador\newline 2. Verificar unicidad de DNI\newline 3. Generar historia clínica HC-YYYY-NNNNN\newline 4. Calcular edad automáticamente\newline 5. Asignar estado ``Activo''\newline 6. Registrar timestamp \\
\hline
\textbf{Salidas} & Paciente creado con ID único, número de historia generado, confirmación, opción imprimir carnet. \\
\hline
\textbf{Postcondición} & Registro en tabla Pacientes, auditoría en AuditLog, historia disponible. \\
\hline
\end{tabular}
\\[2mm]\\[2mm]\textit{Nota.} La siguiente tabla representa el RF-001: Registrar Nuevo Paciente

\end{table}
\newpage
\textbf{Tabla 13}\\[2mm]\textit{RF-002: Modificar Datos de Paciente}

\begin{table}[H]
\small
\begin{tabular}{|p{3cm}|p{11cm}|}
\hline
\tablaheader{Atributo} & \tablaheader{Descripción} \\
\hline
\textbf{Prioridad} & Alta \\
\hline
\textbf{Descripción} & Actualizar datos demográficos del paciente excepto número de historia clínica (inmutable). \\
\hline
\textbf{Procesamiento} & Validar permisos, registrar valores anteriores en auditoría, actualizar campos modificables (dirección, teléfono, email), proteger campos inmutables (NroHistoria, DNI). \\
\hline
\textbf{Postcondición} & Datos actualizados, historial completo en Auditoria\_Pacientes, trazabilidad total. \\
\hline
\end{tabular}
\\[2mm]\textit{Nota.} La siguiente tabla representa el RF-002: Modificar Datos de Paciente}
\end{table}

\textbf{Tabla 14}\\[2mm]\textit{RF-003: Buscar Paciente con Filtros Múltiples}

\begin{table}[H]
\small
\begin{tabular}{|p{3cm}|p{11cm}|}
\hline
\tablaheader{Atributo} & \tablaheader{Descripción} \\
\hline
\textbf{Prioridad} & Crítica \\
\hline
\textbf{Descripción} & Búsqueda rápida de pacientes con múltiples criterios simultáneos. \\
\hline
\textbf{Criterios} & DNI (exacto), Historia clínica (exacto), Nombres/Apellidos (LIKE), Rango fechas nacimiento, Sexo, Estado. \\
\hline
\textbf{Salidas} & Lista paginada (20 por página), tiempo respuesta menor a 2 segundos, datos: Foto, Nro Historia, Nombre, DNI, Edad, Última consulta. \\
\hline
\end{tabular}
\\[2mm]\textit{Nota.} La siguiente tabla representa el RF-003: Buscar Paciente}
\end{table}

\textbf{Tabla 15}\\[2mm]\textit{RF-004: Consultar Historia Clínica Completa}

\begin{table}[H]
\small
\begin{tabular}{|p{3cm}|p{11cm}|}
\hline
\tablaheader{Atributo} & \tablaheader{Descripción} \\
\hline
\textbf{Prioridad} & Crítica \\
\hline
\textbf{Descripción} & Visualizar historia clínica completa con consultas, diagnósticos, tratamientos, alergias y antecedentes. \\
\hline
\textbf{Información} & Datos demográficos, Antecedentes familiares/personales, Alergias (destacadas en rojo), Timeline de consultas, Diagnósticos por fecha, Tratamientos activos/históricos, Exámenes auxiliares. \\
\hline
\textbf{Salidas} & Vista completa organizada cronológicamente, exportable a PDF, con gráficos de evolución de signos vitales. \\
\hline
\end{tabular}
\\[2mm]\textit{Nota.} La siguiente tabla representa el RF-004: Consultar Historia Clínica}
\end{table}
\newpage
\subsubsection{Módulo M02: Gestión de Citas Médicas}

\textbf{Tabla 16}\\[2mm]\textit{RF-005: Programar Nueva Cita Médica}

\begin{table}[H]
\small
\begin{tabular}{|p{3cm}|p{11cm}|}
\hline
\tablaheader{Atributo} & \tablaheader{Descripción} \\
\hline
\textbf{Prioridad} & Crítica \\
\hline
\textbf{Descripción} & Programar citas validando disponibilidad de médico y evitando cruces horarios. \\
\hline
\textbf{Entradas} & ID Paciente, ID Médico, Especialidad, Fecha/hora, Motivo consulta, Tipo (primera vez/control/emergencia). \\
\hline
\textbf{Validaciones} & Paciente activo, Médico con horario disponible, Sin cruces horarios, Dentro de turno laboral, Duración 30 min. \\
\hline
\textbf{Salidas} & Cita registrada con código CITA-YYYY-NNNNNN, estado ``Programada'', notificación SMS/Email al paciente. \\
\hline
\end{tabular}
\\[2mm]\textit{Nota.} La siguiente tabla representa el RF-005: Programar Cita Médica}
\end{table}
\textbf{Tabla 17}\\[2mm]\textit{RF-006: Cancelar o Reprogramar Cita}

\begin{table}[H]
\small
\begin{tabular}{|p{3cm}|p{11cm}|}
\hline
\tablaheader{Atributo} & \tablaheader{Descripción} \\
\hline
\textbf{Prioridad} & Alta \\
\hline
\textbf{Descripción} & Cancelar o reprogramar citas con registro obligatorio de motivo. \\
\hline
\textbf{Procesamiento} & Cancelación: estado ``Cancelada'', motivo obligatorio, liberar agenda. Reprogramación: crear nueva cita vinculada, marcar original como ``Reprogramada''. \\
\hline
\textbf{Postcondición} & Estado actualizado, notificación enviada, auditoría registrada. \\
\hline
\end{tabular}
\\[2mm]\textit{Nota.} La siguiente tabla representa el RF-006: Cancelar/Reprogramar Cita}
\end{table}

\textbf{Tabla 18}\\[2mm]\textit{RF-007: Confirmar Asistencia a Cita}

\begin{table}[H]
\small
\begin{tabular}{|p{3cm}|p{11cm}|}
\hline
\tablaheader{Atributo} & \tablaheader{Descripción} \\
\hline
\textbf{Prioridad} & Media \\
\hline
\textbf{Descripción} & Confirmar asistencia del paciente cuando llega a recepción. \\
\hline
\textbf{Procesamiento} & Cambiar estado de ``Programada'' a ``Confirmada'', registrar hora real de llegada, actualizar lista de espera del médico. \\
\hline
\textbf{Salidas} & Estado confirmado, paciente en cola de espera del médico, notificación al médico. \\
\hline
\end{tabular}
\\[2mm]\textit{Nota.} La siguiente tabla representa el RF-007: Confirmar Asistencia}
\end{table}
\newpage
\textbf{Tabla 19}\\[2mm]\textit{RF-008: Consultar Agenda Médica por Fecha}

\begin{table}[H]
\small
\begin{tabular}{|p{3cm}|p{11cm}|}
\hline
\tablaheader{Atributo} & \tablaheader{Descripción} \\
\hline
\textbf{Prioridad} & Alta \\
\hline
\textbf{Descripción} & Visualizar agenda completa del médico con todas las citas del día. \\
\hline
\textbf{Filtros} & Por médico, Por fecha, Por especialidad, Por estado de cita. \\
\hline
\textbf{Salidas} & Vista calendario con slots de 30 minutos, citas confirmadas en verde, programadas en amarillo, espacios libres en blanco. \\
\hline
\end{tabular}
\\[2mm]\textit{Nota.} La siguiente tabla representa el RF-008: Consultar Agenda Médica}
\end{table}

\subsubsection{Módulo M03: Consultas y Diagnósticos}

\textbf{Tabla 20}\\[2mm]\textit{RF-009: Registrar Consulta Médica}

\begin{table}[H]
\small
\begin{tabular}{|p{3cm}|p{11cm}|}
\hline
\tablaheader{Atributo} & \tablaheader{Descripción} \\
\hline
\textbf{Prioridad} & Crítica \\
\hline
\textbf{Descripción} & Registrar consulta médica completa durante atención del paciente. \\
\hline
\textbf{Signos Vitales} & Presión arterial (mmHg), Temperatura (°C), Frecuencia cardíaca/respiratoria, Peso (kg), Talla (m), IMC (calculado), Saturación O2 (\%). \\
\hline
\textbf{Anamnesis} & Motivo consulta, Tiempo enfermedad, Síntomas, Relato cronológico, Examen físico, Plan de trabajo. \\
\hline
\textbf{Postcondición} & Consulta en tabla Consultas, cita cambia a estado ``Atendida'', datos disponibles para diagnóstico. \\
\hline
\end{tabular}
\\[2mm]\textit{Nota.} La siguiente tabla representa el RF-009: Registrar Consulta Médica}
\end{table}

\textbf{Tabla 21}\\[2mm]\textit{RF-010: Registrar Diagnóstico CIE-10}

\begin{table}[H]
\small
\begin{tabular}{|p{3cm}|p{11cm}|}
\hline
\tablaheader{Atributo} & \tablaheader{Descripción} \\
\hline
\textbf{Prioridad} & Crítica \\
\hline
\textbf{Descripción} & Registrar diagnósticos con Clasificación Internacional de Enfermedades CIE-10 OMS. \\
\hline
\textbf{Entradas} & Código CIE-10 (A00-Z99), Descripción (autocompletado), Tipo (presuntivo/definitivo), Clasificación (principal/secundario/complicación). \\
\hline
\textbf{Validación} & Código CIE-10 válido en catálogo, formato correcto, búsqueda por palabra clave. \\
\hline
\textbf{Salidas} & Diagnóstico registrado, estadísticas morbilidad actualizadas, alerta si es notificación obligatoria MINSA. \\
\hline
\end{tabular}
\\[2mm]\textit{Nota.} La siguiente tabla representa el RF-010: Registrar Diagnóstico CIE-10}
\end{table}

\textbf{Tabla 22}\\[2mm]\textit{RF-011: Modificar Diagnóstico con Justificación}

\begin{table}[H]
\small
\begin{tabular}{|p{3cm}|p{11cm}|}
\hline
\tablaheader{Atributo} & \tablaheader{Descripción} \\
\hline
\textbf{Prioridad} & Alta \\
\hline
\textbf{Descripción} & Permitir modificación de diagnósticos con justificación médica obligatoria (implicancia legal). \\
\hline
\textbf{Procesamiento} & Validar rol Médico, solicitar justificación detallada (mínimo 50 caracteres), registrar valores anteriores en Auditoria\_Diagnosticos, mantener trazabilidad completa. \\
\hline
\textbf{Postcondición} & Diagnóstico actualizado, justificación registrada, auditoría inmutable. \\
\hline
\end{tabular}
\\[2mm]\textit{Nota.} La siguiente tabla representa el RF-011: Modificar Diagnóstico}
\end{table}

\textbf{Tabla 23}\\[2mm]\textit{RF-012: Buscar Diagnósticos por CIE-10}

\begin{table}[H]
\small
\begin{tabular}{|p{3cm}|p{11cm}|}
\hline
\tablaheader{Atributo} & \tablaheader{Descripción} \\
\hline
\textbf{Prioridad} & Media \\
\hline
\textbf{Descripción} & Búsqueda inteligente en catálogo CIE-10 con 14,000+ códigos usando FULLTEXT INDEX. \\
\hline
\textbf{Criterios} & Por código exacto (J06.9), Por descripción (``respiratorias''), Por capítulo (I-XXII), Por sistema (cardiovascular, respiratorio). \\
\hline
\textbf{Salidas} & Lista de códigos coincidentes con descripción, capítulo, restricciones de edad/sexo. \\
\hline
\end{tabular}
\\[2mm]\textit{Nota.} La siguiente tabla representa el RF-012: Buscar CIE-10}
\end{table}

\subsubsection{Módulo M04: Tratamientos}

\textbf{Tabla 24}\\[2mm]\textit{RF-013: Prescribir Tratamiento Médico}

\begin{table}[H]
\small
\begin{tabular}{|p{3cm}|p{11cm}|}
\hline
\tablaheader{Atributo} & \tablaheader{Descripción} \\
\hline
\textbf{Prioridad} & Crítica \\
\hline
\textbf{Descripción} & Prescribir tratamientos farmacológicos asociados al diagnóstico con validaciones de seguridad. \\
\hline
\textbf{Datos} & Medicamento (desde catálogo), Dosis/presentación, Frecuencia (cada 8h), Vía administración (oral, IV, IM, tópica), Duración (días), Indicaciones especiales. \\
\hline
\textbf{Alertas} & Alergias del paciente, Interacciones medicamentosas, Sin stock en farmacia, Sugerencia genéricos alternativos. \\
\hline
\textbf{Salidas} & Tratamiento registrado, receta imprimible con código barras, solicitud automática a farmacia. \\
\hline
\end{tabular}
\\[2mm]\textit{Nota.} La siguiente tabla representa el RF-013: Prescribir Tratamiento}
\end{table}

\textbf{Tabla 25}\\[2mm]\textit{RF-014: Consultar Tratamientos Activos del Paciente}

\begin{table}[H]
\small
\begin{tabular}{|p{3cm}|p{11cm}|}
\hline
\tablaheader{Atributo} & \tablaheader{Descripción} \\
\hline
\textbf{Prioridad} & Alta \\
\hline
\textbf{Descripción} & Visualizar todos los tratamientos activos actuales del paciente para evitar interacciones. \\
\hline
\textbf{Información} & Medicamentos actuales, Dosis y frecuencias, Fechas inicio/fin, Médico prescriptor, Alerta de proximidad a fin de tratamiento. \\
\hline
\textbf{Salidas} & Lista organizada por fecha, medicamentos próximos a terminar destacados, opción renovar receta. \\
\hline
\end{tabular}
\\[2mm]\textit{Nota.} La siguiente tabla representa el RF-014: Tratamientos Activos}
\end{table}

\textbf{Tabla 26}\\[2mm]\textit{RF-015: Imprimir Receta Médica}

\begin{table}[H]
\small
\begin{tabular}{|p{3cm}|p{11cm}|}
\hline
\tablaheader{Atributo} & \tablaheader{Descripción} \\
\hline
\textbf{Prioridad} & Alta \\
\hline
\textbf{Descripción} & Generar e imprimir receta médica oficial con todos los requisitos legales MINSA. \\
\hline
\textbf{Contenido} & Logo hospital, Datos paciente, CMP médico, Fecha/hora, Medicamentos con dosis detalladas, Firma digital, Código QR verificación, Vigencia de receta. \\
\hline
\textbf{Salidas} & PDF imprimible formato A5, almacenado en sistema, código QR para validación en farmacia. \\
\hline
\end{tabular}
\\[2mm]\textit{Nota.} La siguiente tabla representa el RF-015: Imprimir Receta}
\end{table}

\textbf{Tabla 27}\\[2mm]\textit{RF-016: Registrar Alerta de Interacción Medicamentosa}

\begin{table}[H]
\small
\begin{tabular}{|p{3cm}|p{11cm}|}
\hline
\tablaheader{Atributo} & \tablaheader{Descripción} \\
\hline
\textbf{Prioridad} & Crítica \\
\hline
\textbf{Descripción} & Detectar y alertar sobre interacciones medicamentosas peligrosas en tiempo real. \\
\hline
\textbf{Validación} & Comparar nuevo medicamento con tratamientos activos, consultar tabla de interacciones conocidas, validar contra alergias registradas. \\
\hline
\textbf{Alertas} & Nivel Crítico (bloquea prescripción), Nivel Alto (advertencia requiere confirmación), Nivel Moderado (información). \\
\hline
\end{tabular}
\\[2mm]\textit{Nota.} La siguiente tabla representa el RF-016: Alerta Interacciones}
\end{table}
\newpage
\subsubsection{Módulo M05: Personal Médico}

\textbf{Tabla 28}\\[2mm]\textit{RF-017: Registrar Médico con Credenciales}

\begin{table}[H]
\small
\begin{tabular}{|p{3cm}|p{11cm}|}
\hline
\tablaheader{Atributo} & \tablaheader{Descripción} \\
\hline
\textbf{Prioridad} & Alta \\
\hline
\textbf{Descripción} & Registrar personal médico con validación de CMP y RNE vigentes. \\
\hline
\textbf{Datos} & DNI, Nombres/Apellidos, CMP (Colegio Médico Perú - 6 dígitos), RNE (Registro Nacional Especialistas), Especialidad principal, Subespecialidad, Teléfono, Email institucional, Fecha ingreso. \\
\hline
\textbf{Validación} & CMP único (no duplicado), formato válido CMP, especialidad debe existir en catálogo. \\
\hline
\textbf{Postcondición} & Médico registrado, disponible para asignación de citas y horarios. \\
\hline
\end{tabular}
\\[2mm]\textit{Nota.} La siguiente tabla representa el RF-017: Registrar Médico}
\end{table}


\textbf{Tabla 29}\\[2mm]\textit{RF-018: Asignar Horarios de Atención}

\begin{table}[H]
\small
\begin{tabular}{|p{3cm}|p{11cm}|}
\hline
\tablaheader{Atributo} & \tablaheader{Descripción} \\
\hline
\textbf{Prioridad} & Alta \\
\hline
\textbf{Descripción} & Configurar horarios de atención del médico por día, turno y consultorio. \\
\hline
\textbf{Datos} & Médico, Día semana (Lunes-Domingo), Hora inicio/fin, Consultorio, Turno (mañana/tarde/noche), Duración slot (30 min). \\
\hline
\textbf{Validación} & Sin cruces horarios del mismo médico, consultorio disponible en ese horario, hora fin mayor a hora inicio. \\
\hline
\textbf{Postcondición} & Horario registrado en HorariosAtencion, disponible para programación de citas. \\
\hline
\end{tabular}
\\[2mm]\textit{Nota.} La siguiente tabla representa el RF-018: Horarios Atención}
\end{table}

\textbf{Tabla 30}\\[2mm]\textit{RF-019: Gestionar Turnos y Guardias}

\begin{table}[H]
\small
\begin{tabular}{|p{3cm}|p{11cm}|}
\hline
\tablaheader{Atributo} & \tablaheader{Descripción} \\
\hline
\textbf{Prioridad} & Media \\
\hline
\textbf{Descripción} & Administrar turnos rotativos y guardias médicas con notificaciones automáticas. \\
\hline
\textbf{Funciones} & Asignar guardias (12h o 24h), Generar rol de turnos mensual, Registrar cambios de turno con médico reemplazante, Alertar turnos próximos. \\
\hline
\textbf{Salidas} & Calendario de guardias, notificación 24h antes, reporte de horas trabajadas. \\
\hline
\end{tabular}
\\[2mm]\textit{Nota.} La siguiente tabla representa el RF-019: Turnos y Guardias}
\end{table}

\textbf{Tabla 31}\\[2mm]\textit{RF-020: Consultar Estadísticas de Productividad}

\begin{table}[H]
\small
\begin{tabular}{|p{3cm}|p{11cm}|}
\hline
\tablaheader{Atributo} & \tablaheader{Descripción} \\
\hline
\textbf{Prioridad} & Media \\
\hline
\textbf{Descripción} & Generar reportes de productividad médica con KPIs operativos. \\
\hline
\textbf{Métricas} & Número de consultas/mes, Promedio duración consulta, Tasa de ausentismo pacientes, Diagnósticos más frecuentes, Satisfacción pacientes. \\
\hline
\textbf{Salidas} & Dashboard con gráficos, exportable a Excel/PDF, comparativa entre médicos. \\
\hline
\end{tabular}
\\[2mm]\textit{Nota.} La siguiente tabla representa el RF-020: Estadísticas Productividad}
\end{table}

\subsubsection{Módulo M06: Inventario de Medicamentos}

\textbf{Tabla 32}\\[2mm]\textit{RF-021: Registrar Medicamento en Catálogo}

\begin{table}[H]
\small
\begin{tabular}{|p{3cm}|p{11cm}|}
\hline
\tablaheader{Atributo} & \tablaheader{Descripción} \\
\hline
\textbf{Prioridad} & Alta \\
\hline
\textbf{Descripción} & Agregar nuevos medicamentos al petitorio farmacológico del hospital. \\
\hline
\textbf{Datos} & Código interno, Nombre genérico (DCI), Nombre comercial, Presentación (tabletas, jarabe), Concentración (500mg), Forma farmacéutica, Unidad medida, Stock mínimo, Precio unitario, Requiere receta (Sí/No). \\
\hline
\textbf{Validación} & Código único, nombre genérico único, precio mayor a cero. \\
\hline
\textbf{Postcondición} & Medicamento disponible para prescripción y movimientos de inventario. \\
\hline
\end{tabular}
\\[2mm]\textit{Nota.} La siguiente tabla representa el RF-021: Registrar Medicamento}
\end{table}


\textbf{Tabla 33}\\[2mm]\textit{RF-022: Registrar Entrada de Inventario}

\begin{table}[H]
\small
\begin{tabular}{|p{3cm}|p{11cm}|}
\hline
\tablaheader{Atributo} & \tablaheader{Descripción} \\
\hline
\textbf{Prioridad} & Crítica \\
\hline
\textbf{Descripción} & Registrar ingreso de medicamentos con control de lotes y vencimientos (DIGEMID). \\
\hline
\textbf{Datos} & Medicamento, Cantidad, Lote fabricante, Fecha fabricación/vencimiento, Proveedor, Documento referencia (factura), Almacén destino. \\
\hline
\textbf{Procesamiento} & Crear registro en Inventario Movimientos (tipo ``Entrada''), crear/actualizar lote en tabla Lotes, actualizar StockActual con trigger automático. \\
\hline
\textbf{Postcondición} & Stock actualizado, lote registrado, movimiento en kardex valorizado. \\
\hline
\end{tabular}
\\[2mm]\textit{Nota.} La siguiente tabla representa el RF-022: Entrada Inventario}
\end{table}

\textbf{Tabla 34}\\[2mm]\textit{RF-023: Registrar Salida de Inventario}

\begin{table}[H]
\small
\begin{tabular}{|p{3cm}|p{11cm}|}
\hline
\tablaheader{Atributo} & \tablaheader{Descripción} \\
\hline
\textbf{Prioridad} & Crítica \\
\hline
\textbf{Descripción} & Registrar dispensación de medicamentos por prescripción médica o emergencia. \\
\hline
\textbf{Datos} & Medicamento, Cantidad, Lote (FIFO - primero en expirar), Motivo (prescripción/emergencia), Documento referencia (receta médica), Paciente destino. \\
\hline
\textbf{Validación} & Stock suficiente, lote no vencido, receta médica válida si requiere. \\
\hline
\textbf{Procesamiento} & Registro en Inventario Movimientos (tipo ``Salida''), descontar de lote más antiguo (FIFO), actualizar StockActual, alertar si llega a stock mínimo. \\
\hline
\end{tabular}
\\[2mm]\textit{Nota.} La siguiente tabla representa el RF-023: Salida Inventario}
\end{table}

\textbf{Tabla 35}\\[2mm]\textit{RF-024: Generar Alerta de Stock Mínimo}

\begin{table}[H]
\small
\begin{tabular}{|p{3cm}|p{11cm}|}
\hline
\tablaheader{Atributo} & \tablaheader{Descripción} \\
\hline
\textbf{Prioridad} & Alta \\
\hline
\textbf{Descripción} & Alertar automáticamente cuando medicamentos lleguen al stock mínimo configurado. \\
\hline
\textbf{Procesamiento} & Consulta diaria (job programado) de medicamentos con StockActual menor o igual a StockMinimo, generar reporte, enviar email a jefe de farmacia, marcar como urgente en dashboard. \\
\hline
\textbf{Salidas} & Lista de medicamentos críticos, cantidad faltante, proveedor sugerido, tiempo promedio de reposición. \\
\hline
\end{tabular}
\\[2mm]\textit{Nota.} La siguiente tabla representa el RF-024: Alerta Stock Mínimo}
\end{table}
\newpage
\textbf{Tabla 36}\\[2mm]\textit{RF-025: Controlar Fecha de Caducidad}

\begin{table}[H]
\small
\begin{tabular}{|p{3cm}|p{11cm}|}
\hline
\tablaheader{Atributo} & \tablaheader{Descripción} \\
\hline
\textbf{Prioridad} & Crítica \\
\hline
\textbf{Descripción} & Monitorear y alertar sobre medicamentos próximos a vencer (90, 60, 30 días). \\
\hline
\textbf{Procesamiento} & Job diario compara FechaVencimiento con GETDATE(), alerta 90 días antes (amarillo), 60 días (naranja), 30 días (rojo), medicamento vencido no se puede dispensar (bloqueo). \\
\hline
\textbf{Salidas} & Dashboard con medicamentos por vencer, reporte de mermas por vencimiento, proceso de destrucción DIGEMID. \\
\hline
\end{tabular}
\\[2mm]\textit{Nota.} La siguiente tabla representa el RF-025: Control Caducidad}
\end{table}
\
\textbf{Tabla 37}\\[2mm]\textit{RF-026: Generar Kardex Valorizado}

\begin{table}[H]
\small
\begin{tabular}{|p{3cm}|p{11cm}|}
\hline
\tablaheader{Atributo} & \tablaheader{Descripción} \\
\hline
\textbf{Prioridad} & Media \\
\hline
\textbf{Descripción} & Generar kardex con detalle de movimientos de inventario con valorización económica. \\
\hline
\textbf{Filtros} & Por medicamento, Por rango fechas, Por tipo movimiento (entrada/salida/ajuste), Por almacén. \\
\hline
\textbf{Salidas} & Reporte con: Fecha, Tipo, Documento, Entrada, Salida, Saldo físico, Valor unitario, Valor total, exportable a Excel. \\
\hline
\end{tabular}
\\[2mm]\textit{Nota.} La siguiente tabla representa el RF-026: Kardex Valorizado}
\end{table}

\subsubsection{Módulo M07: Seguridad y Control de Acceso}

\textbf{Tabla 38}\\[2mm]\textit{RF-027: Autenticación de Usuario}

\begin{table}[H]
\small
\begin{tabular}{|p{3cm}|p{11cm}|}
\hline
\tablaheader{Atributo} & \tablaheader{Descripción} \\
\hline
\textbf{Prioridad} & Crítica \\
\hline
\textbf{Descripción} & Autenticar usuarios con credenciales seguras (hash bcrypt + salt único). \\
\hline
\textbf{Procesamiento} & Recibir usuario/contraseña, buscar usuario en tabla Usuarios, comparar hash con bcrypt\_verify (cost factor 12), validar estado activo, validar cuenta no bloqueada, registrar intento en log. \\
\hline
\textbf{Seguridad} & Máximo 5 intentos fallidos (bloqueo automático), contraseñas NUNCA en texto plano, timeout de sesión 30 minutos inactividad. \\
\hline
\textbf{Postcondición} & Sesión creada con token JWT, permisos cargados según rol, último acceso actualizado. \\
\hline
\end{tabular}
\\[2mm]\textit{Nota.} La siguiente tabla representa el RF-027: Autenticación}
\end{table}
\newpage

\textbf{Tabla 39}\\[2mm]\textit{RF-028: Gestión de Roles RBAC}

\begin{table}[H]
\small
\begin{tabular}{|p{3cm}|p{11cm}|}
\hline
\tablaheader{Atributo} & \tablaheader{Descripción} \\
\hline
\textbf{Prioridad} & Alta \\
\hline
\textbf{Descripción} & Administrar roles con permisos granulares CRUD (Create, Read, Update, Delete). \\
\hline
\textbf{Roles Sistema} & Administrador (todos los permisos), Médico (consultas, diagnósticos, tratamientos), Enfermera (citas, signos vitales), Recepcionista (pacientes, citas), Farmacia (inventario, dispensación), Auditor (solo lectura). \\
\hline
\textbf{Funciones} & Crear rol, asignar permisos por módulo, asignar usuarios al rol, jerarquía de roles (Nivel 1-6). \\
\hline
\end{tabular}
\\[2mm]\textit{Nota.} La siguiente tabla representa el RF-028: Roles RBAC}
\end{table}

\textbf{Tabla 40}\\[2mm]\textit{RF-029: Cambio de Contraseña Obligatorio}

\begin{table}[H]
\small
\begin{tabular}{|p{3cm}|p{11cm}|}
\hline
\tablaheader{Atributo} & \tablaheader{Descripción} \\
\hline
\textbf{Prioridad} & Alta \\
\hline
\textbf{Descripción} & Forzar cambio de contraseña en primer login o cada 90 días. \\
\hline
\textbf{Validación} & Mínimo 8 caracteres, 1 mayúscula, 1 minúscula, 1 número, 1 carácter especial, no puede ser igual a últimas 5 contraseñas. \\
\hline
\textbf{Procesamiento} & Verificar flag CambioPasswordObligatorio, mostrar formulario cambio, validar fortaleza, hashear con bcrypt, actualizar PasswordHash y PasswordSalt. \\
\hline
\end{tabular}
\\[2mm]\textit{Nota.} La siguiente tabla representa el RF-029: Cambio Password}
\end{table}

\textbf{Tabla 41}\\[2mm]\textit{RF-030: Cierre Automático de Sesión}

\begin{table}[H]
\small
\begin{tabular}{|p{3cm}|p{11cm}|}
\hline
\tablaheader{Atributo} & \tablaheader{Descripción} \\
\hline
\textbf{Prioridad} & Media \\
\hline
\textbf{Descripción} & Cerrar sesión automáticamente después de 30 minutos de inactividad. \\
\hline
\textbf{Procesamiento} & Registrar timestamp de última actividad, comparar con tiempo actual, si diferencia mayor a 30 min destruir sesión, invalidar token JWT, registrar cierre en log. \\
\hline
\textbf{Seguridad} & Prevenir acceso no autorizado si usuario deja computadora desatendida. \\
\hline
\end{tabular}
\\[2mm]\textit{Nota.} La siguiente tabla representa el RF-030: Timeout Sesión}
\end{table}

\newpage
\subsubsection{Módulo M08: Auditoría y Trazabilidad}

\textbf{Tabla 42}\\[2mm]\textit{RF-031: Auditoría Automática de Cambios}

\begin{table}[H]
\small
\begin{tabular}{|p{3cm}|p{11cm}|}
\hline
\tablaheader{Atributo} & \tablaheader{Descripción} \\
\hline
\textbf{Prioridad} & Crítica \\
\hline
\textbf{Descripción} & Registrar automáticamente todos los cambios en tablas críticas mediante triggers. \\
\hline
\textbf{Tablas Auditadas} & Pacientes, Consultas, Diagnosticos, Tratamientos, Usuarios, Inventario Movimientos. \\
\hline
\textbf{Datos Capturados} & Tabla afectada, Operación (INSERT/UPDATE/DELETE), ID registro, Usuario (ID y nombre), Fecha/hora exacta, Valores anteriores (JSON), Valores nuevos (JSON), IP origen, Nombre PC. \\
\hline
\textbf{Implementación} & Triggers AFTER INSERT, AFTER UPDATE, AFTER DELETE en cada tabla, inserción en AuditLog inmutable. \\
\hline
\end{tabular}
\\[2mm]\textit{Nota.} La siguiente tabla representa el RF-031: Auditoría Automática}
\end{table}

\textbf{Tabla 43}\\[2mm]\textit{RF-032: Consultar Logs de Auditoría}

\begin{table}[H]
\small
\begin{tabular}{|p{3cm}|p{11cm}|}
\hline
\tablaheader{Atributo} & \tablaheader{Descripción} \\
\hline
\textbf{Prioridad} & Alta \\
\hline
\textbf{Descripción} & Consultar registros de auditoría con filtros múltiples para investigaciones. \\
\hline
\textbf{Filtros} & Por usuario, Por tabla afectada, Por tipo operación, Por rango fechas, Por ID registro, Por dirección IP. \\
\hline
\textbf{Salidas} & Lista paginada de cambios, detalle valores antes/después, usuario responsable, posibilidad de exportar a PDF. \\
\hline
\textbf{Restricción} & Solo rol Auditor y Administrador pueden consultar logs completos. \\
\hline
\end{tabular}
\\[2mm]\textit{Nota.} La siguiente tabla representa el RF-032: Consultar Auditoría}
\end{table}

\textbf{Tabla 44}\\[2mm]\textit{RF-033: Generar Reporte de Auditoría por Usuario}

\begin{table}[H]
\small
\begin{tabular}{|p{3cm}|p{11cm}|}
\hline
\tablaheader{Atributo} & \tablaheader{Descripción} \\
\hline
\textbf{Prioridad} & Media \\
\hline
\textbf{Descripción} & Generar reporte detallado de todas las operaciones realizadas por un usuario específico. \\
\hline
\textbf{Información} & Usuario, Rol, Rango fechas, Total operaciones, Operaciones por tipo (INSERT/UPDATE/DELETE), Tablas más modificadas, Horarios de actividad, Intentos acceso fallidos. \\
\hline
\textbf{Salidas} & Reporte PDF con gráficos, timeline de actividad, exportable a Excel. \\
\hline
\end{tabular}
\\[2mm]\textit{Nota.} La siguiente tabla representa el RF-033: Reporte Auditoría Usuario}
\end{table}
\newpage
\textbf{Tabla 45}\\[2mm]\textit{RF-034: Auditoría de Acceso a Datos Sensibles}

\begin{table}[H]
\small
\begin{tabular}{|p{3cm}|p{11cm}|}
\hline
\tablaheader{Atributo} & \tablaheader{Descripción} \\
\hline
\textbf{Prioridad} & Crítica \\
\hline
\textbf{Descripción} & Registrar todos los accesos a datos sensibles (historias clínicas) para cumplir Ley 29733. \\
\hline
\textbf{Datos Sensibles} & Historia clínica completa, Diagnósticos específicos, Tratamientos, Alergias, Antecedentes médicos. \\
\hline
\textbf{Procesamiento} & Cada consulta a vista de historia clínica inserta registro en log, captura usuario, paciente consultado, motivo acceso, fecha/hora. \\
\hline
\textbf{Postcondición} & Trazabilidad completa de quién accedió a datos de cada paciente. \\
\hline
\end{tabular}
\\[2mm]\textit{Nota.} La siguiente tabla representa el RF-034: Auditoría Datos Sensibles}
\end{table}

\subsubsection{Módulo M09: Reportes y Business Intelligence}

\textbf{Tabla 46}\\[2mm]\textit{RF-035: Dashboard con KPIs en Tiempo Real}

\begin{table}[H]
\small
\begin{tabular}{|p{3cm}|p{11cm}|}
\hline
\tablaheader{Atributo} & \tablaheader{Descripción} \\
\hline
\textbf{Prioridad} & Alta \\
\hline
\textbf{Descripción} & Dashboard ejecutivo con indicadores clave de gestión hospitalaria actualizados en tiempo real. \\
\hline
\textbf{KPIs} & Total pacientes activos, Citas programadas hoy, Citas atendidas/pendientes, Tasa ausentismo, Ocupación de consultorios, Medicamentos en stock crítico, Diagnósticos más frecuentes (top 10), Productividad médica. \\
\hline
\textbf{Visualización} & Gráficos de barras, líneas, donas, medidores, actualización automática cada 5 minutos. \\
\hline
\textbf{Filtros} & Por rango fechas, Por especialidad, Por médico, Por tipo de cita. \\
\hline
\end{tabular}
\\[2mm]\textit{Nota.} La siguiente tabla representa el RF-035: Dashboard KPIs}
\end{table}
\newpage
\textbf{Tabla 47}\\[2mm]\textit{RF-036: Reporte de Citas por Especialidad}

\begin{table}[H]
\small
\begin{tabular}{|p{3cm}|p{11cm}|}
\hline
\tablaheader{Atributo} & \tablaheader{Descripción} \\
\hline
\textbf{Prioridad} & Media \\
\hline
\textbf{Descripción} & Generar reporte estadístico de citas médicas agrupadas por especialidad. \\
\hline
\textbf{Información} & Especialidad, Total citas programadas, Citas atendidas, Citas canceladas, Tasa ausentismo, Promedio citas por día, Comparativa mensual. \\
\hline
\textbf{Filtros} & Por rango fechas, Por estado de cita, Por médico específico. \\
\hline
\textbf{Salidas} & Tabla resumen, gráfico de barras comparativo, exportable a Excel/PDF. \\
\hline
\end{tabular}
\\[2mm]\textit{Nota.} La siguiente tabla representa el RF-036: Reporte Citas}
\end{table}

\textbf{Tabla 48}\\[2mm]\textit{RF-037: Estadísticas de Diagnósticos Frecuentes}

\begin{table}[H]
\small
\begin{tabular}{|p{3cm}|p{11cm}|}
\hline
\tablaheader{Atributo} & \tablaheader{Descripción} \\
\hline
\textbf{Prioridad} & Alta \\
\hline
\textbf{Descripción} & Generar reporte epidemiológico con diagnósticos CIE-10 más frecuentes para reportes MINSA. \\
\hline
\textbf{Información} & Top 20 diagnósticos, Código CIE-10, Descripción, Total casos, Tendencia mensual, Grupo etario afectado, Distribución por sexo. \\
\hline
\textbf{Filtros} & Por rango fechas, Por especialidad, Por grupo etario, Por capítulo CIE-10. \\
\hline
\textbf{Salidas} & Tabla estadística, gráfico de Pareto, mapa de calor por meses, exportable. \\
\hline
\end{tabular}
\\[2mm]\textit{Nota.} La siguiente tabla representa el RF-037: Diagnósticos Frecuentes}
\end{table}

\textbf{Tabla 49}\\[2mm]\textit{RF-038: Reporte de Productividad Médica}

\begin{table}[H]
\small
\begin{tabular}{|p{3cm}|p{11cm}|}
\hline
\tablaheader{Atributo} & \tablaheader{Descripción} \\
\hline
\textbf{Prioridad} & Media \\
\hline
\textbf{Descripción} & Evaluar productividad individual de médicos con métricas cuantitativas. \\
\hline
\textbf{Métricas} & Médico, Total consultas mes, Promedio consultas/día, Promedio duración consulta, Tasa ausentismo pacientes, Días trabajados, Horas efectivas atención, Comparativa con promedio general. \\
\hline
\textbf{Salidas} & Tabla ranking, gráfico radar por médico, identificación outliers, exportable. \\
\hline
\end{tabular}
\\[2mm]\textit{Nota.} La siguiente tabla representa el RF-038: Productividad Médica}
\end{table}
\newpage
\textbf{Tabla 50}\\[2mm]\textit{RF-039: Exportar Reportes a Excel y PDF}

\begin{table}[H]
\small
\begin{tabular}{|p{3cm}|p{11cm}|}
\hline
\tablaheader{Atributo} & \tablaheader{Descripción} \\
\hline
\textbf{Prioridad} & Media \\
\hline
\textbf{Descripción} & Permitir exportación de todos los reportes a formatos estándar para análisis externo. \\
\hline
\textbf{Formatos} & Excel (.xlsx) con formato profesional (colores, bordes, totales), PDF (A4) con logo y marca de agua, CSV para importación a otros sistemas. \\
\hline
\textbf{Procesamiento} & Generar archivo en servidor, comprimir si es mayor a 5MB, ofrecer descarga directa, enviar por email opcionalmente. \\
\hline
\end{tabular}
\\[2mm]\textit{Nota.} La siguiente tabla representa el RF-039: Exportar Reportes}
\end{table}

\subsubsection{Módulo M10: Respaldo y Recuperación}

\textbf{Tabla 51}\\[2mm]\textit{RF-040: Backup Automático Diario}

\begin{table}[H]
\small
\begin{tabular}{|p{3cm}|p{11cm}|}
\hline
\tablaheader{Atributo} & \tablaheader{Descripción} \\
\hline
\textbf{Prioridad} & Crítica \\
\hline
\textbf{Descripción} & Ejecutar respaldo automático completo de base de datos diariamente a las 02:00 AM. \\
\hline
\textbf{Procesamiento} & SQL Server Agent Job programado, backup tipo FULL a carpeta segura, compresión activada, verificación de integridad con CHECKSUM, notificación email si falla. \\
\hline
\textbf{Almacenamiento} & Ruta: /Backups/SIGHC\_FULL\_YYYY-MM-DD.bak, retención 90 días, copia a servidor remoto (offsite). \\
\hline
\textbf{Postcondición} & Archivo .bak verificado, registro en tabla de backups, alerta si ocupa más de 80\% espacio disponible. \\
\hline
\end{tabular}
\\[2mm]\textit{Nota.} La siguiente tabla representa el RF-040: Backup Diario}
\end{table}

\textbf{Tabla 52}\\[2mm]\textit{RF-041: Backup Diferencial Semanal}

\begin{table}[H]
\small
\begin{tabular}{|p{3cm}|p{11cm}|}
\hline
\tablaheader{Atributo} & \tablaheader{Descripción} \\
\hline
\textbf{Prioridad} & Alta \\
\hline
\textbf{Descripción} & Ejecutar respaldo diferencial todos los domingos complementario al diario. \\
\hline
\textbf{Procesamiento} & Backup tipo DIFFERENTIAL (solo cambios desde último FULL), menor tiempo ejecución, menor tamaño archivo, programado domingos 03:00 AM. \\
\hline
\textbf{Beneficio} & Optimiza espacio de almacenamiento, reduce ventana de backup, recuperación más rápida. \\
\hline
\end{tabular}
\\[2mm]\textit{Nota.} La siguiente tabla representa el RF-041: Backup Diferencial}
\end{table}

\textbf{Tabla 53}\\[2mm]\textit{RF-042: Restaurar Base de Datos desde Backup}

\begin{table}[H]
\small
\begin{tabular}{|p{3cm}|p{11cm}|}
\hline
\tablaheader{Atributo} & \tablaheader{Descripción} \\
\hline
\textbf{Prioridad} & Crítica \\
\hline
\textbf{Descripción} & Restaurar base de datos a punto en el tiempo específico desde archivos de backup. \\
\hline
\textbf{Procesamiento} & Seleccionar archivo .bak, validar integridad con RESTORE VERIFYONLY, modo NORECOVERY para backup diferencial, aplicar logs de transacciones, modo RECOVERY final, validar tablas críticas post-restauración. \\
\hline
\textbf{RTO/RPO} & Recovery Time Objective menor a 4 horas, Recovery Point Objective menor a 15 minutos (con logs). \\
\hline
\textbf{Postcondición} & Base de datos operativa, validación de integridad OK, usuarios reconectados, auditoría de restauración registrada. \\
\hline
\end{tabular}
\\[2mm]\textit{Nota.} La siguiente tabla representa el RF-042: Restaurar Base de Datos}
\end{table}



% =====================================================
% 4. REQUERIMIENTOS NO FUNCIONALES
% =====================================================
\newpage
\subsection{Requerimientos No Funcionales}

Los requerimientos no funcionales (RNF) definen criterios de calidad, restricciones técnicas y atributos del sistema que garantizan su correcto funcionamiento.

\textbf{Tabla 54}\\[2mm]\textit{Requerimientos no Funcionales}

\begin{longtable}{|p{1.5cm}|p{2.5cm}|p{5cm}|p{4.5cm}|p{1.8cm}|}
\hline
\tablaheader{ID} & \tablaheader{Categoría} & \tablaheader{Descripción} & \tablaheader{Criterio de Aceptación} & \tablaheader{Prioridad} \\
\hline
\endfirsthead

\hline
\tablaheader{ID} & \tablaheader{Categoría} & \tablaheader{Descripción} & \tablaheader{Criterio de Aceptación} & \tablaheader{Prioridad} \\
\hline
\endhead

RNF-01 & Rendimiento & El tiempo de respuesta del sistema para consultas simples debe ser menor a 2 segundos en el 95\% de los casos. & Medición con herramientas de monitoreo: percentil 95 menor a 2000ms. & Crítica \\
\hline
RNF-02 & Rendimiento & El sistema debe soportar 500 usuarios concurrentes sin degradación mayor al 10\% en rendimiento. & Pruebas de carga con JMeter: 500 usuarios concurrentes con throughput estable. & Alta \\
\hline
RNF-03 & Disponibilidad & El sistema debe tener un uptime de 99.9\% mensual, excluyendo ventanas de mantenimiento programado. & Monitoreo 24/7 con alertas proactivas, máximo 43 minutos de downtime al mes. & Crítica \\
\hline
RNF-04 & Seguridad & Todas las comunicaciones deben usar protocolo HTTPS con TLS 1.3 y certificado SSL válido. & Auditoría con SSL Labs debe obtener calificación A+ mínimo. & Crítica \\
\hline
RNF-05 & Seguridad & Las contraseñas deben almacenarse hasheadas con algoritmo bcrypt (cost factor 12) con salt único por usuario. & Verificación en código: ninguna contraseña en texto plano, auditoría de seguridad aprobada. & Crítica \\
\hline
RNF-06 & Seguridad & Datos sensibles (historias clínicas, diagnósticos) deben cifrarse en reposo usando TDE (Transparent Data Encryption). & Activación de TDE en SQL Server, verificación de cifrado en archivos de base de datos. & Alta \\
\hline
RNF-07 & Escalabilidad & El sistema debe soportar el crecimiento de 50,000 historias clínicas sin degradación de rendimiento. & Pruebas de estrés con base de datos poblada: consultas siguen cumpliendo RNF-01. & Media \\
\hline
RNF-08 & Usabilidad & La interfaz debe ser intuitiva requiriendo máximo 2 clics para operaciones frecuentes. & Test de usabilidad con 20 usuarios: 90\% completan tareas sin ayuda. & Alta \\
\hline
RNF-09 & Mantenibilidad & El 100\% de los objetos de base de datos deben estar documentados con comentarios explicativos. & Revisión de código: todos los SP, funciones, triggers y tablas con documentación. & Alta \\
\hline
RNF-10 & Portabilidad & El sistema debe funcionar en Windows Server 2019+ y SQL Server 2019+. & Certificación de compatibilidad en ambientes de prueba con versiones especificadas. & Media \\
\hline
RNF-11 & Recuperabilidad & RPO (Recovery Point Objective) debe ser menor o igual a 15 minutos. & Configuración de backups y logs: pérdida máxima de 15 min de datos. & Crítica \\
\hline
RNF-12 & Recuperabilidad & RTO (Recovery Time Objective) debe ser menor o igual a 4 horas. & Prueba de restauración completa: sistema operativo en menos de 4 horas. & Crítica \\
\hline
RNF-13 & Auditoría & El 100\% de cambios en tablas críticas debe registrarse automáticamente con triggers. & Verificación en base de datos: triggers activos en todas las tablas de auditoría. & Crítica \\
\hline
RNF-14 & Backup & Retención de backups completos durante 90 días mínimo. & Política de retención configurada, verificación mensual de existencia de backups. & Alta \\
\hline
RNF-15 & Monitoreo & Sistema de alertas proactivas 24/7 para eventos críticos (BD caída, espacio en disco, errores). & Sistema de monitoreo activo con notificaciones automáticas por email/SMS. & Alta \\
\hline
\end{longtable}

\\[2mm]\textit{Nota.} La siguiente tabla representa el RF-042: Restaurar Base de Datos}

% =====================================================
% 5. CASOS DE USO DETALLADOS
% =====================================================

\newpage
\subsection{Casos de Uso }

Los casos de uso (CU) describen interacciones específicas entre actores y el sistema para lograr objetivos definidos.

\textbf{Tabla 54}\\[2mm]\textit{Matriz de Trazabilidad de Casos de Uso y Requerimientos Funcionales}

\begin{table}[H]
\centering
\small
\begin{tabular}{|p{1.6cm}|p{2.2cm}|p{6.5cm}|p{3.5cm}|}
\hline
\tablaheader{Módulo} & \tablaheader{Código} & \tablaheader{Casos de Uso} & \tablaheader{Requerimiento Funcional} \\
\hline
M01 & CU-01 & Registrar Paciente & RF01 \\
\hline
M01 & CU-02 & Actualizar Datos del Paciente & RF02 \\
\hline
M01 & CU-03 & Buscar Paciente & RF04 \\
\hline
M01 & CU-04 & Consultar Historia Clínica & RF03 \\
\hline
M02 & CU-05 & Programar Cita Médica & RF05 \\
\hline
M02 & CU-06 & Reprogramar Cita Médica & RF06 \\
\hline
M02 & CU-07 & Consultar Agenda Médica & RF07 \\
\hline
M02 & CU-08 & Registrar Asistencia a Cita & RF08 \\
\hline
M03 & CU-09 & Consultas y Diagnósticos & RF09 \\
\hline
M03 & CU-10 & Registrar Diagnóstico CIE-10 & RF10 \\
\hline
M03 & CU-11 & Modificar Diagnóstico & RF11 \\
\hline
M03 & CU-12 & Consultar Historial de Consultas & RF12 \\
\hline
M04 & CU-13 & Prescribir Tratamiento & RF13 / RF16 \\
\hline
M04 & CU-14 & Consultar Tratamientos Activos & RF14 \\
\hline
M04 & CU-15 & Generar Receta Médica & RF15 \\
\hline
M05 & CU-16 & Registrar Médico & RF17 \\
\hline
M05 & CU-17 & Gestionar Horarios Médicos & RF18 / RF19 \\
\hline
M05 & CU-18 & Consultar Productividad Médica & RF20 \\
\hline
M06 & CU-19 & Registrar Medicamento & RF21 \\
\hline
M06 & CU-20 & Gestionar Movimientos de Inventario & RF22 / RF23 \\
\hline
M06 & CU-21 & Consultar Inventario & RF24 \\
\hline
M07 & CU-22 & Autenticar Usuario & RF25 \\
\hline
M07 & CU-23 & Gestionar Usuarios y Roles & RF26 / RF27 \\
\hline
M07 & CU-24 & Gestionar Seguridad de Acceso & RF28 \\
\hline
M08 & CU-25 & Registrar Auditoría & RF29 / RF30 \\
\hline
M08 & CU-26 & Consultar Bitácora de Auditoría & RF31 / RF32 \\
\hline
M09 & CU-27 & Generar Reportes & RF33 / RF34 / RF37 \\
\hline
M09 & CU-28 & Exportar Reportes & RF35 \\
\hline
M09 & CU-29 & Visualizar Dashboards & RF36 \\
\hline
M10 & CU-30 & Ejecutar Respaldo de Base de Datos & RF38 / RF40 \\
\hline
M10 & CU-31 & Auditar Accesos Clínicos & RF39 \\
\hline
M10 & CU-32 & Generar Dashboards & RF41 / RF42 \\
\hline
\end{tabular}
\end{table}
\newpage
\subsection{Diagramas de Casos de Uso}
\subsubsection{Casos de uso de negocio}

\begin{figure}[H]
    \centering
        \caption{Diagrama De Caso De Uso De Negocio (Sighc)}
\vspace{3mm}

    \includegraphics[width=0.5\linewidth]{DIAGRAMA DE CASO DE USO DE NEGOCIO.png}
\end{figure}

\textit{Nota.} El diagrama presenta la interacción entre los actores (Personal de Admisión, Personal Médico y Administrador) y las funcionalidades principales del sistema, tales como la gestión de historias clínicas bajo la norma NTS N° 139-MINSA y la codificación de diagnósticos CIE-10. Elaboración propia (2025).
\newpage
\subsubsection{Casos de uso por módulo}

M01 – Gestión de Pacientes e Historias Clínicas

\begin{figure}[H]
    \centering
        \caption{Diagrama de caso de uso del modulo 01}
\vspace{3mm}

    \includegraphics[width=0.5\linewidth]{M01 – Gestión de Pacientes.png}
\end{figure}

\textit{Nota.} El diagrama ilustra las interacciones entre los actores del sistema (Personal de Admisión, Médico y Administrador) y las funcionalidades clave como la gestión de citas y el registro de diagnósticos CIE-10. Se prioriza el flujo de datos conforme a la Norma Técnica de Salud N° 139-MINSA. Elaboración propia (2025).

M02 – Gestión de Citas Médicas

\begin{figure}[H]
    \centering
        \caption{Diagrama de caso de uso del modulo 02}
\vspace{3mm}

    \includegraphics[width=0.5\linewidth]{M02 – Gestión de Citas Médicas.png}
\end{figure}

\textit{Nota.} El diagrama ilustra las interacciones entre los actores principales (Personal de Admisión, Personal Médico y Administrador) y los procesos esenciales del sistema, como la gestión de agendas, el registro de diagnósticos bajo el estándar CIE-10 y el control de seguridad de datos. El diseño se alinea con los requerimientos de la Norma Técnica de Salud N° 139-MINSA. Elaboración propia (2025).

M03 – Consultas y Diagnósticos

\begin{figure}[H]
    \centering
        \caption{Diagrama de caso de uso del modulo 03}
\vspace{3mm}

    \includegraphics[width=0.5\linewidth]{M03 – Consultas y Diagnósticos.png}
\end{figure}

\textit{Nota.} El diagrama ilustra las interacciones de los actores (Personal de Admisión, Médico, Farmacéutico y Administrador) con los módulos funcionales del sistema. Se destaca la integración de la norma NTS N° 139-MINSA para el manejo de historias clínicas y el uso de estándares CIE-10 para el registro de diagnósticos médicos. Elaboración propia (2025).

M04 – Tratamientos y Prescripciones

\begin{figure}[H]
    \centering
        \caption{Diagrama de caso de uso del modulo 04}
\vspace{3mm}

    \includegraphics[width=0.5\linewidth]{M04 – Tratamientos y Prescripciones.png}
\end{figure}

\textit{Nota.} El diagrama representa las interacciones entre los actores (Personal de Admisión, Personal Médico, Personal de Farmacia y Administrador) y las funciones principales del sistema. Se destacan los procesos de registro de pacientes, atención clínica con codificación CIE-10 y la trazabilidad de datos sensibles conforme a la Ley N° 29733 de Protección de Datos Personales. Elaboración propia (2025).

M05 – Gestión de Personal Médico

\begin{figure}[H]
    \centering
        \caption{Diagrama de caso de uso del modulo 05}
\vspace{3mm}

    \includegraphics[width=0.5\linewidth]{M05 – Gestión de Personal Médico.png}
\end{figure}

\textit{Nota.}  El diagrama detalla las interacciones entre los actores del sistema (Personal de Admisión, Médico, Farmacia y Administrador) y las funcionalidades del SIGHC. Se enfatiza la automatización de procesos clínicos y la seguridad de la información sensible, bajo el cumplimiento de la Norma Técnica de Salud N° 139-MINSA y la Ley N° 29733. Elaboración propia (2025).

M06 – Inventario de Medicamentos

\begin{figure}[H]
    \centering
        \caption{Diagrama de caso de uso del modulo 06}
\vspace{3mm}

    \includegraphics[width=0.5\linewidth]{M06 – Inventario de Medicamentos.png}
\end{figure}

\textit{Nota.}  El diagrama ilustra las interacciones entre los actores principales (Personal de Admisión, Personal Médico y Administrador) con los módulos funcionales del sistema. Se destaca la integración de la norma NTS N° 139-MINSA para la gestión de historias clínicas y el cumplimiento de la Ley N° 29733 de Protección de Datos Personales en el manejo de información sensible. Elaboración propia (2025).

M07 – Seguridad y Control de Acceso

\begin{figure}[H]
    \centering
        \caption{Diagrama de caso de uso del modulo 07}
\vspace{3mm}

    \includegraphics[width=0.5\linewidth]{M07 – Seguridad y Control de Acceso.png}
\end{figure}

\textit{Nota.}  El diagrama presenta la interacción entre los actores (Admisión, Médico, Farmacia y Administrador) y las funcionalidades centrales del sistema SIGHC. Se destacan procesos críticos como el registro de historias clínicas electrónicas, la programación de citas y la gestión de diagnósticos estandarizados con CIE-10, bajo el marco de la Norma Técnica de Salud N° 139-MINSA. Elaboración propia (2025).

M08 – Auditoría y Trazabilidad

\begin{figure}[H]
    \centering
        \caption{Diagrama de caso de uso del modulo 08}
\vspace{3mm}

    \includegraphics[width=0.5\linewidth]{M08 – Auditoría y Trazabilidad.png}
\end{figure}

\textit{Nota.}  El diagrama representa las interacciones entre los actores del sistema (Personal de Admisión, Médico, Farmacéutico y Administrador) y las funcionalidades principales del SIGHC. Se enfatiza la automatización del flujo clínico, el registro de diagnósticos mediante el estándar internacional CIE-10 y el cumplimiento de la Norma Técnica de Salud N° 139-MINSA para la gestión de documentos médicos electrónicos. Elaboración propia (2025).

M09 – Reportes y Business Intelligence

\begin{figure}[H]
    \centering
        \caption{Diagrama de caso de uso del modulo 09}
\vspace{3mm}

    \includegraphics[width=0.5\linewidth]{M09 – Reportes y Business Intelligence.png}
\end{figure}

\textit{Nota.}   El diagrama describe las funciones principales del sistema organizadas por módulos, desde la admisión hasta la gestión de farmacia y laboratorio. Se representa la interacción de los diversos actores con los requerimientos funcionales, asegurando la trazabilidad de la atención médica según la Norma Técnica de Salud N° 139-MINSA y los estándares de seguridad para el manejo de bases de datos clínicos. Elaboración propia (2025).

M10 – Respaldo y Recuperación

\begin{figure}[H]
    \centering
        \caption{Diagrama de caso de uso del modulo 10}
\vspace{3mm}

    \includegraphics[width=0.5\linewidth]{M10 – Respaldo y Recuperación.png}
\end{figure}

\textit{Nota.}   El diagrama detalla las interacciones de los actores con los ocho módulos funcionales del sistema, integrando procesos de admisión, atención médica y gestión de farmacia. Se destaca la implementación de estándares de seguridad y confidencialidad de acuerdo con la Norma Técnica de Salud N° 139-MINSA y la Ley N° 29733 de Protección de Datos Personales. Elaboración propia (2025).

\subsection{Especificación de Casos de Uso}
Se mostra las especificaciones detalladas de caso de uso dependiendo los modulos.

\setcounter{table}{0}
\begin{table}[H]
\centering
\caption{Especificación del Caso de Uso R1-CU-01}
\centering
\small
\begin{tabular}{|p{4cm}|p{10cm}|}
\hline
\multicolumn{2}{|l|}{\textbf{Caso de Uso: CU-01 -- Registrar Paciente}} \\
\hline
\textbf{Código} & R1-CU-01 \\
\hline
\textbf{Nombre} & Registrar Paciente \\
\hline
\textbf{Versión} & 1.0 \\
\hline
\textbf{Descripción} &
Permite al personal de admisión registrar de manera completa y segura a un nuevo paciente en el sistema SIGHC. El proceso incluye la validación de identidad del paciente, el registro de datos personales y de contacto, y la generación automática de un número único de historia clínica, el cual será utilizado en todas las atenciones médicas posteriores. \\
\hline
\textbf{Actores} &
Personal de Admisión (actor principal) \newline
Sistema (actor automatizado)
 \\
\hline
\textbf{Tipo} & Primario \\
\hline
\textbf{Referencias} & RF-001: Registrar Nuevo Paciente \\
\hline
\textbf{Flujo Básico} &
1. El personal de admisión accede al módulo de Gestión de Pacientes.\newline
2. Selecciona la opción "Registrar Paciente".\newline
3. El sistema muestra el formulario de registro.\newline
4. El personal ingresa los datos personales del paciente (DNI, nombres, apellidos, fecha de nacimiento, sexo, dirección, teléfono y correo electrónico).\newline
5. El sistema valida la obligatoriedad y el formato correcto de los datos ingresados.\newline
6. El sistema verifica que el DNI no se encuentre previamente registrado.\newline
7. El sistema genera automáticamente un número único de historia clínica.\newline
8. El sistema almacena la información del paciente en la base de datos.\newline
9. El sistema confirma el registro exitoso del paciente.
 \\
\hline
\textbf{Flujo Alternativo} &
A1: DNI duplicado\newline
En el paso 6, si el DNI ya existe, el sistema muestra un mensaje de error e impide continuar con el registro. \\
\hline
\textbf{Requerimientos Especiales} &
Validación de unicidad del DNI.\newline
Registro automático de fecha y usuario que realizó el registro. \\
\hline
\textbf{Pre-Condiciones} &
El personal de admisión debe estar autenticado en el sistema. \\
\hline
\textbf{Post-Condiciones} &
El paciente queda registrado con una historia clínica única.\\
\hline
\textbf{Puntos de Extensión} &
R1-CU-02 – Actualizar Datos del Paciente \\
\hline
\end{tabular}
\end{table}

\begin{table}[H]
\centering
\caption{Especificación del Caso de Uso R1-CU-02}
\vspace{3mm}
\small
\begin{tabular}{|p{4cm}|p{10cm}|}
\hline
\multicolumn{2}{|l|}{\textbf{Caso de Uso: CU-02 -- Actualizar Datos del Paciente}} \\
\hline
\textbf{Código} & R1-CU-02 \\
\hline
\textbf{Nombre} & Actualizar Datos del Paciente \\
\hline
\textbf{Versión} & 1.0 \\
\hline
\textbf{Descripción} &
Permite al personal de admisión modificar o actualizar la información personal y de contacto de un paciente previamente registrado, garantizando que los datos se mantengan actualizados y confiables. \\
\hline
\textbf{Actores} &
Personal de Admisión (actor principal) \newline
Sistema \\
\hline
\textbf{Tipo} & Primario \\
\hline
\textbf{Referencias} & RF-002: Modificar Datos de Paciente \\
\hline
\textbf{Flujo Básico} &
1. El personal de admisión accede al módulo de Gestión de Pacientes.\newline
2. Busca al paciente mediante DNI o número de historia clínica.\newline
3. El sistema muestra los datos actuales del paciente.\newline
4. El personal selecciona la opción "Actualizar Datos".\newline
5. Modifica los campos necesarios.\newline
6. El sistema valida los nuevos datos ingresados.\newline
7. El sistema guarda los cambios realizados.\newline
8. El sistema confirma la actualización exitosa. \\
\hline
\textbf{Flujo Alternativo} &
A1: Datos inválidos\newline
En el paso 6, si los datos no cumplen con el formato requerido, el sistema muestra un mensaje de error.\\
\hline
\textbf{Requerimientos Especiales} &
El sistema debe registrar en la bitácora de auditoría todas las modificaciones realizadas, indicando usuario, fecha, hora y campos alterados.\newline
No se debe permitir la modificación del número de historia clínica ni del DNI, salvo por usuarios con privilegios administrativos.\newline
El sistema debe validar la consistencia y formato de los datos actualizados.\newline
Garantizar la confidencialidad de la información personal del paciente conforme a las normativas de protección de datos.\\
\hline
\textbf{Pre-Condiciones} &
El usuario debe estar autenticado.\\
\hline
\textbf{Post-Condiciones} &
Paciente localizado o mensaje informativo mostrado.\\
\hline
\textbf{Puntos de Extensión} &
R1-CU-03 – Buscar Paciente (para localizar previamente al paciente).\newline
R1-CU-25 – Registrar Auditoría (registro automático de cambios).\\
\hline
\end{tabular}
\end{table}

\begin{table}[H]
\centering
\caption{Especificación del Caso de Uso R1-CU-03}
\vspace{3mm}
\small
\begin{tabular}{|p{4cm}|p{10cm}|}
\hline
\multicolumn{2}{|l|}{\textbf{Caso de Uso: CU-03 -- Buscar Paciente}} \\
\hline
\textbf{Código} & R1-CU-0 \\
\hline
\textbf{Nombre} & Buscar Paciente \\
\hline
\textbf{Versión} & 1.0 \\
\hline
\textbf{Descripción} &
Permite a los usuarios autorizados localizar a un paciente registrado en el sistema utilizando distintos criterios de búsqueda, facilitando el acceso rápido a su información. \\
\hline
\textbf{Actores} &
Personal de Admisión (actor principal) \newline
Medico \\
\hline
\textbf{Tipo} & Primario \\
\hline
\textbf{Referencias} & RF-004: Consultar Historia Clínica Completa\\
\hline
\textbf{Flujo Básico} &
1. El usuario accede a la opción "Buscar Paciente"..\newline
2. Ingresa uno o más criterios de búsqueda (DNI, nombres, apellidos o historia clínica).\newline
3. El sistema procesa la búsqueda.\newline
4. El sistema muestra la lista de pacientes coincidentes.\\
\hline
\textbf{Flujo Alternativo} &
A1: Sin resultados\newline
Si no se encuentran coincidencias, el sistema muestra un mensaje informativo.\\
\hline
\textbf{Requerimientos Especiales} &
El sistema debe permitir la búsqueda mediante múltiples criterios combinados (DNI, nombres, apellidos, historia clínica).\newline
El tiempo de respuesta de la búsqueda no debe exceder los 2 segundos bajo condiciones normales.\newline
Solo usuarios autenticados y autorizados pueden realizar búsquedas.\newline
Los resultados deben mostrarse respetando los niveles de acceso según el rol del usuario.\\
\hline
\textbf{Pre-Condiciones} &
El usuario debe estar autenticado.\\
\hline
\textbf{Post-Condiciones} &
Paciente localizado o mensaje informativo mostrado.\\
\hline
\textbf{Puntos de Extensión} &
R1-CU-02 – Actualizar Datos del Paciente.\newline
R1-CU-04 – Consultar Historia Clínica.\newline
R1-CU-05 – Programar Cita Médica \\
\hline
\end{tabular}
\end{table}
%5
\begin{table}[H]
\centering
\caption{Especificación del Caso de Uso R1-CU-04}
\vspace{3mm}
\small
\begin{tabular}{|p{4cm}|p{10cm}|}
\hline
\multicolumn{2}{|l|}{\textbf{Caso de Uso: CU-04 -- Consultar Historia Clínica}} \\
\hline
\textbf{Código} & R1-CU-04 \\
\hline
\textbf{Nombre} & Consultar Historia Clínica \\
\hline
\textbf{Versión} & 1.0 \\
\hline
\textbf{Descripción} &
Permite al médico consultar de forma integral la historia clínica de un paciente, incluyendo antecedentes, consultas, diagnósticos, tratamientos y prescripciones registradas. \\
\hline
\textbf{Actores} &
Medico (actor principal) \newline
Sistema (actor automatizado)
 \\
\hline
\textbf{Tipo} & Primario \\
\hline
\textbf{Referencias} & RF-003: Buscar Paciente con Filtros Múltiples\\
\hline
\textbf{Flujo Básico} &
1. El médico busca al paciente en el sistema.\newline
2. Selecciona la opción "Consultar Historia Clínica".\newline
3. El sistema valida los permisos del médico.\newline
4. El sistema muestra la historia clínica completa del paciente. \\
\hline
\textbf{Flujo Alternativo} &
A1: Acceso no autorizado\newline
Si el médico no tiene permisos, el sistema impide el acceso. \\
\hline
\textbf{Requerimientos Especiales} &
El acceso a la historia clínica debe estar restringido exclusivamente a personal médico autorizado.\newline
El sistema debe registrar cada acceso a la historia clínica en la auditoría clínica.\newline
La información debe mostrarse de forma cronológica y estructurada.\newline
Garantizar la integridad y confidencialidad de los datos clínicos visualizados.\\
\hline
\textbf{Pre-Condiciones} &
El paciente debe estar registrado.\\
\hline
\textbf{Post-Condiciones} &
El paciente queda registrado con una historia clínica única.\\
\hline
\textbf{Puntos de Extensión} &
R1-CU-09 – Registrar Consulta Médica.\newline
R1-CU-12 – Consultar Historial de Consultas.\newline
R1-CU-31 – Auditar Accesos Clínicos.\\
\hline
\end{tabular}
\end{table}

% 5
\begin{table}[H]
\centering
\caption{Especificación del Caso de Uso R1-CU-05}
\vspace{3mm}
\small
\begin{tabular}{|p{4cm}|p{10cm}|}
\hline
\multicolumn{2}{|l|}{\textbf{Caso de Uso: CU-05 -- Programar Cita Médica}} \\
\hline
\textbf{Código} & R1-CU-05 \\
\hline
\textbf{Nombre} & Programar Cita Médica \\
\hline
\textbf{Versión} & 1.0 \\
\hline
\textbf{Descripción} &
Permite al personal de admisión programar una cita médica para un paciente, validando la disponibilidad del médico, la especialidad y el horario solicitado. \\
\hline
\textbf{Actores} &
Personal de Admisión (actor principal) \newline
Sistema \\
\hline
\textbf{Tipo} & Primario \\
\hline
\textbf{Referencias} & RF-005: Programar Nueva Cita Médica \\
\hline
\textbf{Flujo Básico} &
1. El personal de admisión selecciona al paciente.\newline
2. Elige la especialidad médica requerida..\newline
3. El sistema muestra los médicos disponibles.\newline
4. Se selecciona médico, fecha y hora.\newline
5. El sistema valida la disponibilidad.\newline
6. El sistema registra la cita médica.\newline
7. El El sistema confirma la programación. \\
\hline
\textbf{Flujo Alternativo} &
A1: Horario no disponible\newline
En el paso 5, si el horario está ocupado, el sistema solicita seleccionar otro horario.\\
\hline
\textbf{Requerimientos Especiales} &
El sistema debe validar automáticamente la disponibilidad del médico, especialidad y horario.\newline
No se deben permitir solapamientos de citas para un mismo médico.\newline
El sistema debe permitir la configuración de duración estándar de citas por especialidad.\newline
Se debe registrar la acción en la bitácora del sistema.\\
\hline
\textbf{Pre-Condiciones} &
El usuario debe estar autenticado.\\
\hline
\textbf{Post-Condiciones} &
Paciente localizado o mensaje informativo mostrado.\\
\hline
\textbf{Puntos de Extensión} &
R1-CU-06 – Reprogramar Cita Médica.\newline
R1-CU-07 – Consultar Agenda Médica.\newline
R1-CU-08 – Registrar Asistencia a Cita.\newline
R1-CU-25 – Registrar Auditoría.\\
\hline
\end{tabular}
\end{table}
% 6
\begin{table}[H]
\centering
\caption{Especificación del Caso de Uso R1-CU-06}
\vspace{3mm}
\small
\begin{tabular}{|p{4cm}|p{10cm}|}
\hline
\multicolumn{2}{|l|}{\textbf{Caso de Uso: CU-06 -- Reprogramar Cita Médica}} \\
\hline
\textbf{Código} & R1-CU-06 \\
\hline
\textbf{Nombre} & Reprogramar Cita Médica \\
\hline
\textbf{Versión} & 1.0 \\
\hline
\textbf{Descripción} &
Permite al personal de admisión modificar la fecha, hora o médico asignado de una cita médica previamente programada, asegurando la correcta gestión de la agenda y notificando los cambios correspondientes. \\
\hline
\textbf{Actores} &
Personal de Admisión (actor principal) \newline
Sistema \\
\hline
\textbf{Tipo} & Primario \\
\hline
\textbf{Referencias} & RF-006: Cancelar o Reprogramar Cita\\
\hline
\textbf{Flujo Básico} &
1. El personal de admisión accede al módulo de citas médicas.\newline
2. Busca la cita previamente programada.\newline
3. Selecciona la opción "Reprogramar Cita".\newline
4. El sistema muestra los datos actuales de la cita.\newline
5. El personal ingresa la nueva fecha, hora o médico.\newline
6. El sistema valida la disponibilidad del nuevo horario.\newline
7. El sistema actualiza la información de la cita.\newline
8. El sistema confirma la reprogramación exitosa.\\
\hline
\textbf{Flujo Alternativo} &
A1: Horario no disponible\newline
En el paso 6, si el horario seleccionado no está disponible, el sistema solicita elegir otro horario.\\
\hline
\textbf{Requerimientos Especiales} &
Validación automática de conflictos de agenda.\newline
Registro de auditoría de la reprogramación.\\
\hline
\textbf{Pre-Condiciones} &
La cita debe existir y encontrarse activa.\\
\hline
\textbf{Post-Condiciones} &
Cita médica reprogramada correctamente.\\
\hline
\textbf{Puntos de Extensión} &
R1-CU-07 – Consultar Agenda Médica\\
\hline
\end{tabular}
\end{table}
% 7
\begin{table}[H]
\centering
\caption{Especificación del Caso de Uso R1-CU-07}
\vspace{3mm}
\small
\begin{tabular}{|p{4cm}|p{10cm}|}
\hline
\multicolumn{2}{|l|}{\textbf{Caso de Uso: CU-07 -- Consultar Agenda Médica}} \\
\hline
\textbf{Código} & R1-CU-07 \\
\hline
\textbf{Nombre} & Consultar Agenda Médica \\
\hline
\textbf{Versión} & 1.0 \\
\hline
\textbf{Descripción} &
Permite al médico y al personal de admisión visualizar la agenda médica organizada por fecha, especialidad o profesional, facilitando la planificación y control de las atenciones médicas.\\
\hline
\textbf{Actores} &
Medico\newline
Personal de Admisión (actor principal)\\
\hline
\textbf{Tipo} & Primario \\
\hline
\textbf{Referencias} & RF-007: Confirmar Asistencia a Cita \\
\hline
\textbf{Flujo Básico} &
1.	El usuario accede al módulo de citas médicas. \newline
2.	Selecciona la opción "Consultar Agenda". \newline
3.	Define los filtros de búsqueda (fecha, médico o especialidad). \newline
4.	El sistema procesa la solicitud. \newline
5.	El sistema muestra la agenda médica correspondiente\\
\hline
\textbf{Flujo Alternativo} &
A1: Sin citas programadas\newline
Si no existen citas para el criterio seleccionado, el sistema muestra un mensaje informativo.\\
\hline
\textbf{Requerimientos Especiales} &
Actualización de agenda en tiempo real. \newline
Visualización diferenciada por roles.\\
\hline
\textbf{Pre-Condiciones} &
Usuario autenticado en el sistema.\\
\hline
\textbf{Post-Condiciones} &
Agenda médica consultada correctamente..\\
\hline
\textbf{Puntos de Extensión} &
R1-CU-05 – Programar Cita Médica\\
\hline
\end{tabular}
\end{table}
%8
\begin{table}[H]
\centering
\caption{Especificación del Caso de Uso R1-CU-08}
\vspace{3mm}
\small
\begin{tabular}{|p{4cm}|p{10cm}|}
\hline
\multicolumn{2}{|l|}{\textbf{Caso de Uso: CU-08 -- Registrar Asistencia a Cita}} \\
\hline
\textbf{Código} & R1-CU-08 \\
\hline
\textbf{Nombre} & Registrar Asistencia a Cita \\
\hline
\textbf{Versión} & 1.0 \\
\hline
\textbf{Descripción} &
Permite al personal de admisión registrar la asistencia o inasistencia del paciente a su cita médica, información utilizada para control operativo y generación de reportes.\\
\hline
\textbf{Actores} &
Personal de Admisión (actor principal) \newline
Sistema \\
\hline
\textbf{Tipo} & Primario \\
\hline
\textbf{Referencias} & RF08 – Registro de asistencia a citas \\
\hline
\textbf{Flujo Básico} &
1.	El personal de admisión accede al listado de citas del día.\newline
2.	Selecciona la cita correspondiente.\newline
3.	Selecciona la opción "Registrar Asistencia".\newline
4.	Indica si el paciente asistió o no.\newline
5.	El sistema guarda la información registrada.\\
\hline
\textbf{Flujo Alternativo} &
A1: Cita cancelada\newline
Si la cita fue cancelada previamente, el sistema no permite registrar asistencia.\\
\hline
\textbf{Requerimientos Especiales} &
Registro automático de fecha y usuario.\newline
Información disponible para reportes estadísticos.\\
\hline
\textbf{Pre-Condiciones} &
Cita médica programada.\\
\hline
\textbf{Post-Condiciones} &
Asistencia registrada correctamente.\\
\hline
\textbf{Puntos de Extensión} &
R1-CU-27 – Generar Reportes\\
\hline
\end{tabular}
\end{table}

%9
\begin{table}[H]
\centering
\caption{Especificación del Caso de Uso R1-CU-09}
\vspace{3mm}
\small
\begin{tabular}{|p{4cm}|p{10cm}|}
\hline
\multicolumn{2}{|l|}{\textbf{Caso de Uso: CU-09 -- Registrar Consulta Médica}} \\
\hline
\textbf{Código} & R1-CU-09 \\
\hline
\textbf{Nombre} & Registrar Consulta Médica \\
\hline
\textbf{Versión} & 1.0 \\
\hline
\textbf{Descripción} &
Permite al médico registrar de forma detallada la atención brindada al paciente durante una consulta médica, incluyendo signos vitales, motivo de consulta, evaluación clínica y observaciones. \\
\hline
\textbf{Actores} &
Medica (actor principal) \newline
Sistema \\
\hline
\textbf{Tipo} & Primario \\
\hline
\textbf{Referencias} & RF-009: Registrar Consulta Médica \\
\hline
\textbf{Flujo Básico} &
1.	El médico accede a la cita médica del paciente. \newline
2.	Selecciona la opción "Registrar Consulta". \newline
3.	Ingresa los signos vitales del paciente. \newline
4.	Registra el motivo de consulta y evaluación clínica. \newline
5.	El sistema valida la información ingresada. \newline
6.	El sistema guarda la consulta médica.\\
\hline
\textbf{Flujo Alternativo} &
A1: Información incompleta\newline
Si faltan datos obligatorios, el sistema solicita completarlos.\\
\hline
\textbf{Requerimientos Especiales} &
Campos clínicos obligatorios configurables.\newline
Registro de fecha, hora y médico responsable.\\
\hline
\textbf{Pre-Condiciones} &
Paciente con cita médica registrada.\\
\hline
\textbf{Post-Condiciones} &
Consulta médica registrada correctamente.\\
\hline
\textbf{Puntos de Extensión} &
R1-CU-10 – Registrar Diagnóstico CIE-10\\
\hline
\end{tabular}
\end{table}

%10
\begin{table}[H]
\centering
\caption{Especificación del Caso de Uso R1-CU-10}
\vspace{3mm}
\small
\begin{tabular}{|p{4cm}|p{10cm}|}
\hline
\multicolumn{2}{|l|}{\textbf{Caso de Uso: CU-10 -- Registrar Diagnóstico CIE-10}} \\
\hline
\textbf{Código} & R1-CU-10 \\
\hline
\textbf{Nombre} & Registrar Diagnóstico CIE-10 \\
\hline
\textbf{Versión} & 1.0 \\
\hline
\textbf{Descripción} &
Permite al médico registrar uno o más diagnósticos asociados a una consulta médica utilizando la clasificación internacional CIE-10, garantizando la estandarización clínica. \\
\hline
\textbf{Actores} &
Medico (actor principal) \newline
Sistema \\
\hline
\textbf{Tipo} & Primario \\
\hline
\textbf{Referencias} & RF-010: Registrar Diagnóstico CIE-10 \\
\hline
\textbf{Flujo Básico} &
1.	El médico accede a la consulta médica registrada. \newline
2.	Selecciona la opción "Registrar Diagnóstico". \newline
3.	Busca el diagnóstico en el catálogo CIE-10. \newline
4.	Selecciona el diagnóstico correspondiente. \newline
5.	El sistema registra el diagnóstico asociado a la consulta. \\
\hline
\textbf{Flujo Alternativo} &
A1: Diagnóstico no encontrado\newline
El sistema permite refinar la búsqueda por código o descripción.\\
\hline
\textbf{Requerimientos Especiales} &
Catálogo CIE-10 actualizado.\newline
Permitir registrar múltiples diagnósticos por consulta.\\
\hline
\textbf{Pre-Condiciones} &
Consulta médica registrada.\\
\hline
\textbf{Post-Condiciones} &
Diagnóstico registrado correctamente.\\
\hline
\textbf{Puntos de Extensión} &
R1-CU-11 – Modificar Diagnóstico\\
\hline
\end{tabular}
\end{table}

%11
\begin{table}[H]
\centering
\caption{Especificación del Caso de Uso R1-CU-11}
\vspace{3mm}
\small
\begin{tabular}{|p{4cm}|p{10cm}|}
\hline
\multicolumn{2}{|l|}{\textbf{Caso de Uso: CU-11 -- Modificar Diagnóstico}} \\
\hline
\textbf{Código} & R1-CU-11 \\
\hline
\textbf{Nombre} & Modificar Diagnóstico \\
\hline
\textbf{Versión} & 1.0 \\
\hline
\textbf{Descripción} &
Permite al médico modificar un diagnóstico previamente registrado cuando se requiera una corrección o actualización clínica, dejando constancia del cambio para fines médicos y legales.\\
\hline
\textbf{Actores} &
Medico (actor principal) \newline
Sistema \\
\hline
\textbf{Tipo} & Primario \\
\hline
\textbf{Referencias} & RF-011: Modificar Diagnóstico con Justificación \\
\hline
\textbf{Flujo Básico} &
1.	El médico accede al historial de diagnósticos del paciente.\newline
2.	Selecciona el diagnóstico a modificar.\newline
3.	Selecciona la opción "Modificar Diagnóstico".\newline
4.	Ingresa el nuevo diagnóstico o corrección.\newline
5.	Registra la justificación del cambio.\newline
6.	El sistema guarda la modificación y registra la auditoría.\\
\hline
\textbf{Flujo Alternativo} &
A1: Justificación no ingresada\newline
El sistema impide continuar hasta que se ingrese la justificación obligatoria.\\
\hline
\textbf{Requerimientos Especiales} &
Registro de auditoría obligatorio.\newline
Conservación del historial de cambios.\\
\hline
\textbf{Pre-Condiciones} &
Diagnóstico previamente registrado.\\
\hline
\textbf{Post-Condiciones} &
Diagnóstico modificado y auditado correctamente.\\
\hline
\textbf{Puntos de Extensión} &
R1-CU-25 – Registrar Auditoría\\
\hline
\end{tabular}
\end{table}

%12
\begin{table}[H]
\centering
\caption{Especificación del Caso de Uso R1-CU-12}
\vspace{3mm}
\small
\begin{tabular}{|p{4cm}|p{10cm}|}
\hline
\multicolumn{2}{|l|}{\textbf{Caso de Uso: CU-12 -- Consultar Historial de Consultas}} \\
\hline
\textbf{Código} & R1-CU-12 \\
\hline
\textbf{Nombre} & Consultar Historial de Consultas \\
\hline
\textbf{Versión} & 1.0 \\
\hline
\textbf{Descripción} &
Permite al médico consultar el historial completo de consultas médicas realizadas a un paciente, incluyendo fechas, diagnósticos, tratamientos y observaciones clínicas, con fines de seguimiento y toma de decisiones\\
\hline
\textbf{Actores} &
Medico (actor principal)\\
\hline
\textbf{Tipo} & Primario \\
\hline
\textbf{Referencias} & RF-012: Buscar Diagn´osticos por CIE-10 \\
\hline
\textbf{Flujo Básico} &
1.	El médico busca al paciente en el sistema.\newline
2.	Selecciona la opción "Consultar Historial de Consultas".\newline
3.	El sistema valida los permisos del médico.\newline
4.	El sistema muestra la lista cronológica de consultas.\newline
5.	El médico selecciona una consulta para ver el detalle\\
\hline
\textbf{Flujo Alternativo} &
A1: Historial vacío\newline
Si el paciente no registra consultas previas, el sistema muestra un mensaje informativo.\\
\hline
\textbf{Requerimientos Especiales} &
Visualización ordenada cronológicamente.\newline
Acceso restringido según rol médico.\\
\hline
\textbf{Pre-Condiciones} &
Paciente registrado en el sistema.\\
\hline
\textbf{Post-Condiciones} &
Historial de consultas visualizado.\\
\hline
\textbf{Puntos de Extensión} &
R1-CU-09 – Registrar Consulta Médica\\
\hline
\end{tabular}
\end{table}

%13
\begin{table}
\centering
\caption{Especificación del Caso de Uso R1-CU-13}
\vspace{3mm}
\small
\begin{tabular}{|p{4cm}|p{10cm}|}
\hline
\multicolumn{2}{|l|}{\textbf{Caso de Uso: CU-13 -- Prescribir Tratamiento}} \\
\hline
\textbf{Código} & R1-CU-13 \\
\hline
\textbf{Nombre} & Prescribir Tratamiento \\
\hline
\textbf{Versión} & 1.0 \\
\hline
\textbf{Descripción} &
Permite al médico prescribir un tratamiento médico al paciente, ya sea farmacológico o no farmacológico, asociado a un diagnóstico previamente registrado.\\
\hline
\textbf{Actores} &
Medico (actor principal) \newline
Sistema \\
\hline
\textbf{Tipo} & Primario \\
\hline
\textbf{Referencias} & 
RF-013: Prescribir Tratamiento Médico\newline
RF-016: Registrar Alerta de Interacción Medicamentosa\\
\hline
\textbf{Flujo Básico} &
1.	El médico accede a la consulta médica del paciente.\newline
2.	Selecciona la opción "Prescribir Tratamiento".\newline
3.	Registra el tipo de tratamiento, indicaciones y duración.\newline
4.	El sistema valida la información ingresada.\newline
5.	El sistema guarda el tratamiento prescrito.\\
\hline
\textbf{Flujo Alternativo} &
A1: Información incompleta\newline
El sistema solicita completar los campos obligatorios.\\
\hline
\textbf{Requerimientos Especiales} &
Validación de alergias registradas del paciente.\newline
Asociación obligatoria a un diagnóstico.\\
\hline
\textbf{Pre-Condiciones} &
Diagnóstico registrado para la consulta.\\
\hline
\textbf{Post-Condiciones} &
Tratamiento registrado correctamente.\\
\hline
\textbf{Puntos de Extensión} &
R1-CU-15 – Generar Receta Médica.\\
\hline
\end{tabular}
\end{table}

%14
\begin{table}
\centering
\caption{Especificación del Caso de Uso R1-CU-14}
\vspace{3mm}
\small
\begin{tabular}{|p{4cm}|p{10cm}|}
\hline
\multicolumn{2}{|l|}{\textbf{Caso de Uso: CU-14 -- Consultar Tratamientos Activos}} \\
\hline
\textbf{Código} & R1-CU-14 \\
\hline
\textbf{Nombre} & Consultar Tratamientos Activos \\
\hline
\textbf{Versión} & 1.0 \\
\hline
\textbf{Descripción} &
Permite al médico consultar los tratamientos activos de un paciente, facilitando el control y seguimiento de su evolución clínica. \\
\hline
\textbf{Actores} &
Medico (actor principal) \newline
Sistema \\
\hline
\textbf{Tipo} & Primario \\
\hline
\textbf{Referencias} & RF-014: Consultar Tratamientos Activos del Paciente \\
\hline
\textbf{Flujo Básico} &
1.	El médico busca al paciente.\newline
2.	Selecciona la opción "Consultar Tratamientos Activos".\newline
3.	El sistema muestra los tratamientos vigentes.\\
\hline
\textbf{Flujo Alternativo} &
A1: Sin tratamientos activos\newline
El sistema muestra un mensaje informativo.\\
\hline
\textbf{Requerimientos Especiales} &
Identificación clara de tratamientos vigentes y vencidos.\\
\hline
\textbf{Pre-Condiciones} &
Paciente con historial clínico.\\
\hline
\textbf{Post-Condiciones} &
Tratamientos activos visualizados.\\
\hline
\textbf{Puntos de Extensión} &
R1-CU-13 – Prescribir Tratamiento\\
\hline
\end{tabular}
\end{table}

%15
\begin{table}
\centering
\caption{Especificación del Caso de Uso R1-CU-15}
\vspace{3mm}
\small
\begin{tabular}{|p{4cm}|p{10cm}|}
\hline
\multicolumn{2}{|l|}{\textbf{Caso de Uso: CU-15 -- Generar Receta Médica}} \\
\hline
\textbf{Código} & R1-CU-15 \\
\hline
\textbf{Nombre} & Generar Receta Médica \\
\hline
\textbf{Versión} & 1.0 \\
\hline
\textbf{Descripción} &
Permite al médico generar una receta médica electrónica a partir de un tratamiento farmacológico prescrito, asegurando su validez clínica y legal.\\
\hline
\textbf{Actores} &
Medico (actor principal) \newline
Sistema \\
\hline
\textbf{Tipo} & Primario \\
\hline
\textbf{Referencias} & RF-015: Imprimir Receta Médica \\
\hline
\textbf{Flujo Básico} &
1.	El médico selecciona el tratamiento farmacológico.\newline
2.	Selecciona la opción "Generar Receta Médica".\newline
3.	El sistema genera la receta electrónica.\newline
4.	El sistema guarda y muestra la receta.\\
\hline
\textbf{Flujo Alternativo} &
A1: Tratamiento no farmacológico\newline
El sistema no permite generar receta médica.\\
\hline
\textbf{Requerimientos Especiales} &
Generación de receta en formato PDF.\newline
Inclusión de firma digital del médico.\\
\hline
\textbf{Pre-Condiciones} &
Tratamiento farmacológico registrado.\\
\hline
\textbf{Post-Condiciones} &
Receta médica generada.\\
\hline
\textbf{Puntos de Extensión} &
R1-CU-28 – Exportar Reportes\\
\hline
\end{tabular}
\end{table}

%16
\begin{table}[H]
\centering
\caption{Especificación del Caso de Uso R1-CU-16}
\vspace{3mm}
\small
\begin{tabular}{|p{4cm}|p{10cm}|}
\hline
\multicolumn{2}{|l|}{\textbf{Caso de Uso: CU-16 -- Registrar Médico}} \\
\hline
\textbf{Código} & R1-CU-16 \\
\hline
\textbf{Nombre} & Registrar Médico \\
\hline
\textbf{Versión} & 1.0 \\
\hline
\textbf{Descripción} &
Permite al administrador registrar médicos en el sistema, incluyendo datos profesionales, especialidades y número de colegiatura.\\
\hline
\textbf{Actores} &
Administrador (actor principal) \newline
Sistema \\
\hline
\textbf{Tipo} & Primario \\
\hline
\textbf{Referencias} & RF-017: Registrar M´edico con Credenciales\\
\hline
\textbf{Flujo Básico} &
1.	El administrador accede al módulo de gestión de personal médico. \newline
2.	Selecciona la opción "Registrar Médico". \newline
3.	Ingresa los datos profesionales requeridos. \newline
4.	El sistema valida la información. \newline
5.	El sistema registra al médico.\\
\hline
\textbf{Flujo Alternativo} &
A1: CMP duplicado\newline
El sistema muestra un mensaje de error.\\
\hline
\textbf{Requerimientos Especiales} &
Validación de número de colegiatura.\\
\hline
\textbf{Pre-Condiciones} &
Administrador autenticado.\\
\hline
\textbf{Post-Condiciones} &
Médico registrado correctamente.\\
\hline
\textbf{Puntos de Extensión} &
R1-CU-17 – Gestionar Horarios Médicos\\
\hline
\end{tabular}
\end{table}

%17
\begin{table}[H]
\centering
\caption{Especificación del Caso de Uso R1-CU-17}
\vspace{3mm}
\small
\begin{tabular}{|p{4cm}|p{10cm}|}
\hline
\multicolumn{2}{|l|}{\textbf{Caso de Uso: CU-17 -- Gestionar Horarios Médicos}} \\
\hline
\textbf{Código} & R1-CU-17 \\
\hline
\textbf{Nombre} & Gestionar Horarios Médicos \\
\hline
\textbf{Versión} & 1.0 \\
\hline
\textbf{Descripción} &
Permite al administrador asignar y modificar los horarios, turnos y guardias del personal médico, asegurando una correcta planificación del servicio. \\
\hline
\textbf{Actores} &
Administrador (actor principal) \newline
Sistema \\
\hline
\textbf{Tipo} & Primario \\
\hline
\textbf{Referencias} & 
RF-018: Asignar Horarios de Atención \newline
RF-019: Gestionar Turnos y Guardias\\
\hline
\textbf{Flujo Básico} &
1.	El administrador selecciona al médico.\newline
2.	Accede a la opción "Gestionar Horarios".\newline
3.	Define turnos, días y horarios.\newline
4.	El sistema valida conflictos de agenda.\newline
5.	El sistema guarda los horarios asignados. \\
\hline
\textbf{Flujo Alternativo} &
A1: Conflicto de horario\newline
El sistema solicita ajustar los horarios.\\
\hline
\textbf{Requerimientos Especiales} &
Validación automática de solapamientos.\\
\hline
\textbf{Pre-Condiciones} &
Médico registrado en el sistema.\\
\hline
\textbf{Post-Condiciones} &
Horarios médicos asignados correctamente.\\
\hline
\textbf{Puntos de Extensión} &
R1-CU-05 – Programar Cita Médica\\
\hline
\end{tabular}
\end{table}


%18
\begin{table}[H]
\centering
\caption{Especificación del Caso de Uso R1-CU-18}
\vspace{3mm}
\small
\begin{tabular}{|p{4cm}|p{10cm}|}
\hline
\multicolumn{2}{|l|}{\textbf{Caso de Uso: CU-18 -- Consultar Productividad Médica}} \\
\hline
\textbf{Código} & R1-CU-18 \\
\hline
\textbf{Nombre} & Consultar Productividad Médica \\
\hline
\textbf{Versión} & 1.0 \\
\hline
\textbf{Descripción} &
Permite al administrador consultar indicadores de productividad del personal médico, tales como número de consultas realizadas, citas atendidas y rendimiento por periodo, apoyando la toma de decisiones administrativas. \\
\hline
\textbf{Actores} &
Administrador (actor principal) \newline
Sistema \\
\hline
\textbf{Tipo} & Primario \\
\hline
\textbf{Referencias} & RF-020: Consultar Estad´ısticas de Productividad \\
\hline
\textbf{Flujo Básico} &
1.	El administrador accede al módulo de gestión de personal médico. \newline
2.	Selecciona la opción "Consultar Productividad Médica". \newline
3.	Define el periodo y criterios de análisis. \newline
4.	El sistema procesa la información. \newline
5.	El sistema muestra los indicadores de productividad.\\
\hline
\textbf{Flujo Alternativo} &
A1: Sin datos disponibles\newline
El sistema muestra un mensaje informativo.\\
\hline
\textbf{Requerimientos Especiales} &
Cálculo automático de indicadores.\newline
Visualización gráfica de resultados.\\
\hline
\textbf{Pre-Condiciones} &
Médico registrado en el sistema.\\
\hline
\textbf{Post-Condiciones} &
Productividad médica visualizada.\\
\hline
\textbf{Puntos de Extensión} &
R1-CU-27 – Generar Reportes\\
\hline
\end{tabular}
\end{table}

%19
\begin{table}[H]
\centering
\caption{Especificación del Caso de Uso R1-CU-19}
\vspace{3mm}
\small
\begin{tabular}{|p{4cm}|p{10cm}|}
\hline
\multicolumn{2}{|l|}{\textbf{Caso de Uso: CU-19 -- Registrar Medicamento}} \\
\hline
\textbf{Código} & R1-CU-19 \\
\hline
\textbf{Nombre} & Registrar Medicamento \\
\hline
\textbf{Versión} & 1.0 \\
\hline
\textbf{Descripción} &
Permite al personal de farmacia registrar medicamentos en el sistema, incluyendo información básica, presentación y stock inicial, para su posterior control y dispensación. \\
\hline
\textbf{Actores} &
Farmaceutico (actor principal) \newline
Sistema \\
\hline
\textbf{Tipo} & Primario \\
\hline
\textbf{Referencias} & RF-021: Registrar Medicamento en Catálogo \\
\hline
\textbf{Flujo Básico} &
1.	El farmacéutico accede al módulo de inventario.\newline
2.	Selecciona la opción "Registrar Medicamento".\newline
3.	Ingresa nombre, presentación, lote y stock inicial.\newline
4.	El sistema valida la información.\newline
5.	El sistema registra el medicamento\\
\hline
\textbf{Flujo Alternativo} &
A1: Medicamento duplicado\newline
El sistema notifica que el medicamento ya existe.\\
\hline
\textbf{Requerimientos Especiales} &
Control de fechas de vencimiento.\newline
Registro del usuario responsable.\\
\hline
\textbf{Pre-Condiciones} &
Usuario de farmacia autenticado\\
\hline
\textbf{Post-Condiciones} &
Medicamento registrado correctamente\\
\hline
\textbf{Puntos de Extensión} &
R1-CU-20 – Gestionar Movimientos de Inventario\\
\hline
\end{tabular}
\end{table}

%20
\begin{table}[H]
\centering
\caption{Especificación del Caso de Uso R1-CU-20}
\vspace{3mm}
\small
\begin{tabular}{|p{4cm}|p{10cm}|}
\hline
\multicolumn{2}{|l|}{\textbf{Caso de Uso: CU-20 -- Gestionar Movimientos de Inventario}} \\
\hline
\textbf{Código} & R1-CU-20 \\
\hline
\textbf{Nombre} & Gestionar Movimientos de Inventario \\
\hline
\textbf{Versión} & 1.0 \\
\hline
\textbf{Descripción} &
Permite al personal de farmacia registrar las entradas y salidas de medicamentos, garantizando el control del stock, la trazabilidad de los movimientos y la correcta gestión del inventario. \\
\hline
\textbf{Actores} &
Farmacéutico (actor principal) \newline
Sistema \\
\hline
\textbf{Tipo} & Primario \\
\hline
\textbf{Referencias} & 
RF-022: Registrar Entrada de Inventario\newline
RF-023: Registrar Salida de Inventario\\
\hline
\textbf{Flujo Básico} &
1.	El farmacéutico accede al módulo de inventario.\newline
2.	Selecciona el medicamento correspondiente.\newline
3.	Indica el tipo de movimiento (entrada o salida).\newline
4.	Ingresa la cantidad a registrar.\newline
5.	El sistema valida el stock disponible.\newline
6.	El sistema actualiza el inventario.\newline
7.	El sistema registra el movimiento.\\
\hline
\textbf{Flujo Alternativo} &
A1: Stock insuficiente\newline
El sistema impide registrar la salida y muestra un mensaje de alerta.\\
\hline
\textbf{Requerimientos Especiales} &
Actualización de stock en tiempo real.\newline
Registro automático para auditoría.\\
\hline
\textbf{Pre-Condiciones} &
Medicamento previamente registrado.\\
\hline
\textbf{Post-Condiciones} &
Inventario actualizado correctamente.\\
\hline
\textbf{Puntos de Extensión} &
R1-CU-21 – Consultar Inventario\\
\hline
\end{tabular}
\end{table}

%21
\begin{table}[H]
\centering
\caption{Especificación del Caso de Uso R1-CU-21}
\vspace{3mm}
\small
\begin{tabular}{|p{4cm}|p{10cm}|}
\hline
\multicolumn{2}{|l|}{\textbf{Caso de Uso: CU-21 -- Consultar Inventario}} \\
\hline
\textbf{Código} & R1-CU-21 \\
\hline
\textbf{Nombre} & Consultar Inventario \\
\hline
\textbf{Versión} & 1.0 \\
\hline
\textbf{Descripción} &
Permite al personal autorizado consultar el inventario de medicamentos, visualizando cantidades disponibles, estado de stock y alertas de stock mínimo. \\
\hline
\textbf{Actores} &
Farmacéutico (actor principal) \newline
Administrador \\
\hline
\textbf{Tipo} & Primario \\
\hline
\textbf{Referencias} & RF-024: Generar Alerta de Stock Mínimo \\
\hline
\textbf{Flujo Básico} &
1.	El farmacéutico accede al módulo de inventario. \newline
2.	Selecciona el medicamento correspondiente. \newline
3.	Indica el tipo de movimiento (entrada o salida). \newline
4.	Ingresa la cantidad a registrar. \newline
5.	El sistema valida el stock disponible. \newline
6.	El sistema actualiza el inventario. \newline
7.	El sistema registra el movimiento.\\
\hline
\textbf{Flujo Alternativo} &
A1: Stock insuficiente\newline
El sistema impide registrar la salida y muestra un mensaje de alerta.\\
\hline
\textbf{Requerimientos Especiales} &
Actualización de stock en tiempo real.  \newline
Registro automático para auditoría.\\
\hline
\textbf{Pre-Condiciones} &
Medicamento previamente registrado.\\
\hline
\textbf{Post-Condiciones} &
Inventario actualizado correctamente.\\
\hline
\textbf{Puntos de Extensión} &
R1-CU-21 – Consultar Inventario\\
\hline
\end{tabular}
\end{table}

%22
\begin{table}[H]
\centering
\caption{Especificación del Caso de Uso R1-CU-22}
\vspace{3mm}
\small
\begin{tabular}{|p{4cm}|p{10cm}|}
\hline
\multicolumn{2}{|l|}{\textbf{Caso de Uso: CU-22 -- Autenticar Usuario}} \\
\hline
\textbf{Código} & R1-CU-22 \\
\hline
\textbf{Nombre} & Autenticar Usuario \\
\hline
\textbf{Versión} & 1.0 \\
\hline
\textbf{Descripción} &
Permite al personal de admisión modificar o actualizar la información personal y de contacto de un paciente previamente registrado, garantizando que los datos se mantengan actualizados y confiables. \\
\hline
\textbf{Actores} &
Personal de Admisión (actor principal) \newline
Sistema \\
\hline
\textbf{Tipo} & Primario \\
\hline
\textbf{Referencias} & RF-025: Controlar Fecha de Caducidad \\
\hline
\textbf{Flujo Básico} &
1.	El usuario accede al módulo de inventario.\newline
2.	Selecciona la opción “Consultar Inventario”.\newline
3.	El sistema muestra el listado de medicamentos con su stock.\\
\hline
\textbf{Flujo Alternativo} &
A1: Inventario sin registros\newline
El sistema muestra un mensaje informativo.\\
\hline
\textbf{Requerimientos Especiales} &
Alertas automáticas por stock mínimo.\\
\hline
\textbf{Pre-Condiciones} &
Usuario autenticado.\\
\hline
\textbf{Post-Condiciones} &
Inventario consultado correctamente.\\
\hline
\textbf{Puntos de Extensión} &
R1-CU-19 – Registrar Medicamento.\\
\hline
\end{tabular}
\end{table}

%23
\begin{table}[H]
\centering
\caption{Especificación del Caso de Uso R1-CU-23}
\vspace{3mm}
\small
\begin{tabular}{|p{4cm}|p{10cm}|}
\hline
\multicolumn{2}{|l|}{\textbf{Caso de Uso: CU-23 -- Gestionar Usuarios y Roles}} \\
\hline
\textbf{Código} & R1-CU-23 \\
\hline
\textbf{Nombre} & Gestionar Usuarios y Roles \\
\hline
\textbf{Versión} & 1.0 \\
\hline
\textbf{Descripción} &
Permite al administrador del sistema crear, modificar, desactivar y administrar las cuentas de usuario del SIGHC, así como definir y asignar roles y permisos específicos. Este caso de uso es fundamental para garantizar que cada usuario acceda únicamente a las funcionalidades que le corresponden según su perfil (administrativo, médico, farmacia, dirección, entre otros), fortaleciendo la seguridad, el control interno y el correcto uso del sistema. \\
\hline
\textbf{Actores} &
Administrador del Sistema (actor principal) \newline
Sistema \\
\hline
\textbf{Tipo} & Primario \\
\hline
\textbf{Referencias} & 
RF-026: Generar Kardex Valorizado\newline
RF-027: Autenticación de Usuario \\
\hline
\textbf{Flujo Básico} &
1.	El administrador accede al módulo de Seguridad y Usuarios.\newline
2.	Selecciona la opción "Gestionar Usuarios y Roles".\newline
3.	El sistema muestra el listado de usuarios registrados.\newline
4.	El administrador puede crear un nuevo usuario o seleccionar uno existente.\newline
5.	El sistema muestra el formulario con los datos del usuario.\newline
6.	El administrador asigna o modifica el rol correspondiente y los permisos asociados.\newline
7.	El sistema valida la coherencia de los permisos asignados.\newline
8.	El sistema guarda los cambios realizados.\newline
9.	El sistema confirma la operación exitosa. \\
\hline
\textbf{Flujo Alternativo} &
A1: Rol inexistente o inválido\newline
En el paso 6, si el rol no existe, el sistema muestra un mensaje de error y solicita seleccionar un rol válido. \\
\hline
\textbf{Requerimientos Especiales} &
Control de acceso basado en roles (RBAC).\newline
Registro de auditoría de creación y modificación de usuarios.\newline
Restricción de acceso exclusivo para administradores. \\
\hline
\textbf{Pre-Condiciones} &
El administrador debe estar autenticado y activo. \\
\hline
\textbf{Post-Condiciones} &
Usuarios y roles gestionados correctamente en el sistema. \\
\hline
\textbf{Puntos de Extensión} &
R1-CU-25 – Registrar Auditoría \\
\hline
\end{tabular}
\end{table}

%24
\begin{table}[H]
\centering
\caption{Especificación del Caso de Uso R1-CU-24}
\vspace{3mm}
\small
\begin{tabular}{|p{4cm}|p{10cm}|}
\hline
\multicolumn{2}{|l|}{\textbf{Caso de Uso: CU-24 -- Gestionar Seguridad de Acceso}} \\
\hline
\textbf{Código} & R1-CU-24 \\
\hline
\textbf{Nombre} & Gestionar Seguridad de Acceso \\
\hline
\textbf{Versión} & 1.0 \\
\hline
\textbf{Descripción} &
Permite al sistema y al administrador definir, aplicar y controlar las políticas de seguridad de acceso al SIGHC, tales como control de intentos fallidos, bloqueo y desbloqueo de cuentas, gestión de contraseñas y aplicación de normas de seguridad informática, con el fin de proteger la información clínica y administrativa. \\
\hline
\textbf{Actores} &
Administrador del Sistema (actor principal) \newline
Sistema \\
\hline
\textbf{Tipo} & Primario \\
\hline
\textbf{Referencias} & RF-028: Gesti´on de Roles RBAC \\
\hline
\textbf{Flujo Básico} &
1.	El sistema monitorea permanentemente los intentos de acceso de los usuarios.\newline
2.	Detecta intentos fallidos consecutivos de autenticación.\newline
3.	El sistema aplica automáticamente las políticas de seguridad configuradas.\newline
4.	En caso de bloqueo, el sistema registra el evento.\newline
5.	El sistema notifica al administrador del incidente de seguridad. \\
\hline
\textbf{Flujo Alternativo} &
A1:Desbloqueo manual de cuenta\newline
El administrador accede al módulo de seguridad y desbloquea la cuenta afectada. \\
\hline
\textbf{Requerimientos Especiales} &
Configuración parametrizable de políticas de seguridad.\newline
Registro detallado de eventos de seguridad. \\
\hline
\textbf{Pre-Condiciones} &
Usuario previamente registrado en el sistema. \\
\hline
\textbf{Post-Condiciones} &
Políticas de seguridad aplicadas correctamente. \\
\hline
\textbf{Puntos de Extensión} &
R1-CU-22 – Autenticar Usuario \\
\hline
\end{tabular}
\end{table}

%25
\begin{table}[H]
\centering
\caption{Especificación del Caso de Uso R1-CU-25}
\vspace{3mm}
\small
\begin{tabular}{|p{4cm}|p{10cm}|}
\hline
\multicolumn{2}{|l|}{\textbf{Caso de Uso: CU-25 -- Registrar Auditoría}} \\
\hline
\textbf{Código} & R1-CU-25 \\
\hline
\textbf{Nombre} & ARegistrar Auditoría\\
\hline
\textbf{Versión} & 1.0 \\
\hline
\textbf{Descripción} &
Permite al sistema registrar de forma automática y transparente todas las acciones relevantes realizadas por los usuarios dentro del SIGHC, garantizando la trazabilidad de las operaciones, la seguridad de la información y el cumplimiento de normativas internas y externas.\\
\hline
\textbf{Actores} &
Sistema  (actor principal) \\
\hline
\textbf{Tipo} & Secundario / Automatizado \\
\hline
\textbf{Referencias} & 
RF-029: Cambio de Contraseña Obligatorio\newline
RF-030: Cierre Autom´atico de Sesion \\
\hline
\textbf{Flujo Básico} &
1.	Un usuario ejecuta una acción relevante dentro del sistema.\newline
2.	El sistema identifica el tipo de operación realizada.\newline
3.	El sistema registra automáticamente el usuario, \newlinefecha, hora, módulo y descripción de la acción.\newline
4.	La información se almacena de manera segura en la bitácora de auditoría. \\
\hline
\textbf{Flujo Alternativo} &
A1: Error de almacenamiento\newline
El sistema genera una alerta administrativa. \\
\hline
\textbf{Requerimientos Especiales} &
Un usuario ejecuta una acción relevante dentro del sistema.\newline
El sistema identifica el tipo de operación realizada.\newline
El sistema registra automáticamente el usuario, fecha, hora, módulo y descripción de la acción.\newline
La información se almacena de manera segura en la bitácora de auditoría. \\
\hline
\textbf{Pre-Condiciones} &
Usuario autenticado. \\
\hline
\textbf{Post-Condiciones} &
Acción registrada correctamente en la bitácora. \\
\hline
\textbf{Puntos de Extensión} &
R1-CU-26 – Consultar Bitácora de Auditoría \\
\hline
\end{tabular}
\end{table}

%26
\begin{table}[H]
\centering
\caption{Especificación del Caso de Uso R1-CU-26}
\vspace{3mm}
\small
\begin{tabular}{|p{4cm}|p{10cm}|}
\hline
\multicolumn{2}{|l|}{\textbf{Caso de Uso: CU-26 -- Consultar Bitácora de Auditoría}} \\
\hline
\textbf{Código} & R1-CU-26 \\
\hline
\textbf{Nombre} & Consultar Bitácora de Auditoría \\
\hline
\textbf{Versión} & 1.0 \\
\hline
\textbf{Descripción} &
Permite al administrador del sistema consultar, analizar y supervisar los registros de auditoría generados, utilizando filtros avanzados que facilitan la detección de incidentes, errores operativos o accesos indebidos.\\
\hline
\textbf{Actores} &
Administrador del Sistema (actor principal) \\
\hline
\textbf{Tipo} & Primario \\
\hline
\textbf{Referencias} & RF-031: Auditoría Automática de Cambios \\
\hline
\textbf{Flujo Básico} &
1.	El administrador accede al módulo de Auditoría.\newline
2.	Selecciona la opción "Consultar Bitácora de Auditoría".\newline
3.	Define criterios de búsqueda (usuario, fecha, módulo o tipo de acción).\newline
4.	El sistema procesa la consulta.\newline
5.	El sistema muestra los registros encontrados. \\
\hline
\textbf{Flujo Alternativo} &
A1: No existen registros\newline
El sistema muestra un mensaje informativo. \\
\hline
\textbf{Requerimientos Especiales} &
Acceso restringido solo a administradores.\newline
Visualización clara y ordenada de registros. \\
\hline
\textbf{Pre-Condiciones} &
Administrador autenticado. \\
\hline
\textbf{Post-Condiciones} &
Registros de auditoría consultados. \\
\hline
\textbf{Puntos de Extensión} &
R1-CU-25 – Registrar Auditoría \\
\hline
\end{tabular}
\end{table}

%27
\begin{table}
\centering
\caption{Especificación del Caso de Uso R1-CU-27}
\vspace{3mm}
\small
\begin{tabular}{|p{4cm}|p{10cm}|}
\hline
\multicolumn{2}{|l|}{\textbf{Caso de Uso: CU-27 -- Generar Reportes}} \\
\hline
\textbf{Código} & R1-CU-27 \\
\hline
\textbf{Nombre} & Generar Reportes \\
\hline
\textbf{Versión} & 1.0 \\
\hline
\textbf{Descripción} &
Permite generar reportes estadísticos y operativos del sistema, apoyando la toma de decisiones administrativas y clínicas. \\
\hline
\textbf{Actores} &
Administrador \newline
Directivo \\
\hline
\textbf{Tipo} & Primario \\
\hline
\textbf{Referencias} & 
RF-033: Generar Reporte de Auditor´ıa por Usuario.\newline
RF-034: Auditor´ıa de Acceso a Datos Sensibles.\newline
RF-037: Estad´ısticas de Diagn´osticos Frecuentes. \\
\hline
\textbf{Flujo Básico} &
1.	El usuario accede al módulo de reportes.\newline
2.	Selecciona el tipo de reporte.\newline
3.	Define criterios y período.\newline
4.	El sistema procesa la información.\newline
5.	El sistema genera el reporte. \\
\hline
\textbf{Flujo Alternativo} &
\\
\hline
\textbf{Requerimientos Especiales} &
Datos consolidados y confiables. \\
\hline
\textbf{Pre-Condiciones} &
Usuario autorizado. \\
\hline
\textbf{Post-Condiciones} &
Reporte generado. \\
\hline
\textbf{Puntos de Extensión} &
R1-CU-28 – Exportar Reportes \\
\hline
\end{tabular}
\end{table}

%28
\begin{table}
\centering
\caption{Especificación del Caso de Uso R1-CU-28}
\vspace{3mm}
\small
\begin{tabular}{|p{4cm}|p{10cm}|}
\hline
\multicolumn{2}{|l|}{\textbf{Caso de Uso: CU-28 -- Exportar Reportes}} \\
\hline
\textbf{Código} & R1-CU-28 \\
\hline
\textbf{Nombre} & Exportar Reportes \\
\hline
\textbf{Versión} & 1.0 \\
\hline
\textbf{Descripción} &
Permite exportar los reportes generados a formatos digitales para su análisis externo. \\
\hline
\textbf{Actores} &
Administrador  \\
\hline
\textbf{Tipo} & Primario \\
\hline
\textbf{Referencias} & RF-035: Dashboard con KPIs en Tiempo Real \\
\hline
\textbf{Flujo Básico} &
1.	El usuario selecciona un reporte generado.\newline
2.	Elige el formato de exportación.\newline
3.	El sistema genera el archivo descargable. \\
\hline
\textbf{Flujo Alternativo} &
\\
\hline
\textbf{Requerimientos Especiales} &
Compatibilidad con PDF y Excel. \\
\hline
\textbf{Pre-Condiciones} &
Reporte previamente generado. \\
\hline
\textbf{Post-Condiciones} &
Reporte exportado. \\
\hline
\textbf{Puntos de Extensión} &
\\
\hline
\end{tabular}
\end{table}


%29
\begin{table}
\centering
\caption{Especificación del Caso de Uso R1-CU-29}
\vspace{3mm}
\small
\begin{tabular}{|p{4cm}|p{10cm}|}
\hline
\multicolumn{2}{|l|}{\textbf{Caso de Uso: CU-29 -- Visualizar Dashboards}} \\
\hline
\textbf{Código} & R1-CU-29 \\
\hline
\textbf{Nombre} & Visualizar Dashboards \\
\hline
\textbf{Versión} & 1.0 \\
\hline
\textbf{Descripción} &
Permite visualizar indicadores clave del sistema mediante tableros gráficos interactivos. \\
\hline
\textbf{Actores} &
Administrador\newline
Directivo \\
\hline
\textbf{Tipo} & Primario \\
\hline
\textbf{Referencias} & RF-036: Reporte de Citas por Especialidad \\
\hline
\textbf{Flujo Básico} &
1.	El usuario accede al módulo de dashboards.\newline
2.	El sistema carga los indicadores.\newline
3.	El sistema muestra gráficos y métricas. \\
\hline
\textbf{Flujo Alternativo} &
\\
\hline
\textbf{Requerimientos Especiales} &
Actualización en tiempo real. \\
\hline
\textbf{Pre-Condiciones} &
Usuario autorizado. \\
\hline
\textbf{Post-Condiciones} &
Dashboards visualizados. \\
\hline
\textbf{Puntos de Extensión} &
\\
\hline
\end{tabular}
\end{table}

%30
\begin{table}
\centering
\caption{Especificación del Caso de Uso R1-CU-30}
\vspace{3mm}
\small
\begin{tabular}{|p{4cm}|p{10cm}|}
\hline
\multicolumn{2}{|l|}{\textbf{Caso de Uso: CU-30 -- Ejecutar Respaldo de Base de Datos}} \\
\hline
\textbf{Código} & R1-CU-30 \\
\hline
\textbf{Nombre} & Ejecutar Respaldo de Base de Datos \\
\hline
\textbf{Versión} & 1.0 \\
\hline
\textbf{Descripción} &
Permite realizar copias de seguridad de la base de datos del sistema para garantizar la continuidad operativa. \\
\hline
\textbf{Actores} &
Administrador del Sistema \\
\hline
\textbf{Tipo} & Primario \\
\hline
\textbf{Referencias} &
RF-038: Reporte de Productividad Médica.\newline
RF-039: Exportar Reportes a Excel y PDF \\
\hline
\textbf{Flujo Básico} &
1.	El administrador accede al módulo de respaldo.\newline
2.	Selecciona la opción "Ejecutar Respaldo".\newline
3.	El sistema crea la copia de seguridad.\newline
4.	El sistema confirma la operación. \\
\hline
\textbf{Flujo Alternativo} &
\\
\hline
\textbf{Requerimientos Especiales} &
Almacenamiento seguro del respaldo. \\
\hline
\textbf{Pre-Condiciones} &
Administrador autenticado. \\
\hline
\textbf{Post-Condiciones} &
Respaldo generado. \\
\hline
\textbf{Puntos de Extensión} &
\\
\hline
\end{tabular}
\end{table}

%31
\begin{table}
\centering
\caption{Especificación del Caso de Uso R1-CU-31}
\vspace{3mm}
\small
\begin{tabular}{|p{4cm}|p{10cm}|}
\hline
\multicolumn{2}{|l|}{\textbf{Caso de Uso: CU-31 -- Auditar Accesos Clínicos}} \\
\hline
\textbf{Código} & R1-CU-31 \\
\hline
\textbf{Nombre} & Auditar Accesos Clínicos \\
\hline
\textbf{Versión} & 1.0 \\
\hline
\textbf{Descripción} &
Permite al personal de admisión modificar o actualizar la información personal y de contacto de un paciente previamente registrado, garantizando que los datos se mantengan actualizados y confiables. \\
\hline
\textbf{Actores} &
Administrador de Sistema \\
\hline
\textbf{Tipo} & Primario \\
\hline
\textbf{Referencias} & RF-039: Exportar Reportes a Excel y PDF \\
\hline
\textbf{Flujo Básico} &
1.	El administrador accede al módulo de auditoría clínica.\newline
2.	Define criterios de búsqueda.\newline
3.	El sistema muestra los accesos registrados. \\
\hline
\textbf{Flujo Alternativo} &
\\
\hline
\textbf{Requerimientos Especiales} &
Cumplimiento de normativas de protección de datos. \\
\hline
\textbf{Pre-Condiciones} &
Administrador autenticado. \\
\hline
\textbf{Post-Condiciones} &
Accesos clínicos auditados. \\
\hline
\textbf{Puntos de Extensión} &
\\
\hline
\end{tabular}
\end{table}

%32
\begin{table}
\centering
\caption{Especificación del Caso de Uso R1-CU-32}
\vspace{3mm}
\small
\begin{tabular}{|p{4cm}|p{10cm}|}
\hline
\multicolumn{2}{|l|}{\textbf{Caso de Uso: CU-32 -- Generar Dashboards}} \\
\hline
\textbf{Código} & R1-CU-32 \\
\hline
\textbf{Nombre} & Generar Dashboards \\
\hline
\textbf{Versión} & 1.0 \\
\hline
\textbf{Descripción} &
Permite al sistema generar dashboards dinámicos basados en indicadores configurables \\
\hline
\textbf{Actores} &
Administrador de Sistema \\
\hline
\textbf{Tipo} & Secundario / Automatizado \\
\hline
\textbf{Referencias} & 
RF-041: Backup Diferencial Semanal\newline
RF-042: Restaurar Base de Datos desde Backup  \\
\hline
\textbf{Flujo Básico} &
1.	El sistema recopila información de los módulos.\newline
2.	Procesa los indicadores definidos.\newline
3.	Actualiza los dashboards disponibles. \\
\hline
\textbf{Flujo Alternativo} &
\\
\hline
\textbf{Requerimientos Especiales} &
Procesamiento eficiente de datos. \\
\hline
\textbf{Pre-Condiciones} &
Datos disponibles en el sistema. \\
\hline
\textbf{Post-Condiciones} &
Dashboards generados y actualizados. \\
\hline\\
\hline
\end{tabular}
\end{table}

\FloatBarrier%%no borrar esto

\newpage
\section{Módulos del Sistema}
\subsection{Visión General del Sistema}
El Sistema Integral de Gestión de Historias Clínicas (SIGHC) está estructurado en módulos funcionales, cada uno responsable de un conjunto específico de procesos del negocio clínico y administrativo.
Esta modularización permite una mejor organización del sistema, facilita su mantenimiento, escalabilidad y asegura una clara separación de responsabilidades.

Cada módulo agrupa funcionalidades relacionadas, alineadas con los requisitos funcionales y los casos de uso definidos, garantizando un flujo de información coherente entre las áreas asistenciales, administrativas y de gestión.

\subsection{Descripción General de los Módulos}
\subsubsection{M01 -- Gestión de Pacientes e Historias Clínicas}
Este módulo permite el registro, actualización y consulta de la información personal de los pacientes, así como la creación y mantenimiento de su historia clínica electrónica.

Proporciona la base de información necesaria para el seguimiento médico, asegurando la integridad y confidencialidad de los datos clínicos.

Principales funcionalidades:

- Registro de pacientes\newline
- Modificación de datos personales\newline
- Búsqueda de pacientes\newline
- Consulta de historia clínica

\subsubsection{M02 – Gestión de Citas Médicas}
Este módulo gestiona el proceso completo de programación de citas médicas, permitiendo organizar la atención de los pacientes de manera eficiente. Controla la disponibilidad de los médicos y evita conflictos de horarios.

Principales funcionalidades:

- Programación, reprogramación y cancelación de citas\newline
- Consulta de agenda médica\newline
- Registro de asistencia de pacientes

\subsubsection{M03 – Consultas y Diagnósticos}
Este módulo permite registrar y consultar la información clínica generada durante la atención médica. Incluye el registro de consultas médicas y diagnósticos utilizando estándares internacionales.

Principales funcionalidades:

- Registro de consultas médicas\newline
- Registro y modificación de diagnósticos CIE-10\newline
- Consulta del historial de consultas

\subsubsection{M04 – Tratamientos y Prescripciones}
Este módulo administra los tratamientos médicos prescritos a los pacientes, asegurando la correcta indicación terapéutica y el control de riesgos clínicos.

Principales funcionalidades:

- Prescripción de tratamientos médicos\newline
- Consulta de tratamientos activos\newline
- Generación de recetas médicas electrónicas\newline
- Validación de alergias e interacciones medicamentosas

\subsubsection{M05 – Gestión de Personal Médico}
Este módulo permite la administración del personal médico del establecimiento, facilitando la organización de horarios, turnos y guardias.

Principales funcionalidades:

- Registro de médicos\newline
- Asignación de horarios médicos\newline
- Gestión de turnos y guardias\newline
- Consulta de productividad médica

\subsubsection{M06 – Inventario de Medicamentos}
Este módulo gestiona el control de medicamentos disponibles en farmacia, asegurando el abastecimiento oportuno y la correcta dispensación.

Principales funcionalidades:

- Registro de medicamentos\newline
- Registro de entradas y salidas\newline
- Control de stock y fechas de vencimiento

\subsubsection{M07 – Seguridad y Control de Acceso}
Este módulo garantiza la seguridad del sistema mediante mecanismos de autenticación, autorización y control de accesos.

Principales funcionalidades:

- Autenticación de usuarios\newline
- Gestión de roles y permisos\newline
- Aplicación de políticas de seguridad

\subsubsection{M08 – Auditoría y Trazabilidad}
Este módulo permite supervisar y auditar las acciones realizadas dentro del sistema, asegurando la transparencia y el cumplimiento normativo.

Principales funcionalidades:

- Registro automático de auditoría\newline
- Consulta de bitácoras de auditoría\newline
- Auditoría de accesos clínicos

\subsubsection{M09 – Reportes y Business Intelligence}
Este módulo proporciona herramientas de análisis y visualización de información para apoyar la toma de decisiones.

Principales funcionalidades:

- Generación de dashboards\newline
- Exportación de reportes\newline
- Análisis de productividad médica y diagnósticos frecuentes

\subsubsection{M10 – Respaldo y Recuperación}
Este módulo garantiza la disponibilidad y continuidad del sistema mediante mecanismos de respaldo y recuperación de la información.

Principales funcionalidades:

- Ejecución de respaldos automáticos\newline
- Restauración de la base de datos

\subsection{Módulos del Sistema}

El sistema SaaS de gestión para la clínica se encuentra organizado en módulos funcionales, los cuales permiten dividir el sistema en componentes bien definidos y coherentes. Cada módulo se enfoca en una necesidad específica del proceso clínico y administrativo, facilitando el uso del sistema, el mantenimiento de la información y la correcta atención al paciente. Esta organización modular contribuye a la escalabilidad, mantenibilidad y claridad operativa del sistema.

\subsubsection{Gestión de Pacientes e Historias Clínicas}

El módulo de Gestión de Pacientes e Historias Clínicas se encarga de administrar toda la información personal y médica de los pacientes de la clínica. A través de este módulo se realiza el registro de los pacientes mediante la creación de una historia clínica única, la cual consolida antecedentes médicos, consultas previas, diagnósticos y tratamientos realizados.

La información se mantiene actualizada y disponible para el personal autorizado, evitando la duplicidad de registros y la pérdida de datos. El uso de historias clínicas electrónicas mejora la continuidad de la atención médica, ya que el profesional de salud puede acceder de forma rápida y segura al historial completo del paciente, facilitando la toma de decisiones clínicas y reduciendo errores derivados del manejo de información incompleta.

\subsubsection{Gestión de Citas Médicas}
El módulo de Gestión de Citas Médicas permite organizar de manera eficiente la atención de los pacientes mediante una agenda, ya que se encarga de la programación, modificación y cancelación de citas, considerando la disponibilidad de los médicos y los horarios establecidos por la clínica.
Con este módulo se reducen los tiempos de espera y se evita el cruce de horarios, optimizando el uso de los recursos médicos. Asimismo, facilita el seguimiento del estado de las citas, permitiendo identificar aquellas que se encuentran programadas, atendidas o canceladas. La correcta gestión de citas contribuye a mejorar la experiencia del paciente y a mantener un orden adecuado en la atención diaria de la clínica.
\subsubsection{Consultas y Diagnósticos}
El módulo de Consultas y Diagnósticos indica el registro de la información clínica generada durante la atención médica. En este módulo, el médico documenta los síntomas del paciente, observaciones clínicas y el diagnóstico correspondiente. Toda la información se guarda en la historia clínica del paciente, permitiendo mantener un seguimiento adecuado del estado de su salud en sus próximas consultas. Además, el acceso a diagnósticos previos permite al personal médico realizar evaluaciones más precisas y brindar una atención más segura y eficiente.
\subsubsection{Tratamientos y Prescripciones}
El módulo de Tratamientos y Prescripciones permite gestionar las indicaciones médicas derivadas de los diagnósticos realizados durante la consulta. El médico puede registrar los tratamientos asignados al paciente, así como emitir prescripciones médicas de manera digital.
En este módulo se detallan los medicamentos prescritos, la dosis, la frecuencia y la duración del tratamiento, contribuyendo a la reducción de errores. Asimismo, el sistema mantiene un historial de tratamientos, lo que facilita el seguimiento del paciente y la evaluación de la efectividad del tratamiento. 

% =====================================================
% 6. MODELO DE DATOS - ARQUITECTURA
% =====================================================
\newpage
\section{Diseño de Base de Datos}

\subsection{Catálogo de Tablas del Sistema}


\begin{table}[H]
\centering
\small
\begin{tabular}{|l|c|p{8cm}|}
\hline
\tablaheader{Categoría} & \tablaheader{Cantidad} & \tablaheader{Tablas} \\
\hline
Entidades Principales & 5 & Pacientes, Medicos, Citas, Consultas, Diagnosticos \\
\hline
Catálogos & 4 & CIE10, Medicamentos, Especialidades, EstadosCita \\
\hline
Transaccionales & 4 & Tratamientos, Inventario Movimientos, Prescripciones, SignosVitales \\
\hline
Auditoría & 3 & AuditLog, Auditoria\_Pacientes, Auditoria\_Diagnosticos \\
\hline
Seguridad & 2 & Usuarios, Roles \\
\hline
\multicolumn{2}{|c|}{\textbf{TOTAL}} & \textbf{18 tablas} \\
\hline
\end{tabular}
\caption{Distribución de tablas por categoría}
\end{table}

\subsection{Diagrama de Relaciones Entre Entidades (ERD)}

\subsubsection{Matriz de Relaciones del Sistema}

\begin{longtable}{|p{3.5cm}|p{3.5cm}|p{1.5cm}|p{6cm}|}
\hline
\tablaheader{Tabla Origen} & \tablaheader{Tabla Destino} & \tablaheader{Cardinalidad} & \tablaheader{Descripción de la Relación} \\
\hline
\endfirsthead
\hline
\tablaheader{Tabla Origen} & \tablaheader{Tabla Destino} & \tablaheader{Cardinalidad} & \tablaheader{Descripción} \\
\hline
\endhead

Pacientes & Citas & 1:N & Un paciente puede tener múltiples citas a lo largo del tiempo. \\
\hline
Pacientes & Consultas & 1:N & Un paciente puede tener múltiples consultas médicas. \\
\hline
Pacientes & Usuarios & N:1 & Múltiples pacientes son registrados por un usuario. \\
\hline
Medicos & Especialidades & N:1 & Muchos médicos pertenecen a una especialidad. \\
\hline
Medicos & Citas & 1:N & Un médico atiende múltiples citas. \\
\hline
Medicos & Consultas & 1:N & Un médico realiza múltiples consultas. \\
\hline
Medicos & HorariosAtencion & 1:N & Un médico tiene múltiples horarios de atención configurados. \\
\hline
Citas & Consultas & 1:1 & Una cita programada genera exactamente una consulta médica. \\
\hline
Citas & Usuarios & N:1 & Múltiples citas son programadas por un usuario. \\
\hline
Consultas & Diagnosticos & 1:N & Una consulta puede tener uno o más diagnósticos médicos. \\
\hline
Diagnosticos & CIE10 & N:1 & Múltiples diagnósticos usan un código CIE-10. \\
\hline
Diagnosticos & Tratamientos & 1:N & Un diagnóstico puede tener múltiples tratamientos prescritos. \\
\hline
Tratamientos & Medicamentos & N:1 & Múltiples tratamientos usan un medicamento. \\
\hline
Medicamentos & Inventario Movimientos & 1:N & Un medicamento tiene múltiples movimientos de inventario. \\
\hline
Medicamentos & Lotes & 1:N & Un medicamento puede tener múltiples lotes con diferentes vencimientos. \\
\hline
Lotes & Inventario Movimientos & 1:N & Un lote tiene múltiples movimientos asociados. \\
\hline
Inventario Movimientos & Usuarios & N:1 & Múltiples movimientos son registrados por un usuario. \\
\hline
Usuarios & Roles & N:1 & Múltiples usuarios tienen un rol asignado. \\
\hline
Roles & Permisos & 1:N & Un rol tiene múltiples permisos granulares asignados. \\
\hline
AuditLog & Usuarios & N:1 & Múltiples registros de auditoría pertenecen a un usuario. \\
\hline
Auditoria\_Pacientes & Pacientes & N:1 & Múltiples registros de auditoría corresponden a un paciente. \\
\hline
Auditoria\_Pacientes & Usuarios & N:1 & Múltiples auditorías son realizadas por un usuario. \\
\hline
Auditoria\_Diagnosticos & Diagnosticos & N:1 & Múltiples auditorías corresponden a un diagnóstico. \\
\hline
Auditoria\_Diagnosticos & Usuarios & N:1 & Múltiples auditorías de diagnósticos son realizadas por un usuario (médico). \\
\hline
\caption{Matriz completa de relaciones entre las 18 tablas del SIGHC}
\end{longtable}

\subsubsection{Descripción de Cardinalidades}

\begin{itemize}[leftmargin=1.5em]
    \item \textbf{1:1 (Uno a Uno):} Una entidad de A se relaciona con exactamente una entidad de B.
    \begin{itemize}
        \item Ejemplo: Citas ↔ Consultas (Una cita genera una consulta única)
    \end{itemize}
    
    \item \textbf{1:N (Uno a Muchos):} Una entidad de A se relaciona con múltiples entidades de B.
    \begin{itemize}
        \item Ejemplo: Pacientes → Citas (Un paciente tiene muchas citas)
        \item Ejemplo: Medicos → Consultas (Un médico realiza muchas consultas)
    \end{itemize}
    
    \item \textbf{N:1 (Muchos a Uno):} Múltiples entidades de A se relacionan con una entidad de B.
    \begin{itemize}
        \item Ejemplo: Medicos → Especialidades (Muchos médicos, una especialidad)
        \item Ejemplo: Diagnosticos → CIE10 (Muchos diagnósticos usan un código CIE-10)
    \end{itemize}
\end{itemize}

\subsubsection{Reglas de Integridad Referencial}

\textbf{ON DELETE CASCADE:} Se aplica en las siguientes relaciones:
\begin{itemize}[leftmargin=1.5em]
    \item Consultas → Diagnosticos (Si se elimina una consulta, se eliminan sus diagnósticos)
    \item Diagnosticos → Tratamientos (Si se elimina un diagnóstico, se eliminan sus tratamientos)
    \item Medicos → HorariosAtencion (Si se elimina un médico, se eliminan sus horarios)
\end{itemize}

\textbf{ON DELETE RESTRICT:} Se aplica en las siguientes relaciones:
\begin{itemize}[leftmargin=1.5em]
    \item Pacientes (No se puede eliminar si tiene citas o consultas)
    \item Medicos (No se puede eliminar si tiene citas programadas)
    \item Medicamentos (No se puede eliminar si tiene movimientos de inventario)
    \item Roles (No se puede eliminar si tiene usuarios asignados)
\end{itemize}

\textbf{ON UPDATE CASCADE:} Se aplica en todas las claves foráneas para mantener consistencia.

\subsection{Normalización del Modelo de Datos}

\subsubsection{Cumplimiento de las Formas Normales}

\textbf{A. Primera Forma Normal (1NF):}
\begin{itemize}[leftmargin=1.5em]
    \item Todos los campos contienen valores atómicos
    \item No existen grupos repetitivos
    \item Cada tabla tiene Primary Key definida
\end{itemize}

\textbf{B. Segunda Forma Normal (2NF):}
\begin{itemize}[leftmargin=1.5em]
    \item  Cumple 1NF
    \item  Todos los atributos no clave dependen completamente de la PK
    \item No existen dependencias parciales
\end{itemize}

\textbf{C. Tercera Forma Normal (3NF):}
\begin{itemize}[leftmargin=1.5em]
    \item Cumple 2NF
    \item No existen dependencias transitivas
    \item  Los catálogos están separados (Especialidades, CIE10, Medicamentos)
\end{itemize}

\textbf{D. Excepciones controladas (Desnormalización intencional):}
\begin{itemize}[leftmargin=1.5em]
    \item \textbf{Consultas.IdPaciente e IdMedico:} Se desnormalizan para optimizar consultas frecuentes de historial médico (evita JOIN adicional con Citas).
    
    \item \textbf{AuditLog.UsuarioNombre:} Se desnormaliza el nombre del usuario para consultas rápidas de auditoría sin JOIN.
    
    \item \textbf{Consultas.IMC:} Columna calculada persistida para evitar recalcular el Índice de Masa Corporal en cada consulta.
\end{itemize}

\vspace{1cm}

\begin{center}
\fbox{\parbox{0.9\textwidth}{
\textbf{RESUMEN:} Las 18 tablas del Sistema SIGHC están completamente normalizadas hasta 3NF, garantizando integridad referencial, eliminación de redundancia y optimización del espacio de almacenamiento. Las 24 relaciones entre entidades están correctamente definidas con Foreign Keys, índices y constraints que aseguran la consistencia de los datos en todo momento.
}}
\end{center}


% =====================================================
% 7. DICCIONARIO DE DATOS COMPLETO
% =====================================================
% =====================================================
% 7. ARQUITECTURA COMPLETA DE BASE DE DATOS
% =====================================================
\newpage
\section{Arquitectura Completa de Base de Datos}


El SIGHC utiliza una arquitectura de base de datos relacional implementada en \textbf{Microsoft SQL Server 2019 Enterprise}, con las siguientes características:

\begin{itemize}[leftmargin=1.5em]
    \item \textbf{Normalización:} Todas las tablas están normalizadas hasta Tercera Forma Normal (3NF)
    \item \textbf{Integridad Referencial:} Uso extensivo de Foreign Keys con CASCADE DELETE/UPDATE según corresponda
    \item \textbf{Índices:} 38 índices compuestos y simples para optimización de consultas
    \item \textbf{Triggers:} 6 triggers de auditoría automática en tablas críticas
    \item \textbf{Procedimientos Almacenados:} 12 SP para operaciones complejas
    \item \textbf{Funciones:} 8 funciones escalares y con valores de tabla
    \item \textbf{Vistas:} 10 vistas para consultas frecuentes y reportes
    \item \textbf{Seguridad:} Cifrado TDE activado para datos en reposo
\end{itemize}



\subsection{Descripción de Tablas del Sistema}

El sistema SIGHC está compuesto por \textbf{18 tablas relacionales} organizadas en 5 categorías funcionales, todas normalizadas hasta la Tercera Forma Normal (3NF) para garantizar integridad, eliminar redundancia y optimizar el rendimiento.

\begin{longtable}{|c|p{4cm}|p{4cm}|p{5cm}|}
\hline
\tablaheader{No.} & \tablaheader{Tabla} & \tablaheader{Categoría} & \tablaheader{Descripción Funcional} \\
\hline
\endfirsthead
\hline
\tablaheader{No.} & \tablaheader{Tabla} & \tablaheader{Categoría} & \tablaheader{Descripción Funcional} \\
\hline
\endhead

1 & Pacientes & Entidad Principal & Almacena datos demográficos y clínicos de todos los pacientes del hospital con historia clínica única. \\
\hline
2 & Medicos & Entidad Principal & Registro del personal médico con CMP, RNE, especialidad y horarios de atención. \\
\hline
3 & Especialidades & Catálogo & Catálogo de especialidades médicas (Cardiología, Pediatría, Traumatología, etc.). \\
\hline
4 & Citas & Transaccional & Agenda de citas médicas con validación de disponibilidad y control de estados. \\
\hline
5 & Consultas & Transaccional & Registro de consultas médicas: anamnesis, signos vitales, examen físico, evolución clínica. \\
\hline
6 & Diagnosticos & Transaccional & Diagnósticos médicos con clasificación CIE-10, tipo (presuntivo/definitivo) y clasificación (principal/secundario). \\
\hline
7 & CIE10 & Catálogo & Catálogo completo de la Clasificación Internacional de Enfermedades versión 10 de la OMS (14,000+ códigos). \\
\hline
8 & Tratamientos & Transaccional & Prescripciones médicas: medicamentos, dosis, frecuencia, vía de administración, duración. \\
\hline
9 & Medicamentos & Catálogo & Catálogo de medicamentos: nombre genérico, comercial, presentación, stock, precio. \\
\hline
10 & Inventario Movimientos & Transaccional & Movimientos de entrada y salida de medicamentos del almacén farmacéutico. \\
\hline
11 & Lotes & Complementaria & Control de lotes de medicamentos con fecha de vencimiento para trazabilidad DIGEMID. \\
\hline
12 & HorariosAtencion & Complementaria & Horarios de atención configurados por médico: días, horas, consultorio, turno. \\
\hline
13 & Usuarios & Seguridad & Usuarios del sistema con credenciales cifradas (hash bcrypt + salt único). \\
\hline
14 & Roles & Seguridad & Roles RBAC: Administrador, Médico, Enfermera, Recepcionista, Farmacia, Auditor. \\
\hline
15 & Permisos & Seguridad & Permisos granulares por rol: CREATE, READ, UPDATE, DELETE a nivel de tabla/módulo. \\
\hline
16 & AuditLog & Auditoría & Log inmutable de todas las operaciones críticas con valores anteriores y nuevos en formato JSON. \\
\hline
17 & Auditoria\_Pacientes & Auditoría & Auditoría específica de cambios en historias clínicas de pacientes (histórico completo). \\
\hline
18 & Auditoria\_Diagnosticos & Auditoría & Auditoría específica de diagnósticos médicos (dato crítico con implicancias legales). \\
\hline
\caption{Catálogo completo de las 18 tablas del Sistema SIGHC}
\label{tab:catalogo_completo}
\end{longtable}
\subsection{Diseño de Software}
\subsubsection{Principios SOLID}
El desarrollo del sistema SIGHC se basa en una metodología híbrida que combina SCRUM y Rational Unified Process (RUP). SCRUM se emplea para la gestión iterativa de los requerimientos y la priorización de funcionalidades, mientras que RUP aporta un marco estructurado para el análisis, diseño y documentación formal del sistema, garantizando trazabilidad y rigor técnico.

Asimismo, el diseño del sistema se fundamenta en los principios SOLID, destacando principalmente el Principio de Responsabilidad Única (SRP), evidenciado en la separación clara de módulos y entidades, y el Principio de Segregación de Interfaces (ISP), aplicado mediante el modelo de control de accesos basado en roles (RBAC). Estos principios contribuyen a la mantenibilidad, escalabilidad y seguridad del sistema.
\begin{figure}[H]
    \centering
        \caption{Diagrama de actividad 01}

    \includegraphics[width=0.8\linewidth]{DA 01.PNG}
\end{figure}

\textit{Nota.}  Este diagrama representa el proceso de registro y gestión de pacientes, incluyendo la validación del DNI, la verificación de duplicidad y la generación automática del número único de historia clínica.

\begin{figure}[H]
    \centering
        \caption{Diagrama de actividad 02}

    \includegraphics[width=0.8\linewidth]{DA 02.PNG}
\end{figure}

\textit{Nota.}  El diagrama describe el flujo para la programación de citas médicas, considerando la selección del paciente, la disponibilidad del médico y la notificación al paciente en caso de confirmación.

\begin{figure}[H]
    \centering
        \caption{Diagrama de actividad 03}

    \includegraphics[width=0.5\linewidth]{DA 03.PNG}
\end{figure}

\textit{Nota.}  Este diagrama muestra el proceso de atención médica, desde la confirmación de la cita hasta el registro de la consulta y el diagnóstico utilizando la clasificación CIE-10.

\begin{figure}[H]
    \centering
        \caption{Diagrama de actividad 04}

    \includegraphics[width=0.5\linewidth]{DA 04.PNG}
\end{figure}

\textit{Nota.}  El diagrama detalla el flujo de prescripción de tratamientos médicos, incorporando validaciones por alergias e interacciones y la generación de la receta médica electrónica.

\begin{figure}[H]
    \centering
        \caption{Diagrama de actividad 05}

    \includegraphics[width=0.8\linewidth]{DA 05.PNG}
\end{figure}

\textit{Nota.}  Este diagrama representa la gestión del personal médico, incluyendo el registro del profesional, la validación de credenciales y la asignación de especialidades y horarios de atención.

\begin{figure}[H]
    \centering
        \caption{Diagrama de actividad 06}

    \includegraphics[width=0.5\linewidth]{DA 06.PNG}
\end{figure}

\textit{Nota.}  El diagrama describe el control del inventario de medicamentos, desde el registro y actualización de stock hasta la generación de alertas por niveles mínimos.

\begin{figure}[H]
    \centering
        \caption{Diagrama de actividad 07}

    \includegraphics[width=0.5\linewidth]{DA 07.PNG}
\end{figure}

\textit{Nota.}  Este diagrama muestra el proceso de autenticación de usuarios, la asignación de roles y permisos, y el control de accesos al sistema conforme al modelo RBAC.

\begin{figure}[H]
    \centering
        \caption{Diagrama de actividad 08}

    \includegraphics[width=0.5\linewidth]{DA 08.PNG}
\end{figure}

\textit{Nota.}  El diagrama representa el registro automático de las acciones realizadas en el sistema, garantizando la trazabilidad, integridad e inmutabilidad de la información auditada.

\begin{figure}[H]
    \centering
        \caption{Diagrama de actividad 09}
    \includegraphics[width=0.5\linewidth]{DA 09.PNG}
\end{figure}
\textit{Nota.}  Este diagrama describe el flujo de generación de reportes y visualización de indicadores, permitiendo el análisis de información clínica y administrativa para la toma de decisiones.

\begin{figure}[H]
    \centering
        \caption{Diagrama de actividad 10}

    \includegraphics[width=0.5\linewidth]{DA 10.PNG}
\end{figure}

\textit{Nota.}  El diagrama muestra el proceso de respaldo y recuperación de la información, asegurando la continuidad operativa y la protección de los datos ante fallos del sistema.

\begin{figure}[H]
    \centering
        \caption{Diagrama de componentes}

    \includegraphics[width=0.8\linewidth]{DC General.png}
\end{figure}

\textit{Nota.}  El diagrama de componentes muestra la estructura lógica del Sistema Integral de Gestión de Historias Clínicas (SIGHC), representando la interacción entre los usuarios, la interfaz de usuario, los módulos funcionales del sistema y la base de datos central. Cada componente corresponde a un módulo del sistema y refleja sus dependencias internas, así como el acceso controlado a la información y los mecanismos de auditoría y respaldo.

% ============================================
% SECCIÓN PRINCIPAL (Nivel 1 APA 7)
% ============================================
\newpage
\section{Modelo de Datos}

% ------------------------
% Subsection (Nivel 2 APA 7)
% ------------------------
\subsection{Modelo lógico}

\begin{figure}[H]
    \centering
    \includegraphics[width=\linewidth]{EGBD.jpg}
    \caption{\textit{Modelo lógico de la base de datos.}}
    \label{fig:modelo-logico}
\end{figure}

% ------------------------
% Subsection (Nivel 2 APA 7)
% ------------------------
\subsection{Diagrama –relación (ERD)}

\begin{figure}[H]
    \centering
    \includegraphics[width=\linewidth]{Diagramas modelado.drawio.png}
    \caption{\textit{Diagrama entidad–relación (ERD) del sistema.}}
    \label{fig:erd}
\end{figure}

\begin{figure}
    \centering
    \includegraphics[width=\linewidth]{image.png}
    \caption{Diagrama Físico}
    \label{fig:placeholder}
\end{figure}
% =====================================================
% 7.3 DICCIONARIO COMPLETO DE LAS 18 TABLAS
% =====================================================
\newpage
\subsection{Diccionario de Datos Completo de las 18 Tablas}

\subsubsection{TABLA 01: Pacientes}

\textbf{Descripción:} Entidad principal que almacena la información demográfica, clínica y de contacto de todos los pacientes del hospital. Genera automáticamente el número de historia clínica único e inmutable.

\textbf{Primary Key:} IdPaciente (INT IDENTITY)\\
\textbf{Unique Keys:} NroHistoriaClinica, DNI\\
\textbf{Foreign Keys:} UsuarioRegistro → Usuarios(IdUsuario)

\begin{longtable}{|p{4.2cm}|p{3.7cm}|p{1cm}|p{5cm}|}
\hline
\tablaheader{Campo} & \tablaheader{Tipo} & \tablaheader{Nulo} & \tablaheader{Descripción / Restricciones} \\
\hline
\endfirsthead
\hline
\tablaheader{Campo} & \tablaheader{Tipo} & \tablaheader{Nulo} & \tablaheader{Descripción} \\
\hline
\endhead

IdPaciente & INT & NO & PK autoincremental. Identificador único del paciente. \\
\hline
NroHistoriaClinica & VARCHAR(15) & NO & Número de historia clínica único formato HC-YYYY-NNNNN. UNIQUE constraint. \\
\hline
Nombres & NVARCHAR(80) & NO & Nombres completos del paciente. Índice para búsquedas. \\
\hline
Apellidos & NVARCHAR(80) & NO & Apellidos completos. Índice compuesto con Nombres. \\
\hline
DNI & VARCHAR(8) & NO & Documento Nacional de Identidad. UNIQUE constraint. \\
\hline
FechaNacimiento & DATE & NO & Fecha de nacimiento. Para calcular edad con FN\_CalcularEdad(). \\
\hline
Sexo & CHAR(1) & NO & M=Masculino, F=Femenino. CHECK (Sexo IN ('M','F')). \\
\hline
GrupoSanguineo & VARCHAR(5) & SÍ & A+, A-, B+, B-, O+, O-, AB+, AB-. \\
\hline
Direccion & NVARCHAR(200) & SÍ & Dirección completa de domicilio. \\
\hline
Telefono & VARCHAR(15) & SÍ & Teléfono de contacto principal. \\
\hline
Email & VARCHAR(100) & SÍ & Correo electrónico para notificaciones. \\
\hline
AntecedentesFamiliares & NVARCHAR(MAX) & SÍ & Antecedentes médicos familiares (texto libre). \\
\hline
AntecedentesPersonales & NVARCHAR(MAX) & SÍ & Antecedentes patológicos personales. \\
\hline
Alergias & NVARCHAR(MAX) & SÍ & Alergias conocidas (CRÍTICO para alertas medicamentosas). \\
\hline
FechaRegistro & DATETIME2 & NO & Timestamp de creación. DEFAULT SYSDATETIME(). \\
\hline
Estado & CHAR(1) & NO & A=Activo, I=Inactivo, F=Fallecido. DEFAULT 'A'. \\
\hline
UsuarioRegistro & INT & NO & FK hacia Usuarios. Quién registró al paciente. \\
\hline
\caption{Tabla 01: Pacientes - Estructura completa}
\end{longtable}

\textbf{Índices:}
\begin{itemize}[leftmargin=1.5em]
    \item \texttt{IX\_Pacientes\_DNI} (NONCLUSTERED) sobre DNI
    \item \texttt{IX\_Pacientes\_Nombres} (NONCLUSTERED) sobre (Nombres, Apellidos)
    \item \texttt{IX\_Pacientes\_Estado} (FILTERED) sobre Estado WHERE Estado='A'
\end{itemize}

\textbf{Relaciones:}
\begin{itemize}[leftmargin=1.5em]
    \item 1:N con Citas (Un paciente puede tener múltiples citas)
    \item 1:N con Consultas (Un paciente puede tener múltiples consultas)
    \item N:1 con Usuarios (Múltiples pacientes registrados por un usuario)
\end{itemize}

\subsubsection{TABLA 02: Medicos}

\textbf{Descripción:} Almacena información del personal médico del hospital con sus credenciales profesionales CMP (Colegio Médico del Perú) y RNE (Registro Nacional de Especialistas).

\textbf{Primary Key:} IdMedico (INT IDENTITY)\\
\textbf{Unique Keys:} DNI, CMP\\
\textbf{Foreign Keys:} IdEspecialidad → Especialidades(IdEspecialidad)

\begin{longtable}{|p{4.2cm}|p{3.5cm}|p{1cm}|p{7cm}|}
\hline
\tablaheader{Campo} & \tablaheader{Tipo} & \tablaheader{Nulo} & \tablaheader{Descripción} \\
\hline
\endfirsthead
\hline
\tablaheader{Campo} & \tablaheader{Tipo} & \tablaheader{Nulo} & \tablaheader{Descripción} \\
\hline
\endhead

IdMedico & INT & NO & PK autoincremental. \\
\hline
Nombres & NVARCHAR(80) & NO & Nombres completos del médico. \\
\hline
Apellidos & NVARCHAR(80) & NO & Apellidos completos. \\
\hline
DNI & VARCHAR(8) & NO & DNI único del médico. \\
\hline
CMP & VARCHAR(10) & NO & Código del Colegio Médico del Perú (obligatorio, único). \\
\hline
RNE & VARCHAR(10) & SÍ & Registro Nacional de Especialistas (si tiene especialidad). \\
\hline
IdEspecialidad & INT & NO & FK hacia Especialidades. Especialidad médica principal. \\
\hline
Telefono & VARCHAR(15) & SÍ & Teléfono de contacto. \\
\hline
Email & VARCHAR(100) & NO & Email institucional para notificaciones. \\
\hline
FechaIngreso & DATE & NO & Fecha de inicio de labores en el hospital. \\
\hline
Estado & CHAR(1) & NO & A=Activo, I=Inactivo, S=Suspendido, R=Retirado. DEFAULT 'A'. \\
\hline
\caption{Tabla 02: Medicos - Estructura completa}
\end{longtable}

\textbf{Relaciones:}
\begin{itemize}[leftmargin=1.5em]
    \item N:1 con Especialidades (Muchos médicos, una especialidad)
    \item 1:N con Citas (Un médico atiende múltiples citas)
    \item 1:N con Consultas (Un médico realiza múltiples consultas)
    \item 1:N con HorariosAtencion (Un médico tiene múltiples horarios)
\end{itemize}

\subsubsection{TABLA 03: Especialidades}

\textbf{Descripción:} Catálogo de especialidades médicas disponibles en el hospital.

\textbf{Primary Key:} IdEspecialidad (INT IDENTITY)\\
\textbf{Unique Keys:} NombreEspecialidad

\begin{longtable}{|p{4.2cm}|p{3.5cm}|p{1cm}|p{5cm}|}
\hline
\tablaheader{Campo} & \tablaheader{Tipo} & \tablaheader{Nulo} & \tablaheader{Descripción} \\
\hline
\endfirsthead
\hline
\tablaheader{Campo} & \tablaheader{Tipo} & \tablaheader{Nulo} & \tablaheader{Descripción} \\
\hline
\endhead

IdEspecialidad & INT & NO & PK autoincremental. \\
\hline
NombreEspecialidad & NVARCHAR(100) & NO & Nombre de la especialidad (ej: Cardiología, Pediatría). UNIQUE. \\
\hline
Descripcion & NVARCHAR(500) & SÍ & Descripción detallada de la especialidad. \\
\hline
Estado & CHAR(1) & NO & A=Activo, I=Inactivo. DEFAULT 'A'. \\
\hline
\caption{Tabla 03: Especialidades - Catálogo}
\end{longtable}

\textbf{Datos iniciales:} Medicina General, Pediatría, Cardiología, Ginecología, Traumatología, Neurología, Oftalmología, Dermatología.

\textbf{Relaciones:}
\begin{itemize}[leftmargin=1.5em]
    \item 1:N con Medicos (Una especialidad, múltiples médicos)
\end{itemize}

\subsubsection{TABLA 04: Citas}

\textbf{Descripción:} Tabla transaccional que almacena las citas médicas programadas con validación de disponibilidad horaria.

\textbf{Primary Key:} IdCita (INT IDENTITY)\\
\textbf{Unique Keys:} CodigoCita\\
\textbf{Foreign Keys:} IdPaciente → Pacientes, IdMedico → Medicos, UsuarioRegistro → Usuarios

\begin{longtable}{|p{4.2cm}|p{3.5cm}|p{1cm}|p{5cm}|}
\hline
\tablaheader{Campo} & \tablaheader{Tipo} & \tablaheader{Nulo} & \tablaheader{Descripción} \\
\hline
\endfirsthead
\hline
\tablaheader{Campo} & \tablaheader{Tipo} & \tablaheader{Nulo} & \tablaheader{Descripción} \\
\hline
\endhead

IdCita & INT & NO & PK autoincremental. \\
\hline
CodigoCita & VARCHAR(20) & NO & Código único CITA-YYYY-NNNNNN. UNIQUE. \\
\hline
IdPaciente & INT & NO & FK hacia Pacientes. Paciente que recibe la atención. \\
\hline
IdMedico & INT & NO & FK hacia Medicos. Médico que atenderá. \\
\hline
FechaCita & DATE & NO & Fecha de la cita. Índice compuesto. \\
\hline
HoraInicio & TIME & NO & Hora de inicio formato HH:MM:SS. \\
\hline
HoraFin & TIME & NO & Hora de fin (generalmente inicio + 30 minutos). \\
\hline
MotivoConsulta & NVARCHAR(500) & NO & Motivo de la consulta. \\
\hline
TipoCita & VARCHAR(20) & NO & PrimeraVez, Control, Emergencia. CHECK constraint. \\
\hline
Estado & VARCHAR(20) & NO & Programada, Confirmada, Atendida, Cancelada, Reprogramada. DEFAULT 'Programada'. \\
\hline
MotivoCancelacion & NVARCHAR(200) & SÍ & Obligatorio si Estado='Cancelada'. \\
\hline
FechaRegistro & DATETIME2 & NO & Timestamp del registro. DEFAULT SYSDATETIME(). \\
\hline
UsuarioRegistro & INT & NO & FK hacia Usuarios. Quién programó la cita. \\
\hline
\caption{Tabla 04: Citas - Estructura completa}
\end{longtable}

\textbf{Índices:}
\begin{itemize}[leftmargin=1.5em]
    \item \texttt{IX\_Citas\_Paciente\_Fecha} sobre (IdPaciente, FechaCita DESC)
    \item \texttt{IX\_Citas\_Medico\_Fecha} sobre (IdMedico, FechaCita, HoraInicio)
    \item \texttt{IX\_Citas\_Estado} sobre (Estado, FechaCita)
\end{itemize}

\textbf{Relaciones:}
\begin{itemize}[leftmargin=1.5em]
    \item N:1 con Pacientes (Muchas citas de un paciente)
    \item N:1 con Medicos (Muchas citas atendidas por un médico)
    \item 1:1 con Consultas (Una cita genera una consulta)
\end{itemize}

\subsubsection{TABLA 05: Consultas}

\textbf{Descripción:} Registro completo de consultas médicas: anamnesis, signos vitales, examen físico y evolución clínica.

\textbf{Primary Key:} IdConsulta (INT IDENTITY)\\
\textbf{Unique Keys:} IdCita (relación 1:1)\\
\textbf{Foreign Keys:} IdCita → Citas, IdPaciente → Pacientes, IdMedico → Medicos

\begin{longtable}{|p{4.2cm}|p{3.5cm}|p{1cm}|p{5cm}|}
\hline
\tablaheader{Campo} & \tablaheader{Tipo} & \tablaheader{Nulo} & \tablaheader{Descripción} \\
\hline
\endfirsthead
\hline
\tablaheader{Campo} & \tablaheader{Tipo} & \tablaheader{Nulo} & \tablaheader{Descripción} \\
\hline
\endhead

IdConsulta & INT & NO & PK autoincremental. \\
\hline
IdCita & INT & NO & FK hacia Citas. UNIQUE (relación 1:1). \\
\hline
IdPaciente & INT & NO & FK hacia Pacientes (desnormalizado para consultas rápidas). \\
\hline
IdMedico & INT & NO & FK hacia Medicos (desnormalizado). \\
\hline
FechaConsulta & DATETIME2 & NO & Fecha y hora exacta de la consulta. DEFAULT SYSDATETIME(). \\
\hline
\multicolumn{4}{|c|}{\cellcolor{tablehead}\textbf{SIGNOS VITALES}} \\
\hline
PresionArterial & VARCHAR(10) & SÍ & Formato 120/80 (sistólica/diastólica mmHg). \\
\hline
Temperatura & DECIMAL(4,1) & SÍ & Temperatura corporal en °C (ej: 36.5). \\
\hline
FrecuenciaCardiaca & INT & SÍ & Frecuencia cardíaca en latidos por minuto. \\
\hline
FrecuenciaRespiratoria & INT & SÍ & Respiraciones por minuto. \\
\hline
Peso & DECIMAL(5,2) & SÍ & Peso en kilogramos (ej: 70.50). \\
\hline
Talla & DECIMAL(5,2) & SÍ & Talla en metros (ej: 1.70). \\
\hline
IMC & DECIMAL(5,2) & SÍ & Índice de Masa Corporal (columna calculada: Peso/(Talla*Talla)). \\
\hline
SaturacionO2 & INT & SÍ & Saturación de oxígeno en porcentaje (ej: 98). \\
\hline
\multicolumn{4}{|c|}{\cellcolor{tablehead}\textbf{ANAMNESIS Y EXAMEN}} \\
\hline
MotivoConsulta & NVARCHAR(MAX) & NO & Motivo principal de consulta. \\
\hline
TiempoEnfermedad & NVARCHAR(200) & SÍ & Duración de la enfermedad actual. \\
\hline
RelatoCronico & NVARCHAR(MAX) & SÍ & Historia cronológica detallada de la enfermedad. \\
\hline
ExamenFisico & NVARCHAR(MAX) & SÍ & Hallazgos del examen físico completo. \\
\hline
PlanTrabajo & NVARCHAR(MAX) & SÍ & Plan de trabajo médico y seguimiento. \\
\hline
\caption{Tabla 05: Consultas - Estructura completa}
\end{longtable}

\textbf{Relaciones:}
\begin{itemize}[leftmargin=1.5em]
    \item 1:1 con Citas (Una consulta por cita)
    \item N:1 con Pacientes (Muchas consultas de un paciente)
    \item N:1 con Medicos (Muchas consultas realizadas por un médico)
    \item 1:N con Diagnosticos (Una consulta puede tener múltiples diagnósticos)
\end{itemize}

\subsubsection{TABLA 06: Diagnosticos}

\textbf{Descripción:} Diagnósticos médicos con clasificación CIE-10 de la OMS.

\textbf{Primary Key:} IdDiagnostico (INT IDENTITY)\\
\textbf{Foreign Keys:} IdConsulta → Consultas, CodigoCIE10 → CIE10

\begin{longtable}{|p{4.2cm}|p{3.5cm}|p{1cm}|p{5cm}|}
\hline
\tablaheader{Campo} & \tablaheader{Tipo} & \tablaheader{Nulo} & \tablaheader{Descripción} \\
\hline
\endfirsthead
\hline
\tablaheader{Campo} & \tablaheader{Tipo} & \tablaheader{Nulo} & \tablaheader{Descripción} \\
\hline
\endhead

IdDiagnostico & INT & NO & PK autoincremental. \\
\hline
IdConsulta & INT & NO & FK hacia Consultas. \\
\hline
CodigoCIE10 & VARCHAR(10) & NO & FK hacia CIE10. Código CIE-10 (ej: J06.9). \\
\hline
DescripcionDiagnostico & NVARCHAR(500) & NO & Descripción detallada del diagnóstico médico. \\
\hline
TipoDiagnostico & VARCHAR(20) & NO & Presuntivo o Definitivo. CHECK constraint. \\
\hline
Clasificacion & VARCHAR(20) & NO & Principal, Secundario, Complicacion. CHECK constraint. \\
\hline
FechaRegistro & DATETIME2 & NO & Timestamp del diagnóstico. DEFAULT SYSDATETIME(). \\
\hline
\caption{Tabla 06: Diagnosticos - Estructura completa}
\end{longtable}

\textbf{Relaciones:}
\begin{itemize}[leftmargin=1.5em]
    \item N:1 con Consultas (Múltiples diagnósticos por consulta)
    \item N:1 con CIE10 (Múltiples diagnósticos usan un código CIE-10)
    \item 1:N con Tratamientos (Un diagnóstico puede tener múltiples tratamientos)
\end{itemize}

\subsubsection{TABLA 07: CIE10}

\textbf{Descripción:} Catálogo completo de la Clasificación Internacional de Enfermedades versión 10 de la OMS (aproximadamente 14,000 códigos).

\textbf{Primary Key:} CodigoCIE10 (VARCHAR)

\begin{longtable}{|p{4.2cm}|p{3.5cm}|p{1cm}|p{5cm}|}
\hline
\tablaheader{Campo} & \tablaheader{Tipo} & \tablaheader{Nulo} & \tablaheader{Descripción} \\
\hline
\endfirsthead
\hline
\tablaheader{Campo} & \tablaheader{Tipo} & \tablaheader{Nulo} & \tablaheader{Descripción} \\
\hline
\endhead

CodigoCIE10 & VARCHAR(10) & NO & PK. Código CIE-10 (ej: A00.0, J06.9, I10). \\
\hline
Descripcion & NVARCHAR(500) & NO & Descripción completa de la enfermedad. Índice FULLTEXT. \\
\hline
Capitulo & VARCHAR(10) & NO & Capítulo CIE-10 (I a XXII). \\
\hline
DescripcionCapitulo & NVARCHAR(200) & NO & Descripción del capítulo. \\
\hline
Sexo & CHAR(1) & SÍ & M/F si aplica solo a un sexo, NULL si aplica a ambos. \\
\hline
EdadMinima & INT & SÍ & Edad mínima aplicable (NULL sin restricción). \\
\hline
EdadMaxima & INT & SÍ & Edad máxima aplicable (NULL sin restricción). \\
\hline
NotificacionObligatoria & BIT & NO & 1=Enfermedad de notificación obligatoria MINSA, 0=No. \\
\hline
Estado & CHAR(1) & NO & A=Activo, I=Inactivo (obsoleto). DEFAULT 'A'. \\
\hline
\caption{Tabla 07: CIE10 - Catálogo OMS}
\end{longtable}

\textbf{Índices especiales:}
\begin{itemize}[leftmargin=1.5em]
    \item FULLTEXT INDEX sobre Descripcion para búsquedas rápidas por palabra clave
\end{itemize}

\textbf{Relaciones:}
\begin{itemize}[leftmargin=1.5em]
    \item 1:N con Diagnosticos (Un código CIE-10 puede ser usado en múltiples diagnósticos)
\end{itemize}

\subsubsection{TABLA 08: Tratamientos}

\textbf{Descripción:} Prescripciones médicas farmacológicas y no farmacológicas.

\textbf{Primary Key:} IdTratamiento (INT IDENTITY)\\
\textbf{Foreign Keys:} IdDiagnostico → Diagnosticos, IdMedicamento → Medicamentos

\begin{longtable}{|p{4.2cm}|p{3.5cm}|p{1cm}|p{5cm}|}
\hline
\tablaheader{Campo} & \tablaheader{Tipo} & \tablaheader{Nulo} & \tablaheader{Descripción} \\
\hline
\endfirsthead
\hline
\tablaheader{Campo} & \tablaheader{Tipo} & \tablaheader{Nulo} & \tablaheader{Descripción} \\
\hline
\endhead

IdTratamiento & INT & NO & PK autoincremental. \\
\hline
IdDiagnostico & INT & NO & FK hacia Diagnosticos. Diagnóstico que motiva el tratamiento. \\
\hline
IdMedicamento & INT & NO & FK hacia Medicamentos. Medicamento prescrito. \\
\hline
Dosis & NVARCHAR(100) & NO & Dosis prescrita (ej: 500mg, 1 tableta, 10ml). \\
\hline
Frecuencia & NVARCHAR(100) & NO & Frecuencia de administración (ej: cada 8 horas, 3 veces al día). \\
\hline
ViaAdministracion & VARCHAR(50) & NO & Oral, Endovenosa, Intramuscular, Topica, Sublingual, Rectal. CHECK. \\
\hline
Duracion & INT & NO & Duración del tratamiento en días. \\
\hline
IndicacionesEspeciales & NVARCHAR(500) & SÍ & Indicaciones adicionales del médico. \\
\hline
FechaInicio & DATE & NO & Fecha de inicio del tratamiento. \\
\hline
FechaFin & DATE & NO & Fecha de finalización (columna calculada: FechaInicio + Duracion). \\
\hline
\caption{Tabla 08: Tratamientos - Estructura completa}
\end{longtable}

\textbf{Relaciones:}
\begin{itemize}[leftmargin=1.5em]
    \item N:1 con Diagnosticos (Múltiples tratamientos por diagnóstico)
    \item N:1 con Medicamentos (Múltiples tratamientos usan un medicamento)
\end{itemize}

\subsubsection{TABLA 09: Medicamentos}

\textbf{Descripción:} Catálogo de medicamentos del petitorio farmacéutico del hospital.

\textbf{Primary Key:} IdMedicamento (INT IDENTITY)\\
\textbf{Unique Keys:} CodigoMedicamento

\begin{longtable}{|p{4.2cm}|p{3.5cm}|p{1cm}|p{5cm}|}
\hline
\tablaheader{Campo} & \tablaheader{Tipo} & \tablaheader{Nulo} & \tablaheader{Descripción} \\
\hline
\endfirsthead
\hline
\tablaheader{Campo} & \tablaheader{Tipo} & \tablaheader{Nulo} & \tablaheader{Descripción} \\
\hline
\endhead

IdMedicamento & INT & NO & PK autoincremental. \\
\hline
CodigoMedicamento & VARCHAR(20) & NO & Código interno único del medicamento. UNIQUE. \\
\hline
NombreGenerico & NVARCHAR(200) & NO & Nombre genérico DCI (ej: Paracetamol, Ibuprofeno). \\
\hline
NombreComercial & NVARCHAR(200) & SÍ & Nombre comercial/marca (ej: Tylenol, Advil). \\
\hline
Presentacion & NVARCHAR(100) & NO & Presentación (ej: Tabletas, Jarabe, Ampolla). \\
\hline
Concentracion & NVARCHAR(50) & SÍ & Concentración (ej: 500mg, 20mg/ml). \\
\hline
FormaFarmaceutica & VARCHAR(50) & NO & Sólido, Líquido, Semisólido, Gas. \\
\hline
UnidadMedida & VARCHAR(20) & NO & mg, ml, g, UI, etc. \\
\hline
StockMinimo & INT & NO & Stock mínimo para generar alerta. DEFAULT 10. \\
\hline
StockActual & INT & NO & Stock actual en almacén. DEFAULT 0. \\
\hline
PrecioUnitario & DECIMAL(10,2) & SÍ & Precio unitario en soles. \\
\hline
RequiereReceta & BIT & NO & 1=Requiere receta médica, 0=Venta libre. DEFAULT 1. \\
\hline
Estado & CHAR(1) & NO & A=Activo, I=Inactivo (descontinuado). DEFAULT 'A'. \\
\hline
\caption{Tabla 09: Medicamentos - Catálogo}
\end{longtable}

\textbf{Índices:}
\begin{itemize}[leftmargin=1.5em]
    \item \texttt{IX\_Medicamentos\_Nombre} sobre NombreGenerico
    \item \texttt{IX\_Medicamentos\_Stock} (FILTERED) WHERE StockActual < StockMinimo
\end{itemize}

\textbf{Relaciones:}
\begin{itemize}[leftmargin=1.5em]
    \item 1:N con Tratamientos (Un medicamento usado en múltiples tratamientos)
    \item 1:N con Inventario Movimientos (Un medicamento tiene múltiples movimientos)
    \item 1:N con Lotes (Un medicamento puede tener múltiples lotes)
\end{itemize}

\subsubsection{TABLA 10: Inventario Movimientos}

\textbf{Descripción:} Registro de movimientos de entrada y salida de medicamentos del almacén.

\textbf{Primary Key:} IdMovimiento (INT IDENTITY)\\
\textbf{Foreign Keys:} IdMedicamento → Medicamentos, IdLote → Lotes, UsuarioRegistro → Usuarios

\begin{longtable}{|p{4.2cm}|p{3.5cm}|p{1cm}|p{5cm}|}
\hline
\tablaheader{Campo} & \tablaheader{Tipo} & \tablaheader{Nulo} & \tablaheader{Descripción} \\
\hline
\endfirsthead
\hline
\tablaheader{Campo} & \tablaheader{Tipo} & \tablaheader{Nulo} & \tablaheader{Descripción} \\
\hline
\endhead

IdMovimiento & INT & NO & PK autoincremental. \\
\hline
IdMedicamento & INT & NO & FK hacia Medicamentos. \\
\hline
IdLote & INT & NO & FK hacia Lotes. Lote del medicamento. \\
\hline
TipoMovimiento & VARCHAR(20) & NO & Entrada, Salida, Ajuste, Devolucion, Merma. CHECK. \\
\hline
Cantidad & INT & NO & Cantidad de unidades (positivo=entrada, negativo=salida). \\
\hline
FechaMovimiento & DATETIME2 & NO & Fecha y hora del movimiento. DEFAULT SYSDATETIME(). \\
\hline
Motivo & NVARCHAR(500) & NO & Motivo del movimiento (ej: Compra, Prescripción médica, Vencimiento). \\
\hline
DocumentoReferencia & VARCHAR(50) & SÍ & Número de documento de respaldo (factura, orden, etc.). \\
\hline
UsuarioRegistro & INT & NO & FK hacia Usuarios. Quién realizó el movimiento. \\
\hline
\caption{Tabla 10: Inventario Movimientos - Transaccional}
\end{longtable}

\textbf{Trigger:} Actualiza automáticamente StockActual en tabla Medicamentos.

\textbf{Relaciones:}
\begin{itemize}[leftmargin=1.5em]
    \item N:1 con Medicamentos
    \item N:1 con Lotes
    \item N:1 con Usuarios
\end{itemize}

\subsubsection{TABLA 11: Lotes}

\textbf{Descripción:} Control de lotes de medicamentos con fecha de vencimiento para cumplir normativa DIGEMID.

\textbf{Primary Key:} IdLote (INT IDENTITY)\\
\textbf{Foreign Keys:} IdMedicamento → Medicamentos

\begin{longtable}{|p{4.2cm}|p{3.5cm}|p{1cm}|p{7cm}|}
\hline
\tablaheader{Campo} & \tablaheader{Tipo} & \tablaheader{Nulo} & \tablaheader{Descripción} \\
\hline
\endfirsthead
\hline
\tablaheader{Campo} & \tablaheader{Tipo} & \tablaheader{Nulo} & \tablaheader{Descripción} \\
\hline
\endhead

IdLote & INT & NO & PK autoincremental. \\
\hline
IdMedicamento & INT & NO & FK hacia Medicamentos. \\
\hline
NumeroLote & VARCHAR(50) & NO & Número de lote del fabricante. \\
\hline
FechaFabricacion & DATE & NO & Fecha de fabricación. \\
\hline
FechaVencimiento & DATE & NO & Fecha de vencimiento. \\
\hline
CantidadInicial & INT & NO & Cantidad inicial del lote. \\
\hline
CantidadActual & INT & NO & Cantidad actual disponible. \\
\hline
Proveedor & NVARCHAR(200) & SÍ & Nombre del proveedor/laboratorio. \\
\hline
Estado & VARCHAR(20) & NO & Disponible, Vencido, Agotado. DEFAULT 'Disponible'. \\
\hline
\caption{Tabla 11: Lotes - Complementaria}
\end{longtable}

\textbf{Índice:}
\begin{itemize}[leftmargin=1.5em]
    \item \texttt{IX\_Lotes\_Vencimiento} sobre FechaVencimiento para alertas
\end{itemize}

\textbf{Relaciones:}
\begin{itemize}[leftmargin=1.5em]
    \item N:1 con Medicamentos
    \item 1:N con Inventario Movimientos
\end{itemize}

\subsubsection{TABLA 12: HorariosAtencion}

\textbf{Descripción:} Horarios de atención configurados por médico para validar disponibilidad de citas.

\textbf{Primary Key:} IdHorario (INT IDENTITY)\\
\textbf{Foreign Keys:} IdMedico → Medicos

\begin{longtable}{|p{4.2cm}|p{3.5cm}|p{1cm}|p{5cm}|}
\hline
\tablaheader{Campo} & \tablaheader{Tipo} & \tablaheader{Nulo} & \tablaheader{Descripción} \\
\hline
\endfirsthead
\hline
\tablaheader{Campo} & \tablaheader{Tipo} & \tablaheader{Nulo} & \tablaheader{Descripción} \\
\hline
\endhead

IdHorario & INT & NO & PK autoincremental. \\
\hline
IdMedico & INT & NO & FK hacia Medicos. \\
\hline
DiaSemana & INT & NO & 1=Lunes, 2=Martes, ..., 7=Domingo. CHECK (1-7). \\
\hline
HoraInicio & TIME & NO & Hora de inicio de atención. \\
\hline
HoraFin & TIME & NO & Hora de finalización de atención. \\
\hline
Consultorio & VARCHAR(50) & SÍ & Número o nombre del consultorio. \\
\hline
Turno & VARCHAR(20) & NO & Mañana, Tarde, Noche. \\
\hline
Estado & CHAR(1) & NO & A=Activo, I=Inactivo. DEFAULT 'A'. \\
\hline
\caption{Tabla 12: HorariosAtencion - Complementaria}
\end{longtable}

\textbf{Relaciones:}
\begin{itemize}[leftmargin=1.5em]
    \item N:1 con Medicos (Un médico tiene múltiples horarios)
\end{itemize}

\subsubsection{TABLA 13: Usuarios}

\textbf{Descripción:} Usuarios del sistema con credenciales de autenticación cifradas.

\textbf{Primary Key:} IdUsuario (INT IDENTITY)\\
\textbf{Unique Keys:} NombreUsuario, Email\\
\textbf{Foreign Keys:} IdRol → Roles

\begin{longtable}{|p{4.2cm}|p{3.5cm}|p{1cm}|p{5cm}|}
\hline
\tablaheader{Campo} & \tablaheader{Tipo} & \tablaheader{Nulo} & \tablaheader{Descripción} \\
\hline
\endfirsthead
\hline
\tablaheader{Campo} & \tablaheader{Tipo} & \tablaheader{Nulo} & \tablaheader{Descripción} \\
\hline
\endhead

IdUsuario & INT & NO & PK autoincremental. \\
\hline
NombreUsuario & VARCHAR(50) & NO & Nombre de usuario único para login. UNIQUE. \\
\hline
PasswordHash & VARBINARY(64) & NO & Hash bcrypt de la contraseña (cost factor 12). \\
\hline
PasswordSalt & VARBINARY(32) & NO & Salt único generado aleatoriamente. \\
\hline
NombreCompleto & NVARCHAR(150) & NO & Nombre completo del usuario. \\
\hline
Email & VARCHAR(100) & NO & Email único. UNIQUE. \\
\hline
IdRol & INT & NO & FK hacia Roles. Rol asignado. \\
\hline
UltimoAcceso & DATETIME2 & SÍ & Fecha y hora del último acceso exitoso. \\
\hline
CambioPasswordObligatorio & BIT & NO & 1=Debe cambiar contraseña en próximo login. DEFAULT 1. \\
\hline
IntentosAccesoFallido & INT & NO & Contador de intentos fallidos. DEFAULT 0. \\
\hline
CuentaBloqueada & BIT & NO & 1=Cuenta bloqueada por seguridad. DEFAULT 0. \\
\hline
FechaCreacion & DATETIME2 & NO & Timestamp de creación. DEFAULT SYSDATETIME(). \\
\hline
Estado & CHAR(1) & NO & A=Activo, I=Inactivo. DEFAULT 'A'. \\
\hline
\caption{Tabla 13: Usuarios - Seguridad}
\end{longtable}

\textbf{Seguridad:} Las contraseñas NUNCA se almacenan en texto plano. Se usa bcrypt con salt único.

\textbf{Relaciones:}
\begin{itemize}[leftmargin=1.5em]
    \item N:1 con Roles (Múltiples usuarios con un rol)
    \item 1:N con Pacientes (Usuario registra múltiples pacientes)
    \item 1:N con Citas (Usuario programa múltiples citas)
    \item 1:N con Inventario Movimientos
\end{itemize}

\subsubsection{TABLA 14: Roles}

\textbf{Descripción:} Roles RBAC del sistema con jerarquía de privilegios.

\textbf{Primary Key:} IdRol (INT IDENTITY)\\
\textbf{Unique Keys:} NombreRol

\begin{longtable}{|p{4.2cm}|p{3.5cm}|p{1cm}|p{5cm}|}
\hline
\tablaheader{Campo} & \tablaheader{Tipo} & \tablaheader{Nulo} & \tablaheader{Descripción} \\
\hline
\endfirsthead
\hline
\tablaheader{Campo} & \tablaheader{Tipo} & \tablaheader{Nulo} & \tablaheader{Descripción} \\
\hline
\endhead

IdRol & INT & NO & PK autoincremental. \\
\hline
NombreRol & VARCHAR(50) & NO & Nombre único del rol. UNIQUE. \\
\hline
Descripcion & NVARCHAR(200) & SÍ & Descripción del rol. \\
\hline
Nivel & INT & NO & Nivel jerárquico (1=mayor privilegio, 6=menor). \\
\hline
Estado & CHAR(1) & NO & A=Activo, I=Inactivo. DEFAULT 'A'. \\
\hline
\caption{Tabla 14: Roles - Seguridad}
\end{longtable}

\textbf{Roles del sistema:}
\begin{itemize}[leftmargin=1.5em]
    \item Administrador (Nivel 1)
    \item Médico (Nivel 2)
    \item Enfermera (Nivel 3)
    \item Recepcionista (Nivel 4)
    \item Farmacia (Nivel 5)
    \item Auditor (Nivel 6)
\end{itemize}

\textbf{Relaciones:}
\begin{itemize}[leftmargin=1.5em]
    \item 1:N con Usuarios
    \item 1:N con Permisos
\end{itemize}

\subsubsection{TABLA 15: Permisos}

\textbf{Descripción:} Permisos granulares CRUD (Create, Read, Update, Delete) por rol y módulo.

\textbf{Primary Key:} IdPermiso (INT IDENTITY)\\
\textbf{Foreign Keys:} IdRol → Roles

\begin{longtable}{|p{4.2cm}|p{3.5cm}|p{1cm}|p{5cm}|}
\hline
\tablaheader{Campo} & \tablaheader{Tipo} & \tablaheader{Nulo} & \tablaheader{Descripción} \\
\hline
\endfirsthead
\hline
\tablaheader{Campo} & \tablaheader{Tipo} & \tablaheader{Nulo} & \tablaheader{Descripción} \\
\hline
\endhead

IdPermiso & INT & NO & PK autoincremental. \\
\hline
IdRol & INT & NO & FK hacia Roles. \\
\hline
Modulo & VARCHAR(50) & NO & Nombre del módulo (ej: Pacientes, Citas, Diagnosticos). \\
\hline
PermisoCrear & BIT & NO & 1=Puede crear registros. DEFAULT 0. \\
\hline
PermisoLeer & BIT & NO & 1=Puede leer registros. DEFAULT 0. \\
\hline
PermisoActualizar & BIT & NO & 1=Puede actualizar registros. DEFAULT 0. \\
\hline
PermisoEliminar & BIT & NO & 1=Puede eliminar registros. DEFAULT 0. \\
\hline
\caption{Tabla 15: Permisos - Seguridad}
\end{longtable}

\textbf{Relaciones:}
\begin{itemize}[leftmargin=1.5em]
    \item N:1 con Roles (Múltiples permisos por rol)
\end{itemize}

\subsubsection{TABLA 16: AuditLog}

\textbf{Descripción:} Log de auditoría inmutable de TODAS las operaciones críticas del sistema.

\textbf{Primary Key:} IdAudit (BIGINT IDENTITY)

\begin{longtable}{|p{4.2cm}|p{3.5cm}|p{1cm}|p{5m}|}
\hline
\tablaheader{Campo} & \tablaheader{Tipo} & \tablaheader{Nulo} & \tablaheader{Descripción} \\
\hline
\endfirsthead
\hline
\tablaheader{Campo} & \tablaheader{Tipo} & \tablaheader{Nulo} & \tablaheader{Descripción} \\
\hline
\endhead

IdAudit & BIGINT & NO & PK autoincremental (soporta miles de millones de registros). \\
\hline
TablaAfectada & VARCHAR(100) & NO & Nombre de la tabla afectada. Índice. \\
\hline
Operacion & VARCHAR(10) & NO & INSERT, UPDATE, DELETE. CHECK. \\
\hline
IdRegistro & INT & NO & ID del registro afectado. \\
\hline
UsuarioID & INT & NO & ID del usuario que realizó la operación. \\
\hline
UsuarioNombre & NVARCHAR(100) & NO & Nombre del usuario (desnormalizado para consultas). \\
\hline
FechaHora & DATETIME2 & NO & Timestamp exacto. DEFAULT SYSDATETIME(). \\
\hline
ValoresAnteriores & NVARCHAR(MAX) & SÍ & JSON con valores anteriores (UPDATE/DELETE). \\
\hline
ValoresNuevos & NVARCHAR(MAX) & SÍ & JSON con valores nuevos (INSERT/UPDATE). \\
\hline
DireccionIP & VARCHAR(50) & SÍ & Dirección IP del cliente. \\
\hline
NombrePC & VARCHAR(100) & SÍ & Nombre del equipo (HOST\_NAME()). \\
\hline
\caption{Tabla 16: AuditLog - Auditoría General}
\end{longtable}

\textbf{Características especiales:}
\begin{itemize}[leftmargin=1.5em]
    \item Tabla write-only (no se permiten DELETE ni UPDATE)
    \item Retención mínima: 3 años por normativa
    \item Índice optimizado: (TablaAfectada, FechaHora DESC)
\end{itemize}

\subsubsection{TABLA 17: Auditoria\_Pacientes}

\textbf{Descripción:} Auditoría específica de cambios en historias clínicas de pacientes (tabla crítica).

\textbf{Primary Key:} IdAuditoriaPaciente (BIGINT IDENTITY)\\
\textbf{Foreign Keys:} IdPaciente → Pacientes, UsuarioID → Usuarios

\begin{longtable}{|p{4.2cm}|p{3.5cm}|p{1cm}|p{5cm}|}
\hline
\tablaheader{Campo} & \tablaheader{Tipo} & \tablaheader{Nulo} & \tablaheader{Descripción} \\
\hline
\endfirsthead
\hline
\tablaheader{Campo} & \tablaheader{Tipo} & \tablaheader{Nulo} & \tablaheader{Descripción} \\
\hline
\endhead

IdAuditoriaPaciente & BIGINT & NO & PK autoincremental. \\
\hline
IdPaciente & INT & NO & FK hacia Pacientes. \\
\hline
TipoOperacion & VARCHAR(10) & NO & INSERT, UPDATE, DELETE. \\
\hline
CampoModificado & VARCHAR(100) & SÍ & Nombre del campo modificado (NULL en INSERT). \\
\hline
ValorAnterior & NVARCHAR(MAX) & SÍ & Valor anterior del campo. \\
\hline
ValorNuevo & NVARCHAR(MAX) & SÍ & Valor nuevo del campo. \\
\hline
FechaHora & DATETIME2 & NO & Timestamp de la modificación. DEFAULT SYSDATETIME(). \\
\hline
UsuarioID & INT & NO & FK hacia Usuarios. Quién realizó el cambio. \\
\hline
UsuarioNombre & NVARCHAR(100) & NO & Nombre del usuario (desnormalizado). \\
\hline
IPAddress & VARCHAR(50) & SÍ & Dirección IP de origen. \\
\hline
\caption{Tabla 17: Auditoria\_Pacientes - Auditoría Específica}
\end{longtable}
\textbf{Trigger asociado:} TRG\_Auditoria\_Pacientes (AFTER INSERT, UPDATE, DELETE)

\newpage

\subsubsection{TABLA 18: Auditoria\_Diagnosticos}
\textbf{Descripción:} Auditoría específica de diagnósticos médicos (dato crítico con implicancias legales).

\textbf{Primary Key:} IdAuditoriaDiagnostico (BIGINT IDENTITY)\\
\textbf{Foreign Keys:} IdDiagnostico → Diagnosticos, UsuarioID → Usuarios

\begin{longtable}{|p{4.2cm}|p{3.5cm}|p{1cm}|p{5cm}|}
\hline
\tablaheader{Campo} & \tablaheader{Tipo} & \tablaheader{Nulo} & \tablaheader{Descripción} \\
\hline
\endfirsthead
\hline
\tablaheader{Campo} & \tablaheader{Tipo} & \tablaheader{Nulo} & \tablaheader{Descripción} \\
\hline
\endhead

IdAuditoriaDiagnostico & BIGINT & NO & PK autoincremental. \\
\hline
IdDiagnostico & INT & NO & FK hacia Diagnosticos. \\
\hline
TipoOperacion & VARCHAR(10) & NO & INSERT, UPDATE, DELETE. \\
\hline
CodigoCIE10Anterior & VARCHAR(10) & SÍ & Código CIE-10 anterior (UPDATE/DELETE). \\
\hline
CodigoCIE10Nuevo & VARCHAR(10) & SÍ & Código CIE-10 nuevo (INSERT/UPDATE). \\
\hline
DescripcionAnterior & NVARCHAR(500) & SÍ & Descripción diagnóstico anterior. \\
\hline
DescripcionNueva & NVARCHAR(500) & SÍ & Descripción diagnóstico nueva. \\
\hline
Justificacion & NVARCHAR(500) & NO & Justificación OBLIGATORIA del cambio (UPDATE/DELETE). \\
\hline
FechaHora & DATETIME2 & NO & Timestamp. DEFAULT SYSDATETIME(). \\
\hline
UsuarioID & INT & NO & FK hacia Usuarios. Médico que realizó el cambio. \\
\hline
UsuarioNombre & NVARCHAR(100) & NO & Nombre del médico. \\
\hline
\caption{Tabla 18: Auditoria\_Diagnosticos - Auditoría }
\end{longtable}

\textbf{Importante:} Cualquier modificación o eliminación de diagnósticos requiere justificación médica obligatoria por implicancias médico-legales.

% =====================================================
% 6.4 DIAGRAMA DE RELACIONES ENTRE ENTIDADES
% =====================================================

% =====================================================
% 8. SCRIPTS SQL COMPLETOS
% =====================================================
\newpage
% =====================================================
% 11. TECNOLOGÍAS Y HERRAMIENTAS
% =====================================================

\section{Tecnologías y Herramientas}

\subsection{Lenguajes y Plataformas}
El Sistema Integral de Gestión de Historias Clínicas (SIGHC) ha sido diseñado bajo una arquitectura cliente--servidor, priorizando la separación de responsabilidades, la escalabilidad y la seguridad de la información.  
Para la capa lógica del sistema se consideran lenguajes de programación de propósito general como Java, .NET o Python, los cuales permiten implementar reglas de negocio complejas, validaciones clínicas y mecanismos de auditoría necesarios en sistemas del sector salud.
\subsection{Principios de Diseño de Software}

Durante el diseño e implementación del Sistema Integral de Gestión de Historias Clínicas (SIGHC) se aplicaron principios de diseño orientado a objetos, en particular los principios SOLID, con el objetivo de mejorar la mantenibilidad, extensibilidad y calidad del software.

El principio de responsabilidad única (Single Responsibility Principle) fue aplicado para asegurar que cada módulo y componente del sistema tenga una única responsabilidad bien definida, facilitando el mantenimiento y la comprensión del código.  
Asimismo, el principio de abierto/cerrado (Open/Closed Principle) permitió que los componentes del sistema puedan extenderse mediante nuevas funcionalidades sin modificar el código existente.

El principio de sustitución de Liskov (Liskov Substitution Principle) garantizó la correcta utilización de abstracciones y herencia, evitando dependencias incorrectas entre componentes.  
Por su parte, el principio de segregación de interfaces (Interface Segregation Principle) fue considerado para definir interfaces específicas y coherentes, evitando dependencias innecesarias.

Finalmente, el principio de inversión de dependencias (Dependency Inversion Principle) permitió desacoplar las capas del sistema, favoreciendo la independencia entre la lógica de negocio y los detalles de implementación, contribuyendo a una arquitectura más flexible y escalable.



\subsection{Base de Datos}
La persistencia de la información se gestiona mediante una base de datos relacional implementada en Microsoft SQL Server o PostgreSQL. Estas plataformas fueron seleccionadas por su soporte avanzado de transacciones ACID, integridad referencial, procedimientos almacenados, disparadores (triggers) y vistas optimizadas.  
El modelo de datos del SIGHC se encuentra normalizado hasta la Tercera Forma Normal (3NF), garantizando consistencia, reducción de redundancia y eficiencia en el acceso a la información clínica.

\subsection{Control de Versiones}
Para la gestión del código fuente y los scripts SQL se utiliza Git como sistema de control de versiones, con repositorios alojados en GitHub. Esta herramienta permite mantener trazabilidad de cambios, trabajo colaborativo, control de incidencias y recuperación ante errores durante el desarrollo.


El control de versiones fue aplicado tanto a los scripts de base de datos como a la documentación técnica en \LaTeX, permitiendo la corrección progresiva de errores y la consolidación de una versión estable del sistema.

\subsection{Cronología de Versiones y Correcciones}

\begin{table}[H]
\centering
\small
\begin{tabular}{|c|c|p{10cm}|}
\hline
\textbf{Fecha} & \textbf{Versión} & \textbf{Cambios y correcciones realizadas} \\
\hline
02/12/2025 & v0.1 &
Creación inicial del repositorio GitHub. Estructuración del proyecto SIGHC. Registro de los primeros scripts de base de datos y plantilla base del documento en \LaTeX. \\ 
\hline
04/12/2025 & v0.2 &
Corrección de errores de compilación en \LaTeX relacionados con paquetes, codificación de caracteres y entornos no cerrados. Ajuste de numeración de secciones y tablas. \\ 
\hline
06/12/2025 & v0.3 &
Primera versión funcional del modelo de base de datos. Corrección del orden de creación de tablas y claves foráneas para evitar errores de dependencia en SQL Server. \\ 
\hline
08/12/2025 & v0.4 &
Identificación y corrección de inconsistencias en el modelo RBAC. Refinamiento de roles y permisos conforme al principio de mínimo privilegio. \\ 
\hline
10/12/2025 & v0.5 &
Corrección de errores de conexión a la base de datos y ajustes en las cadenas de conexión. Validación del entorno de desarrollo y servicios activos. \\ 
\hline
12/12/2025 & v0.6 &
Optimización de procedimientos almacenados y triggers de auditoría. Corrección de errores de ejecución y mejora en la trazabilidad de operaciones críticas. \\ 
\hline
14/12/2025 & v0.7 &
Integración del módulo de seguridad: configuración de roles y permisos en SQL Server. Implementación de políticas de cifrado mediante Transparent Data Encryption (TDE). \\ 
\hline
16/12/2025 & v0.8 &
Corrección de errores de ejecución del servidor backend, incluyendo conflictos de puertos y fallos de inicialización. Ajustes finales en la documentación técnica. \\ 
\hline
17/12/2025 & v1.0 &
Versión final estable del proyecto. Validación integral de scripts SQL, documentación completa del sistema y consolidación del informe técnico para entrega académica. \\ 
\hline
\end{tabular}
\caption{Historial de versiones y correcciones del proyecto SIGHC}
\end{table}

\subsection{Beneficios del Control de Versiones}

El uso sistemático del control de versiones permitió mantener un registro detallado de los errores detectados y corregidos durante el desarrollo del proyecto, facilitando la reversión de cambios, la depuración progresiva del sistema y la consolidación de una versión final estable. Asimismo, evidenció un proceso de desarrollo organizado, alineado con buenas prácticas de ingeniería de software.


\subsection{Entorno de Desarrollo}
El desarrollo del sistema se realizó utilizando Visual Studio Code como entorno de programación, complementado con SQL Server Management Studio o pgAdmin para la administración de la base de datos. Este entorno facilita la depuración, validación y pruebas de los componentes del sistema.
\subsubsection{SQL Server Management Studio (SSMS)}

SQL Server Management Studio (SSMS) fue utilizado como herramienta principal para la administración, diseño y gestión de la base de datos del sistema SIGHC. Esta herramienta proporciona un entorno robusto para la creación y mantenimiento de bases de datos relacionales, permitiendo la ejecución controlada de scripts SQL, la definición de estructuras de datos y la validación de reglas de integridad.

A través de SSMS se desarrollaron y gestionaron los scripts de creación de tablas, claves primarias y foráneas, procedimientos almacenados, funciones y disparadores de auditoría, asegurando la coherencia del modelo de datos y el cumplimiento de las formas normales. Asimismo, la herramienta facilitó la configuración de mecanismos de seguridad, tales como la implementación de control de acceso basado en roles (RBAC), la asignación de permisos y la activación de cifrado de datos.

El uso de SQL Server Management Studio permitió además realizar pruebas controladas de consultas, validar el rendimiento de las operaciones y verificar la correcta ejecución de los mecanismos de auditoría y trazabilidad definidos para el sistema.

\subsubsection{Visual Studio y Visual Studio Code}

Visual Studio y Visual Studio Code fueron empleados como entornos de desarrollo para la gestión del código fuente, scripts complementarios y documentación asociada al proyecto. Estas herramientas proporcionan soporte avanzado para múltiples lenguajes de programación, control de versiones y edición estructurada de archivos, lo que facilitó el desarrollo organizado del sistema.

Visual Studio fue utilizado principalmente para la integración de componentes del sistema y la validación de conexiones con la base de datos, mientras que Visual Studio Code permitió una edición ágil y eficiente de scripts SQL, archivos de configuración y documentación técnica en \LaTeX. La integración con sistemas de control de versiones permitió mantener un historial detallado de cambios, identificar errores y asegurar la consistencia entre las distintas versiones del proyecto.

El uso combinado de estas herramientas contribuyó a mejorar la productividad, reducir errores de implementación y garantizar la mantenibilidad del sistema, proporcionando un entorno de trabajo alineado con los estándares actuales de desarrollo de software.

% =====================================================
% 12. IMPLEMENTACIÓN DEL SISTEMA
% =====================================================
\newpage
\section{Implementación del Sistema}

\subsection{Implementación de la Base de Datos}
La implementación de la base de datos del SIGHC se realizó mediante scripts SQL estructurados que definen tablas, claves primarias, claves foráneas, restricciones de integridad e índices de optimización.  
El diseño contempla 18 tablas principales organizadas de acuerdo con los módulos funcionales del sistema, representando de manera precisa el dominio clínico y administrativo.

\subsection{Procedimientos Almacenados}
Se desarrollaron procedimientos almacenados para encapsular operaciones críticas del sistema, tales como el registro de pacientes, la programación de citas médicas, el registro de consultas y diagnósticos, y la prescripción de tratamientos.  
El uso de procedimientos almacenados mejora el rendimiento del sistema, centraliza la lógica de negocio y refuerza la seguridad al limitar el acceso directo a las tablas.

\subsection{Triggers y Auditoría}
El sistema incorpora triggers automáticos de auditoría que registran las operaciones de inserción, actualización y eliminación sobre tablas sensibles como pacientes, diagnósticos y tratamientos.  
Estos mecanismos garantizan la trazabilidad completa de la información, permitiendo identificar el usuario, la fecha y la acción realizada, cumpliendo con los requerimientos normativos del sector salud.

\subsection{Vistas y Reportes}
Se implementaron vistas SQL para facilitar consultas optimizadas, tales como la visualización de pacientes activos, la agenda médica diaria y las estadísticas de diagnósticos.  
Estas vistas sirven como base para la generación de reportes clínicos y administrativos, mejorando la eficiencia en la toma de decisiones.


\subsection{Script Completo de Creación de Tablas}

\begin{lstlisting}[caption={Creación de Base de Datos y Tablas Principales}]
-- ========================================
-- CREACIÓN DE BASE DE DATOS SIGHC
-- Sistema Integral de Gestión de Historias Clínicas
-- Versión: 1.0
-- Fecha: Diciembre 2025
-- ========================================

-- Crear base de datos principal
CREATE DATABASE SIGHC
ON PRIMARY (
    NAME = 'SIGHC_Data',
    FILENAME = 'C:\SQLData\SIGHC_Data.mdf',
    SIZE = 500MB,
    FILEGROWTH = 100MB
)
LOG ON (
    NAME = 'SIGHC_Log',
    FILENAME = 'C:\SQLData\SIGHC_Log.ldf',
    SIZE = 200MB,
    FILEGROWTH = 50MB
);
GO

USE SIGHC;
GO

-- ========================================
-- TABLA 01: Pacientes (Entidad Principal)
-- ========================================
CREATE TABLE Pacientes (
    IdPaciente INT IDENTITY(1,1) PRIMARY KEY,
    NroHistoriaClinica VARCHAR(15) UNIQUE NOT NULL,
    Nombres NVARCHAR(80) NOT NULL,
    Apellidos NVARCHAR(80) NOT NULL,
    DNI VARCHAR(8) UNIQUE NOT NULL,
    FechaNacimiento DATE NOT NULL,
    Sexo CHAR(1) NOT NULL CHECK (Sexo IN ('M','F')),
    GrupoSanguineo VARCHAR(5),
    Direccion NVARCHAR(200),
    Telefono VARCHAR(15),
    Email VARCHAR(100),
    AntecedentesFamiliares NVARCHAR(MAX),
    AntecedentesPersonales NVARCHAR(MAX),
    Alergias NVARCHAR(MAX),
    FechaRegistro DATETIME2 DEFAULT SYSDATETIME() NOT NULL,
    Estado CHAR(1) DEFAULT 'A' CHECK (Estado IN ('A','I','F')),
    UsuarioRegistro INT NOT NULL
);
GO

-- Comentarios de documentación
EXEC sys.sp_addextendedproperty 
    @name=N'MS_Description', 
    @value=N'Tabla principal que almacena datos demográficos y clínicos de pacientes' , 
    @level0type=N'SCHEMA',@level0name=N'dbo', 
    @level1type=N'TABLE',@level1name=N'Pacientes';
GO

-- ========================================
-- TABLA 02: Especialidades (Catálogo)
-- ========================================
CREATE TABLE Especialidades (
    IdEspecialidad INT IDENTITY(1,1) PRIMARY KEY,
    NombreEspecialidad NVARCHAR(100) UNIQUE NOT NULL,
    Descripcion NVARCHAR(500),
    Estado CHAR(1) DEFAULT 'A' CHECK (Estado IN ('A','I'))
);
GO

-- ========================================
-- TABLA 03: Medicos (Personal Médico)
-- ========================================
CREATE TABLE Medicos (
    IdMedico INT IDENTITY(1,1) PRIMARY KEY,
    Nombres NVARCHAR(80) NOT NULL,
    Apellidos NVARCHAR(80) NOT NULL,
    DNI VARCHAR(8) UNIQUE NOT NULL,
    CMP VARCHAR(10) UNIQUE NOT NULL,
    RNE VARCHAR(10),
    IdEspecialidad INT NOT NULL,
    Telefono VARCHAR(15),
    Email VARCHAR(100) NOT NULL,
    FechaIngreso DATE NOT NULL,
    Estado CHAR(1) DEFAULT 'A' CHECK (Estado IN ('A','I','S','R')),
    CONSTRAINT FK_Medicos_Especialidades 
        FOREIGN KEY (IdEspecialidad) 
        REFERENCES Especialidades(IdEspecialidad)
);
GO

-- ========================================
-- TABLA 04: Citas (Agenda Médica)
-- ========================================
CREATE TABLE Citas (
    IdCita INT IDENTITY(1,1) PRIMARY KEY,
    CodigoCita VARCHAR(20) UNIQUE NOT NULL,
    IdPaciente INT NOT NULL,
    IdMedico INT NOT NULL,
    FechaCita DATE NOT NULL,
    HoraInicio TIME NOT NULL,
    HoraFin TIME NOT NULL,
    MotivoConsulta NVARCHAR(500) NOT NULL,
    TipoCita VARCHAR(20) NOT NULL 
        CHECK (TipoCita IN ('PrimeraVez','Control','Emergencia')),
    Estado VARCHAR(20) DEFAULT 'Programada' NOT NULL
        CHECK (Estado IN ('Programada','Confirmada','Atendida','Cancelada','Reprogramada')),
    MotivoCancelacion NVARCHAR(200),
    FechaRegistro DATETIME2 DEFAULT SYSDATETIME() NOT NULL,
    UsuarioRegistro INT NOT NULL,
    CONSTRAINT FK_Citas_Pacientes 
        FOREIGN KEY (IdPaciente) REFERENCES Pacientes(IdPaciente),
    CONSTRAINT FK_Citas_Medicos 
        FOREIGN KEY (IdMedico) REFERENCES Medicos(IdMedico)
);
GO

-- ========================================
-- TABLA 05: Consultas (Atención Médica)
-- ========================================
CREATE TABLE Consultas (
    IdConsulta INT IDENTITY(1,1) PRIMARY KEY,
    IdCita INT NOT NULL UNIQUE,
    IdPaciente INT NOT NULL,
    IdMedico INT NOT NULL,
    FechaConsulta DATETIME2 DEFAULT SYSDATETIME() NOT NULL,
    -- Signos Vitales
    PresionArterial VARCHAR(10),
    Temperatura DECIMAL(4,1),
    FrecuenciaCardiaca INT,
    FrecuenciaRespiratoria INT,
    Peso DECIMAL(5,2),
    Talla DECIMAL(5,2),
    IMC AS (CASE WHEN Talla > 0 THEN Peso / (Talla * Talla) ELSE NULL END) PERSISTED,
    SaturacionO2 INT,
    -- Anamnesis
    MotivoConsulta NVARCHAR(MAX) NOT NULL,
    TiempoEnfermedad NVARCHAR(200),
    RelatoCronico NVARCHAR(MAX),
    ExamenFisico NVARCHAR(MAX),
    PlanTrabajo NVARCHAR(MAX),
    CONSTRAINT FK_Consultas_Citas 
        FOREIGN KEY (IdCita) REFERENCES Citas(IdCita),
    CONSTRAINT FK_Consultas_Pacientes 
        FOREIGN KEY (IdPaciente) REFERENCES Pacientes(IdPaciente),
    CONSTRAINT FK_Consultas_Medicos 
        FOREIGN KEY (IdMedico) REFERENCES Medicos(IdMedico)
);
GO

-- ========================================
-- TABLA 06: CIE10 (Catálogo de Enfermedades)
-- ========================================
CREATE TABLE CIE10 (
    CodigoCIE10 VARCHAR(10) PRIMARY KEY,
    Descripcion NVARCHAR(500) NOT NULL,
    Capitulo VARCHAR(10) NOT NULL,
    DescripcionCapitulo NVARCHAR(200) NOT NULL,
    Sexo CHAR(1) CHECK (Sexo IN ('M','F')),
    EdadMinima INT,
    EdadMaxima INT,
    NotificacionObligatoria BIT DEFAULT 0,
    Estado CHAR(1) DEFAULT 'A' CHECK (Estado IN ('A','I'))
);
GO

-- Índice FULLTEXT para búsquedas rápidas
CREATE FULLTEXT CATALOG CatalogoCIE10 AS DEFAULT;
GO

CREATE FULLTEXT INDEX ON CIE10(Descripcion)
    KEY INDEX PK__CIE10 ON CatalogoCIE10;
GO

-- ========================================
-- TABLA 07: Diagnosticos (Diagnósticos Médicos)
-- ========================================
CREATE TABLE Diagnosticos (
    IdDiagnostico INT IDENTITY(1,1) PRIMARY KEY,
    IdConsulta INT NOT NULL,
    CodigoCIE10 VARCHAR(10) NOT NULL,
    DescripcionDiagnostico NVARCHAR(500) NOT NULL,
    TipoDiagnostico VARCHAR(20) NOT NULL 
        CHECK (TipoDiagnostico IN ('Presuntivo','Definitivo')),
    Clasificacion VARCHAR(20) NOT NULL 
        CHECK (Clasificacion IN ('Principal','Secundario','Complicacion')),
    FechaRegistro DATETIME2 DEFAULT SYSDATETIME() NOT NULL,
    CONSTRAINT FK_Diagnosticos_Consultas 
        FOREIGN KEY (IdConsulta) REFERENCES Consultas(IdConsulta),
    CONSTRAINT FK_Diagnosticos_CIE10 
        FOREIGN KEY (CodigoCIE10) REFERENCES CIE10(CodigoCIE10)
);
GO

-- ========================================
-- TABLA 08: Medicamentos (Catálogo)
-- ========================================
CREATE TABLE Medicamentos (
    IdMedicamento INT IDENTITY(1,1) PRIMARY KEY,
    CodigoMedicamento VARCHAR(20) UNIQUE NOT NULL,
    NombreGenerico NVARCHAR(200) NOT NULL,
    NombreComercial NVARCHAR(200),
    Presentacion NVARCHAR(100) NOT NULL,
    Concentracion NVARCHAR(50),
    FormaFarmaceutica VARCHAR(50) NOT NULL,
    UnidadMedida VARCHAR(20) NOT NULL,
    StockMinimo INT DEFAULT 10,
    StockActual INT DEFAULT 0,
    PrecioUnitario DECIMAL(10,2),
    RequiereReceta BIT DEFAULT 1,
    Estado CHAR(1) DEFAULT 'A' CHECK (Estado IN ('A','I'))
);
GO

-- ========================================
-- TABLA 09: Tratamientos (Prescripciones)
-- ========================================
CREATE TABLE Tratamientos (
    IdTratamiento INT IDENTITY(1,1) PRIMARY KEY,
    IdDiagnostico INT NOT NULL,
    IdMedicamento INT NOT NULL,
    Dosis NVARCHAR(100) NOT NULL,
    Frecuencia NVARCHAR(100) NOT NULL,
    ViaAdministracion VARCHAR(50) NOT NULL 
        CHECK (ViaAdministracion IN ('Oral','Endovenosa','Intramuscular','Topica','Sublingual','Rectal')),
    Duracion INT NOT NULL,
    IndicacionesEspeciales NVARCHAR(500),
    FechaInicio DATE NOT NULL,
    FechaFin AS DATEADD(DAY, Duracion, FechaInicio) PERSISTED,
    CONSTRAINT FK_Tratamientos_Diagnosticos 
        FOREIGN KEY (IdDiagnostico) REFERENCES Diagnosticos(IdDiagnostico),
    CONSTRAINT FK_Tratamientos_Medicamentos 
        FOREIGN KEY (IdMedicamento) REFERENCES Medicamentos(IdMedicamento)
);
GO

-- ========================================
-- TABLA 10: Usuarios (Seguridad)
-- ========================================
CREATE TABLE Usuarios (
    IdUsuario INT IDENTITY(1,1) PRIMARY KEY,
    NombreUsuario VARCHAR(50) UNIQUE NOT NULL,
    PasswordHash VARBINARY(64) NOT NULL,
    PasswordSalt VARBINARY(32) NOT NULL,
    NombreCompleto NVARCHAR(150) NOT NULL,
    Email VARCHAR(100) UNIQUE NOT NULL,
    IdRol INT NOT NULL,
    UltimoAcceso DATETIME2,
    CambioPasswordObligatorio BIT DEFAULT 1,
    IntentosAccesoFallido INT DEFAULT 0,
    CuentaBloqueada BIT DEFAULT 0,
    FechaCreacion DATETIME2 DEFAULT SYSDATETIME() NOT NULL,
    Estado CHAR(1) DEFAULT 'A' CHECK (Estado IN ('A','I'))
);
GO

-- ========================================
-- TABLA 11: Roles (RBAC)
-- ========================================
CREATE TABLE Roles (
    IdRol INT IDENTITY(1,1) PRIMARY KEY,
    NombreRol VARCHAR(50) UNIQUE NOT NULL,
    Descripcion NVARCHAR(200),
    Nivel INT NOT NULL,
    Estado CHAR(1) DEFAULT 'A' CHECK (Estado IN ('A','I'))
);
GO

ALTER TABLE Usuarios ADD CONSTRAINT FK_Usuarios_Roles
    FOREIGN KEY (IdRol) REFERENCES Roles(IdRol);
GO

-- ========================================
-- TABLA 12: AuditLog (Auditoría Inmutable)
-- ========================================
CREATE TABLE AuditLog (
    IdAudit BIGINT IDENTITY(1,1) PRIMARY KEY,
    TablaAfectada VARCHAR(100) NOT NULL,
    Operacion VARCHAR(10) NOT NULL CHECK (Operacion IN ('INSERT','UPDATE','DELETE')),
    IdRegistro INT NOT NULL,
    UsuarioID INT NOT NULL,
    UsuarioNombre NVARCHAR(100) NOT NULL,
    FechaHora DATETIME2 DEFAULT SYSDATETIME() NOT NULL,
    ValoresAnteriores NVARCHAR(MAX),
    ValoresNuevos NVARCHAR(MAX),
    DireccionIP VARCHAR(50),
    NombrePC VARCHAR(100)
);
GO

-- Índice optimizado para consultas de auditoría
CREATE NONCLUSTERED INDEX IX_AuditLog_Tabla_Fecha
    ON AuditLog(TablaAfectada, FechaHora DESC)
    INCLUDE (Operacion, UsuarioNombre);
GO

-- ========================================
-- INSERCIÓN DE DATOS INICIALES
-- ========================================

-- Especialidades médicas
INSERT INTO Especialidades (NombreEspecialidad, Descripcion) VALUES
('Medicina General', 'Atención médica integral para adultos'),
('Pediatría', 'Atención especializada de niños y adolescentes'),
('Cardiología', 'Enfermedades del corazón y sistema cardiovascular'),
('Ginecología', 'Salud reproductiva femenina'),
('Traumatología', 'Lesiones del sistema musculoesquelético'),
('Neurología', 'Enfermedades del sistema nervioso'),
('Oftalmología', 'Enfermedades de los ojos'),
('Dermatología', 'Enfermedades de la piel');
GO

-- Roles del sistema
INSERT INTO Roles (NombreRol, Descripcion, Nivel) VALUES
('Administrador', 'Acceso total al sistema', 1),
('Médico', 'Acceso a consultas y diagnósticos', 2),
('Enfermera', 'Acceso a citas y signos vitales', 3),
('Recepcionista', 'Registro de pacientes y citas', 4),
('Farmacia', 'Acceso a inventario y prescripciones', 5),
('Auditor', 'Consulta de logs y reportes', 6);
GO
\end{lstlisting}

\subsubsection{Índices de Optimización}

\begin{lstlisting}[caption={Creación de Índices para Optimización de Consultas}]
-- ========================================
-- ÍNDICES DE OPTIMIZACIÓN
-- ========================================

-- Índices en Pacientes
CREATE NONCLUSTERED INDEX IX_Pacientes_DNI ON Pacientes(DNI);
CREATE NONCLUSTERED INDEX IX_Pacientes_Nombres ON Pacientes(Nombres, Apellidos);
CREATE NONCLUSTERED INDEX IX_Pacientes_Estado ON Pacientes(Estado) WHERE Estado = 'A';

-- Índices en Citas
CREATE NONCLUSTERED INDEX IX_Citas_Paciente_Fecha 
    ON Citas(IdPaciente, FechaCita DESC);
CREATE NONCLUSTERED INDEX IX_Citas_Medico_Fecha 
    ON Citas(IdMedico, FechaCita, HoraInicio);
CREATE NONCLUSTERED INDEX IX_Citas_Estado 
    ON Citas(Estado, FechaCita);

-- Índices en Consultas
CREATE NONCLUSTERED INDEX IX_Consultas_Paciente 
    ON Consultas(IdPaciente, FechaConsulta DESC);
CREATE NONCLUSTERED INDEX IX_Consultas_Medico 
    ON Consultas(IdMedico, FechaConsulta DESC);

-- Índices en Diagnósticos
CREATE NONCLUSTERED INDEX IX_Diagnosticos_Consulta 
    ON Diagnosticos(IdConsulta);
CREATE NONCLUSTERED INDEX IX_Diagnosticos_CIE10 
    ON Diagnosticos(CodigoCIE10, FechaRegistro DESC);

-- Índices en Medicamentos
CREATE NONCLUSTERED INDEX IX_Medicamentos_Nombre 
    ON Medicamentos(NombreGenerico);
CREATE NONCLUSTERED INDEX IX_Medicamentos_Stock 
    ON Medicamentos(StockActual) WHERE StockActual < StockMinimo;
GO
\end{lstlisting}

% =====================================================
% 9. PROCEDIMIENTOS ALMACENADOS
% =====================================================
\newpage
\subsection{Procedimientos Almacenados}

\subsubsection{SP\_RegistrarPaciente}

\begin{lstlisting}[caption={Procedimiento para Registrar Nuevo Paciente}]
-- ========================================
-- SP_RegistrarPaciente
-- Registra un nuevo paciente generando historia clínica única
-- ========================================
CREATE PROCEDURE SP_RegistrarPaciente
    @DNI VARCHAR(8),
    @Nombres NVARCHAR(80),
    @Apellidos NVARCHAR(80),
    @FechaNacimiento DATE,
    @Sexo CHAR(1),
    @Direccion NVARCHAR(200),
    @Telefono VARCHAR(15),
    @Email VARCHAR(100),
    @GrupoSanguineo VARCHAR(5),
    @UsuarioRegistro INT,
    @IdPacienteOut INT OUTPUT,
    @NroHistoriaOut VARCHAR(15) OUTPUT
AS
BEGIN
    SET NOCOUNT ON;
    DECLARE @Anio VARCHAR(4) = YEAR(GETDATE());
    DECLARE @Correlativo INT;
    
    BEGIN TRY
        BEGIN TRANSACTION;
        
        -- Verificar DNI duplicado
        IF EXISTS (SELECT 1 FROM Pacientes WHERE DNI = @DNI)
        BEGIN
            RAISERROR('El DNI ya está registrado en el sistema', 16, 1);
            RETURN;
        END
        
        -- Obtener siguiente correlativo
        SELECT @Correlativo = ISNULL(MAX(CAST(RIGHT(NroHistoriaClinica, 5) AS INT)), 0) + 1
        FROM Pacientes
        WHERE NroHistoriaClinica LIKE 'HC-' + @Anio + '-%';
        
        -- Generar número de historia clínica
        SET @NroHistoriaOut = 'HC-' + @Anio + '-' + RIGHT('00000' + CAST(@Correlativo AS VARCHAR), 5);
        
        -- Insertar paciente
        INSERT INTO Pacientes (
            NroHistoriaClinica, Nombres, Apellidos, DNI, FechaNacimiento,
            Sexo, Direccion, Telefono, Email, GrupoSanguineo, UsuarioRegistro
        ) VALUES (
            @NroHistoriaOut, @Nombres, @Apellidos, @DNI, @FechaNacimiento,
            @Sexo, @Direccion, @Telefono, @Email, @GrupoSanguineo, @UsuarioRegistro
        );
        
        SET @IdPacienteOut = SCOPE_IDENTITY();
        
        COMMIT TRANSACTION;
        
        PRINT 'Paciente registrado exitosamente: ' + @NroHistoriaOut;
        
    END TRY
    BEGIN CATCH
        IF @@TRANCOUNT > 0 ROLLBACK TRANSACTION;
        
        DECLARE @ErrorMsg NVARCHAR(4000) = ERROR_MESSAGE();
        RAISERROR(@ErrorMsg, 16, 1);
    END CATCH
END;
GO
\end{lstlisting}

\subsubsection{SP\_ProgramarCita}

\begin{lstlisting}[caption={Procedimiento para Programar Cita Médica}]
-- ========================================
-- SP_ProgramarCita
-- Programa una nueva cita validando disponibilidad
-- ========================================
CREATE PROCEDURE SP_ProgramarCita
    @IdPaciente INT,
    @IdMedico INT,
    @FechaCita DATE,
    @HoraInicio TIME,
    @MotivoConsulta NVARCHAR(500),
    @TipoCita VARCHAR(20),
    @UsuarioRegistro INT,
    @IdCitaOut INT OUTPUT,
    @CodigoCitaOut VARCHAR(20) OUTPUT
AS
BEGIN
    SET NOCOUNT ON;
    DECLARE @HoraFin TIME = DATEADD(MINUTE, 30, @HoraInicio);
    
    BEGIN TRY
        BEGIN TRANSACTION;
        
        -- Validar que paciente existe y está activo
        IF NOT EXISTS (SELECT 1 FROM Pacientes WHERE IdPaciente = @IdPaciente AND Estado = 'A')
        BEGIN
            RAISERROR('El paciente no existe o está inactivo', 16, 1);
            RETURN;
        END
        
        -- Validar que médico existe y está activo
        IF NOT EXISTS (SELECT 1 FROM Medicos WHERE IdMedico = @IdMedico AND Estado = 'A')
        BEGIN
            RAISERROR('El médico no existe o está inactivo', 16, 1);
            RETURN;
        END
        
        -- Validar disponibilidad horaria (no existe cruce)
        IF EXISTS (
            SELECT 1 FROM Citas 
            WHERE IdMedico = @IdMedico 
            AND FechaCita = @FechaCita
            AND Estado NOT IN ('Cancelada', 'Reprogramada')
            AND (
                (@HoraInicio >= HoraInicio AND @HoraInicio < HoraFin) OR
                (@HoraFin > HoraInicio AND @HoraFin <= HoraFin)
            )
        )
        BEGIN
            RAISERROR('El médico ya tiene una cita en ese horario', 16, 1);
            RETURN;
        END
        
        -- Generar código de cita
        DECLARE @Correlativo INT;
        SELECT @Correlativo = ISNULL(MAX(CAST(RIGHT(CodigoCita, 6) AS INT)), 0) + 1
        FROM Citas
        WHERE CodigoCita LIKE 'CITA-' + CAST(YEAR(@FechaCita) AS VARCHAR) + '-%';
        
        SET @CodigoCitaOut = 'CITA-' + CAST(YEAR(@FechaCita) AS VARCHAR) + '-' + 
                             RIGHT('000000' + CAST(@Correlativo AS VARCHAR), 6);
        
        -- Insertar cita
        INSERT INTO Citas (
            CodigoCita, IdPaciente, IdMedico, FechaCita, HoraInicio, HoraFin,
            MotivoConsulta, TipoCita, UsuarioRegistro
        ) VALUES (
            @CodigoCitaOut, @IdPaciente, @IdMedico, @FechaCita, @HoraInicio, @HoraFin,
            @MotivoConsulta, @TipoCita, @UsuarioRegistro
        );
        
        SET @IdCitaOut = SCOPE_IDENTITY();
        
        COMMIT TRANSACTION;
        
        PRINT 'Cita programada exitosamente: ' + @CodigoCitaOut;
        
    END TRY
    BEGIN CATCH
        IF @@TRANCOUNT > 0 ROLLBACK TRANSACTION;
        
        DECLARE @ErrorMsg NVARCHAR(4000) = ERROR_MESSAGE();
        RAISERROR(@ErrorMsg, 16, 1);
    END CATCH
END;
GO
\end{lstlisting}

\subsubsection{SP\_RegistrarConsulta}

\begin{lstlisting}[caption={Procedimiento para Registrar Consulta Médica Completa}]
-- ========================================
-- SP_RegistrarConsulta
-- Registra consulta médica con signos vitales
-- ========================================
CREATE PROCEDURE SP_RegistrarConsulta
    @IdCita INT,
    @PresionArterial VARCHAR(10),
    @Temperatura DECIMAL(4,1),
    @FrecuenciaCardiaca INT,
    @Peso DECIMAL(5,2),
    @Talla DECIMAL(5,2),
    @MotivoConsulta NVARCHAR(MAX),
    @ExamenFisico NVARCHAR(MAX),
    @IdConsultaOut INT OUTPUT
AS
BEGIN
    SET NOCOUNT ON;
    DECLARE @IdPaciente INT, @IdMedico INT;
    
    BEGIN TRY
        BEGIN TRANSACTION;
        
        -- Obtener datos de la cita
        SELECT @IdPaciente = IdPaciente, @IdMedico = IdMedico
        FROM Citas WHERE IdCita = @IdCita AND Estado = 'Programada';
        
        IF @IdPaciente IS NULL
        BEGIN
            RAISERROR('La cita no existe o no está en estado Programada', 16, 1);
            RETURN;
        END
        
        -- Insertar consulta
        INSERT INTO Consultas (
            IdCita, IdPaciente, IdMedico, PresionArterial, Temperatura,
            FrecuenciaCardiaca, Peso, Talla, MotivoConsulta, ExamenFisico
        ) VALUES (
            @IdCita, @IdPaciente, @IdMedico, @PresionArterial, @Temperatura,
            @FrecuenciaCardiaca, @Peso, @Talla, @MotivoConsulta, @ExamenFisico
        );
        
        SET @IdConsultaOut = SCOPE_IDENTITY();
        
        -- Actualizar estado de cita
        UPDATE Citas SET Estado = 'Atendida' WHERE IdCita = @IdCita;
        
        COMMIT TRANSACTION;
        
    END TRY
    BEGIN CATCH
        IF @@TRANCOUNT > 0 ROLLBACK TRANSACTION;
        THROW;
    END CATCH
END;
GO
\end{lstlisting}

% =====================================================
% 10. FUNCIONES
% =====================================================
\newpage
\subsection{Funciones del Sistema}

\subsubsection{FN\_CalcularEdad}

\begin{lstlisting}[caption={Función para Calcular Edad del Paciente}]
-- ========================================
-- FN_CalcularEdad
-- Calcula edad precisa en años
-- ========================================
CREATE FUNCTION FN_CalcularEdad (@FechaNacimiento DATE)
RETURNS INT
AS
BEGIN
    DECLARE @Edad INT;
    
    SET @Edad = DATEDIFF(YEAR, @FechaNacimiento, GETDATE());
    
    -- Ajustar si no ha cumplido años todavía este año
    IF (MONTH(@FechaNacimiento) > MONTH(GETDATE()) OR 
       (MONTH(@FechaNacimiento) = MONTH(GETDATE()) AND DAY(@FechaNacimiento) > DAY(GETDATE())))
    BEGIN
        SET @Edad = @Edad - 1;
    END
    
    RETURN @Edad;
END;
GO

-- Ejemplo de uso:
-- SELECT Nombres, Apellidos, dbo.FN_CalcularEdad(FechaNacimiento) AS Edad FROM Pacientes;
\end{lstlisting}

\subsubsection{FN\_ObtenerHistorialPaciente}

\begin{lstlisting}[caption={Función para Obtener Historial Clínico Completo}]
-- ========================================
-- FN_ObtenerHistorialPaciente
-- Retorna historial clínico completo (TVF)
-- ========================================
CREATE FUNCTION FN_ObtenerHistorialPaciente (@IdPaciente INT)
RETURNS TABLE
AS
RETURN
(
    SELECT 
        c.FechaConsulta,
        CONCAT(m.Nombres, ' ', m.Apellidos) AS NombreMedico,
        e.NombreEspecialidad,
        c.MotivoConsulta,
        d.CodigoCIE10,
        d.DescripcionDiagnostico,
        d.TipoDiagnostico,
        STRING_AGG(CONCAT(med.NombreGenerico, ' ', t.Dosis, ' ', t.Frecuencia), '; ') AS Tratamientos
    FROM Consultas c
    INNER JOIN Medicos m ON c.IdMedico = m.IdMedico
    INNER JOIN Especialidades e ON m.IdEspecialidad = e.IdEspecialidad
    LEFT JOIN Diagnosticos d ON c.IdConsulta = d.IdConsulta
    LEFT JOIN Tratamientos t ON d.IdDiagnostico = t.IdDiagnostico
    LEFT JOIN Medicamentos med ON t.IdMedicamento = med.IdMedicamento
    WHERE c.IdPaciente = @IdPaciente
    GROUP BY c.FechaConsulta, m.Nombres, m.Apellidos, e.NombreEspecialidad,
             c.MotivoConsulta, d.CodigoCIE10, d.DescripcionDiagnostico, d.TipoDiagnostico
);
GO

-- Ejemplo de uso:
-- SELECT * FROM dbo.FN_ObtenerHistorialPaciente(1) ORDER BY FechaConsulta DESC;
\end{lstlisting}

% =====================================================
% 11. TRIGGERS DE AUDITORÍA
% =====================================================
\newpage
\subsection{Triggers de Auditoría Automática}

\subsubsection{TRG\_Auditoria\_Pacientes}

\begin{lstlisting}[caption={Trigger de Auditoría para Tabla Pacientes}]
-- ========================================
-- TRG_Auditoria_Pacientes
-- Registra automáticamente todos los cambios
-- ========================================
CREATE TRIGGER TRG_Auditoria_Pacientes
ON Pacientes
AFTER INSERT, UPDATE, DELETE
AS
BEGIN
    SET NOCOUNT ON;
    
    DECLARE @Operacion VARCHAR(10);
    DECLARE @UsuarioID INT = CAST(SESSION_CONTEXT(N'UsuarioID') AS INT);
    DECLARE @UsuarioNombre NVARCHAR(100) = CAST(SESSION_CONTEXT(N'UsuarioNombre') AS NVARCHAR(100));
    
    -- Determinar tipo de operación
    IF EXISTS (SELECT * FROM inserted) AND EXISTS (SELECT * FROM deleted)
        SET @Operacion = 'UPDATE';
    ELSE IF EXISTS (SELECT * FROM inserted)
        SET @Operacion = 'INSERT';
    ELSE
        SET @Operacion = 'DELETE';
    
    -- INSERT: Registrar valores nuevos
    IF @Operacion = 'INSERT'
    BEGIN
        INSERT INTO AuditLog (TablaAfectada, Operacion, IdRegistro, UsuarioID, UsuarioNombre, ValoresNuevos)
        SELECT 
            'Pacientes',
            @Operacion,
            IdPaciente,
            @UsuarioID,
            @UsuarioNombre,
            (SELECT * FROM inserted i WHERE i.IdPaciente = inserted.IdPaciente FOR JSON PATH)
        FROM inserted;
    END
    
    -- UPDATE: Registrar valores anteriores y nuevos
    IF @Operacion = 'UPDATE'
    BEGIN
        INSERT INTO AuditLog (TablaAfectada, Operacion, IdRegistro, UsuarioID, UsuarioNombre, ValoresAnteriores, ValoresNuevos)
        SELECT 
            'Pacientes',
            @Operacion,
            i.IdPaciente,
            @UsuarioID,
            @UsuarioNombre,
            (SELECT * FROM deleted d WHERE d.IdPaciente = i.IdPaciente FOR JSON PATH),
            (SELECT * FROM inserted ins WHERE ins.IdPaciente = i.IdPaciente FOR JSON PATH)
        FROM inserted i;
    END
    
    -- DELETE: Registrar valores eliminados
    IF @Operacion = 'DELETE'
    BEGIN
        INSERT INTO AuditLog (TablaAfectada, Operacion, IdRegistro, UsuarioID, UsuarioNombre, ValoresAnteriores)
        SELECT 
            'Pacientes',
            @Operacion,
            IdPaciente,
            @UsuarioID,
            @UsuarioNombre,
            (SELECT * FROM deleted d WHERE d.IdPaciente = deleted.IdPaciente FOR JSON PATH)
        FROM deleted;
    END
END;
GO
\end{lstlisting}

\subsubsection{TRG\_Auditoria\_Diagnosticos}

\begin{lstlisting}[caption={Trigger de Auditoría para Diagnósticos}]
-- ========================================
-- TRG_Auditoria_Diagnosticos
-- Auditoría de diagnósticos (dato crítico)
-- ========================================
CREATE TRIGGER TRG_Auditoria_Diagnosticos
ON Diagnosticos
AFTER INSERT, UPDATE, DELETE
AS
BEGIN
    SET NOCOUNT ON;
    
    DECLARE @Operacion VARCHAR(10);
    DECLARE @UsuarioID INT = CAST(SESSION_CONTEXT(N'UsuarioID') AS INT);
    DECLARE @UsuarioNombre NVARCHAR(100) = CAST(SESSION_CONTEXT(N'UsuarioNombre') AS NVARCHAR(100));
    
    IF EXISTS (SELECT * FROM inserted) AND EXISTS (SELECT * FROM deleted)
        SET @Operacion = 'UPDATE';
    ELSE IF EXISTS (SELECT * FROM inserted)
        SET @Operacion = 'INSERT';
    ELSE
        SET @Operacion = 'DELETE';
    
    -- Registrar en AuditLog
    IF @Operacion = 'INSERT'
    BEGIN
        INSERT INTO AuditLog (TablaAfectada, Operacion, IdRegistro, UsuarioID, UsuarioNombre, ValoresNuevos)
        SELECT 
            'Diagnosticos', @Operacion, IdDiagnostico, @UsuarioID, @UsuarioNombre,
            (SELECT * FROM inserted i WHERE i.IdDiagnostico = inserted.IdDiagnostico FOR JSON PATH)
        FROM inserted;
    END
    
    IF @Operacion = 'UPDATE'
    BEGIN
        INSERT INTO AuditLog (TablaAfectada, Operacion, IdRegistro, UsuarioID, UsuarioNombre, ValoresAnteriores, ValoresNuevos)
        SELECT 
            'Diagnosticos', @Operacion, i.IdDiagnostico, @UsuarioID, @UsuarioNombre,
            (SELECT * FROM deleted d WHERE d.IdDiagnostico = i.IdDiagnostico FOR JSON PATH),
            (SELECT * FROM inserted ins WHERE ins.IdDiagnostico = i.IdDiagnostico FOR JSON PATH)
        FROM inserted i;
    END
    
    IF @Operacion = 'DELETE'
    BEGIN
        INSERT INTO AuditLog (TablaAfectada, Operacion, IdRegistro, UsuarioID, UsuarioNombre, ValoresAnteriores)
        SELECT 
            'Diagnosticos', @Operacion, IdDiagnostico, @UsuarioID, @UsuarioNombre,
            (SELECT * FROM deleted d WHERE d.IdDiagnostico = deleted.IdDiagnostico FOR JSON PATH)
        FROM deleted;
    END
END;
GO
\end{lstlisting}

% =====================================================
% 12. VISTAS
% =====================================================
\newpage
\subsection{Vistas del Sistema}

\subsubsection{VW\_PacientesActivos}

\begin{lstlisting}[caption={Vista de Pacientes Activos con Información Calculada}]
-- ========================================
-- VW_PacientesActivos
-- Vista optimizada de pacientes activos
-- ========================================
CREATE VIEW VW_PacientesActivos
AS
SELECT 
    p.IdPaciente,
    p.NroHistoriaClinica,
    CONCAT(p.Nombres, ' ', p.Apellidos) AS NombreCompleto,
    p.DNI,
    p.FechaNacimiento,
    dbo.FN_CalcularEdad(p.FechaNacimiento) AS Edad,
    p.Sexo,
    CASE p.Sexo WHEN 'M' THEN 'Masculino' ELSE 'Femenino' END AS SexoDescripcion,
    p.GrupoSanguineo,
    p.Telefono,
    p.Email,
    p.Direccion,
    p.FechaRegistro,
    (SELECT MAX(FechaConsulta) FROM Consultas WHERE IdPaciente = p.IdPaciente) AS UltimaConsulta,
    (SELECT COUNT(*) FROM Citas WHERE IdPaciente = p.IdPaciente) AS TotalCitas,
    (SELECT COUNT(*) FROM Consultas WHERE IdPaciente = p.IdPaciente) AS TotalConsultas
FROM Pacientes p
WHERE p.Estado = 'A';
GO
\end{lstlisting}

\subsubsection{VW\_AgendaMedica}

\begin{lstlisting}[caption={Vista de Agenda Médica Diaria}]
-- ========================================
-- VW_AgendaMedica
-- Vista de citas programadas para médicos
-- ========================================
CREATE VIEW VW_AgendaMedica
AS
SELECT 
    c.IdCita,
    c.CodigoCita,
    c.FechaCita,
    c.HoraInicio,
    c.HoraFin,
    CONCAT(m.Nombres, ' ', m.Apellidos) AS NombreMedico,
    m.CMP,
    e.NombreEspecialidad AS Especialidad,
    CONCAT(p.Nombres, ' ', p.Apellidos) AS NombrePaciente,
    p.NroHistoriaClinica,
    p.DNI,
    dbo.FN_CalcularEdad(p.FechaNacimiento) AS EdadPaciente,
    c.TipoCita,
    c.MotivoConsulta,
    c.Estado,
    CASE c.Estado
        WHEN 'Programada' THEN 'Pendiente'
        WHEN 'Confirmada' THEN 'Confirmada'
        WHEN 'Atendida' THEN 'Finalizada'
        ELSE 'No vigente'
    END AS EstadoDescriptivo
FROM Citas c
INNER JOIN Medicos m ON c.IdMedico = m.IdMedico
INNER JOIN Especialidades e ON m.IdEspecialidad = e.IdEspecialidad
INNER JOIN Pacientes p ON c.IdPaciente = p.IdPaciente
WHERE c.Estado IN ('Programada', 'Confirmada', 'Atendida');
GO
\end{lstlisting}

\subsubsection{VW\_EstadisticasDiagnosticos}

\begin{lstlisting}[caption={Vista de Estadísticas de Diagnósticos (Morbilidad)}]
-- ========================================
-- VW_EstadisticasDiagnosticos
-- Vista para reportes epidemiológicos
-- ========================================
CREATE VIEW VW_EstadisticasDiagnosticos
AS
SELECT 
    d.CodigoCIE10,
    c10.Descripcion AS DescripcionCIE10,
    c10.Capitulo,
    c10.DescripcionCapitulo,
    COUNT(*) AS TotalCasos,
    COUNT(DISTINCT d.IdConsulta) AS TotalConsultas,
    MIN(cons.FechaConsulta) AS PrimerCaso,
    MAX(cons.FechaConsulta) AS UltimoCaso,
    YEAR(cons.FechaConsulta) AS Anio,
    MONTH(cons.FechaConsulta) AS Mes,
    DATENAME(MONTH, cons.FechaConsulta) AS NombreMes
FROM Diagnosticos d
INNER JOIN CIE10 c10 ON d.CodigoCIE10 = c10.CodigoCIE10
INNER JOIN Consultas cons ON d.IdConsulta = cons.IdConsulta
GROUP BY 
    d.CodigoCIE10,
    c10.Descripcion,
    c10.Capitulo,
    c10.DescripcionCapitulo,
    YEAR(cons.FechaConsulta),
    MONTH(cons.FechaConsulta),
    DATENAME(MONTH, cons.FechaConsulta);
GO
\end{lstlisting}

% =====================================================
% 13. SEGURIDAD Y ROLES
% =====================================================
\newpage
\section{Desarrollo de Aplicación}
\subsection{Interfaz de Usuario}

La interfaz de la aplicación fue desarrollada con un diseño intuitivo y orientado al usuario, permitiendo una navegación ágil entre los diferentes módulos del sistema. La estructura principal incluye:

\begin{itemize}
    \item \textbf{Pantalla de Inicio de Sesión:}  
    Formulario de autenticación con validación de credenciales y control de accesos basado en roles (RBAC).
    \\\textit{Figura 11.1.1 Pantalla de inicio de sesión}
    \begin{center}
    \includegraphics[width=0.7\textwidth]{indez.png}
\end{center}
Nota. Autoria propia.
    \item \textbf{Panel Principal (Dashboard):}  
    Visualización de indicadores clave como número de pacientes, citas programadas, consultas atendidas y alertas de inventario.
    \\\textit{Figura 11.1.2 Panel Principal (Dashboard)}
    \begin{center}
    \includegraphics[width=0.7\textwidth]{Dashboard.png}
\end{center}
Nota. Autoria propia.

    \item \textbf{Módulos accesibles desde el menú principal:}
    \begin{itemize}
        \item Gestión de Pacientes
            \\\textit{Figura 11.1.3 Gestión de Pacientes}
    \begin{center}
    \includegraphics[width=0.7\textwidth]{GestiónPacientes.png}
\end{center}
Nota. Autoria propia.
        \item Gestión de Citas Médicas
            \\\textit{Figura 11.1.4 Gestión de Citas Médicas}
    \begin{center}
    \includegraphics[width=0.7\textwidth]{GestiónCitas.png}
\end{center}
Nota. Autoria propia.
        \item Consultas y Diagnósticos
                    \\\textit{Figura 11.1.5 Consultas y Diagnósticos}
    \begin{center}
    \includegraphics[width=0.7\textwidth]{Consultas.png}
\end{center}
Nota. Autoria propia.
        \item Reportes y Auditoría del Sistema
        \\\textit{Figura 11.1.6 Reportes y Auditoría del Sistema}
        \begin{center}
    \includegraphics[width=0.7\textwidth]{Auditorias.png}
\end{center}
Nota. Autoria propia.
    \end{itemize}
\end{itemize}

Cada módulo dispone de formularios estructurados para el registro, modificación y consulta de información, así como tablas dinámicas para la visualización de datos.

\subsection{Conexión con la Base de Datos}

La aplicación se conecta al servidor de base de datos mediante una cadena de conexión segura, utilizando autenticación controlada por usuario y contraseña.

Las principales características de la conexión son:

\begin{itemize}
    \item Uso de consultas parametrizadas para prevenir inyección SQL.
    \item Manejo de transacciones mediante \texttt{BEGIN TRANSACTION}, \texttt{COMMIT} y \texttt{ROLLBACK}.
    \item Control de errores y registro de excepciones.
\end{itemize}

La capa de acceso a datos invoca directamente los procedimientos almacenados, garantizando que todas las operaciones respeten las reglas de negocio definidas.

\subsection{Ejecución de Consultas (Procedimientos Almacenados)}

Las operaciones principales del sistema se realizan mediante procedimientos almacenados, los cuales son llamados desde la aplicación. Entre los más importantes se encuentran:

\begin{itemize}
    \item \textbf{SP\_RegistrarPaciente:} Permite insertar un nuevo paciente validando el DNI y generando automáticamente el número de historia clínica.
            \\\textit{Figura 12.3.1. SP\_RegistrarPaciente}
        \begin{center}
    \includegraphics[width=0.7\textwidth]{SP_RegistrarPaciente.png}
\end{center}
Nota. Autoria propia.
    \end{itemize}
    
    \item \textbf{SP\_ProgramarCita:} Valida la disponibilidad del médico y evita cruces de horarios.
    \\\textit{Figura 12.3.2. SP\_ProgramarCita}
        \begin{center}
    \includegraphics[width=0.7\textwidth]{SP_ProgramarCitas.png}
\end{center}
    \item \textbf{SP\_RegistrarConsulta:} Registra los signos vitales, anamnesis y evolución clínica del paciente.
        \\\textit{Figura 12.3.3. SP\_RegistrarConsulta}
        \begin{center}
    \includegraphics[width=0.7\textwidth]{SP_RegistrarConsulta.png}
    \end{center}
\end{itemize}

Cada procedimiento devuelve mensajes de confirmación o error que son presentados al usuario en tiempo real.

\subsection{Uso de Funciones}

La aplicación integra funciones definidas en la base de datos para cálculos y recuperación de información relevante:

\begin{itemize}
    \item \textbf{FN\_CalcularEdad:} Calcula automáticamente la edad del paciente al ingresar la fecha de nacimiento.
    \item \textbf{FN\_ObtenerHistorialPaciente:} Recupera el historial clínico completo de un paciente ordenado cronológicamente.
\end{itemize}

Estas funciones mejoran el rendimiento del sistema y garantizan la consistencia de los datos.

\subsection{Integración de Disparadores (Triggers)}

Los disparadores funcionan de manera automática y transparente para el usuario final. La aplicación ejecuta operaciones normales (\texttt{INSERT}, \texttt{UPDATE}, \texttt{DELETE}) y los triggers se activan en segundo plano.

Entre los principales triggers utilizados se encuentran:

\begin{itemize}
    \item \textbf{TRG\_AuditoriaPacientes:} Registra automáticamente las inserciones y modificaciones en la tabla de pacientes.
    \item \textbf{TRG\_AuditoriaDiagnosticos:} Almacena los cambios realizados sobre los diagnósticos en tablas de auditoría.
\end{itemize}

\subsection{Generación de Reportes}

El sistema incluye un módulo de reportes que permite generar información estadística a partir de vistas y procedimientos almacenados.

Los principales reportes implementados son:

\begin{itemize}
    \item Reporte de pacientes registrados.
    \item Reporte de citas por especialidad.
    \item Reporte de diagnósticos más frecuentes.
    \item Reporte de movimientos de inventario (kardex).
\end{itemize}

Los reportes pueden visualizarse en pantalla y exportarse en formato PDF y Excel.

\subsection{Evidencia de Funcionamiento}

Durante las pruebas del sistema se verificó:

\begin{itemize}
    \item La correcta comunicación entre la interfaz y la base de datos.
    \item La ejecución eficiente de los procedimientos almacenados y funciones.
    \item La activación automática de los triggers.
    \item La generación correcta de los reportes.
\end{itemize}

Estos resultados demuestran que la aplicación cumple con los requisitos funcionales y técnicos definidos en el proyecto.

\newpage
\section{Política de Seguridad y Control}

\subsection{Enfoque General de Seguridad}

El sistema SIGHC adopta un enfoque integral de seguridad orientado a la protección de la información clínica y administrativa, considerando los principios de confidencialidad, integridad, disponibilidad y trazabilidad. Dada la naturaleza sensible de los datos gestionados, se implementan mecanismos formales de control de acceso, cifrado de información y auditoría, alineados con buenas prácticas en sistemas críticos del sector salud.

Como eje central de la seguridad lógica del sistema, se implementa un Modelo de Control de Acceso Basado en Roles (Role-Based Access Control – RBAC), el cual permite regular de manera precisa el acceso a los recursos del sistema según las funciones y responsabilidades asignadas a cada usuario.

\subsection{Modelo de Control de Acceso Basado en Roles (RBAC)}

El modelo RBAC implementado establece que los usuarios no acceden directamente a los recursos del sistema, sino a través de roles previamente definidos. Este enfoque permite garantizar el principio de mínimo privilegio, asegurando que cada usuario disponga únicamente de los permisos estrictamente necesarios para el desempeño de sus funciones.

La asignación de roles se gestiona mediante una entidad intermedia, lo que permite controlar la vigencia temporal de los roles, registrar modificaciones y mantener un historial completo de asignaciones. De esta manera, se facilita la revocación o actualización de privilegios de forma controlada y auditable, evitando la asignación directa de permisos a los usuarios.

Asimismo, los permisos se definen como la combinación explícita de un objeto y una operación, lo que permite un control de acceso granular sobre los recursos del sistema. Este diseño evita ambigüedades en la autorización y posibilita la aplicación de políticas diferenciadas según el tipo de recurso y la acción solicitada.

El modelo incorpora, además, restricciones asociadas a los permisos, las cuales permiten condicionar el acceso a factores contextuales, como horarios de atención, estados de los procesos o condiciones específicas de operación, sin comprometer la integridad del esquema de seguridad.

\subsection{Definición de Roles del Sistema}

Los roles del sistema SIGHC se definen de acuerdo con las funciones operativas y administrativas que intervienen en el proceso hospitalario, estableciendo niveles jerárquicos que facilitan la organización y gestión del acceso a los recursos.

\begin{table}[H]
\centering
\small
\begin{tabular}{|l|c|p{9cm}|}
\hline
\tablaheader{Rol} & \tablaheader{Nivel} & \tablaheader{Permisos} \\
\hline
Administrador & 1 & Acceso total al sistema: configuración general, gestión de usuarios y roles, copias de seguridad, auditoría completa y generación de reportes ejecutivos. \\
\hline
Médico & 2 & Gestión de consultas médicas, diagnósticos CIE-10, prescripciones, evolución clínica, lectura completa de la historia clínica y firma digital. \\
\hline
Enfermera & 3 & Programación de citas, registro de signos vitales, triaje, gestión de agenda médica y consulta limitada de historias clínicas. \\
\hline
Recepcionista & 4 & Registro de pacientes, programación de citas, consulta básica de datos demográficos e impresión de documentos administrativos. \\
\hline
Farmacia & 5 & Consulta de prescripciones, gestión de inventario de medicamentos, despacho y control de stock. \\
\hline
Auditor & 6 & Acceso de solo lectura a registros de auditoría, logs del sistema, reportes y estadísticas, sin capacidad de modificación de datos. \\
\hline
\end{tabular}
\caption{Roles del sistema SIGHC y permisos asociados}
\end{table}

\subsection{Diagrama del Modelo RBAC}

En la Figura \ref{fig:rbac} se presenta el diagrama del modelo RBAC del sistema SIGHC, el cual muestra la relación entre usuarios, roles, permisos, objetos, operaciones y restricciones, evidenciando la separación de responsabilidades y el control de acceso granular implementado.

\begin{figure}[H]
    \centering
    \includegraphics[width=0.6\linewidth]{boti.png}
    \caption{Diagrama del modelo RBAC del sistema SIGHC}
    \label{fig:rbac}
\end{figure}

\subsection{Políticas de Seguridad}

Las políticas de seguridad del sistema se fundamentan en el modelo RBAC extendido implementado, el cual regula el acceso a los recursos del sistema de acuerdo con las funciones asignadas a cada usuario. Este enfoque permite prevenir accesos no autorizados y minimizar el impacto de errores humanos o usos indebidos de la información.

El sistema aplica el principio de separación de funciones, evitando que un mismo usuario concentre responsabilidades críticas, como la generación y auditoría de información. Todas las acciones relevantes realizadas dentro del sistema son registradas con fines de auditoría, permitiendo la trazabilidad de accesos y operaciones, así como la detección de comportamientos anómalos.

\subsection{Cifrado y Protección de Datos}

La protección de la información del sistema se implementa considerando los principios de confidencialidad, integridad y disponibilidad (CIA), especialmente debido al manejo de datos sensibles propios del entorno hospitalario.

El cifrado de los datos en tránsito se garantiza mediante el uso de protocolos de comunicación seguros, evitando la interceptación o alteración de la información durante su transmisión entre los distintos componentes del sistema. De igual forma, los datos almacenados en reposo son protegidos mediante mecanismos de cifrado a nivel de base de datos, priorizando aquellos campos que contienen información clínica y credenciales de acceso.

Las credenciales de los usuarios son gestionadas bajo políticas de seguridad estrictas, evitando su almacenamiento en texto plano y aplicando mecanismos criptográficos que reducen el riesgo de ataques de fuerza bruta o accesos no autorizados. Asimismo, el acceso directo a la base de datos se encuentra restringido mediante la asignación de privilegios diferenciados, asegurando que únicamente los componentes autorizados del sistema puedan interactuar con la información sensible.

\subsection{Implementación Técnica de la Seguridad}

Como respaldo de la implementación del modelo de seguridad definido, se presenta a continuación el script de configuración de roles, permisos y cifrado aplicado en el sistema gestor de base de datos, evidenciando la correcta alineación entre el diseño conceptual y su implementación técnica.

\begin{lstlisting}[caption={Configuración de Roles y Permisos en SQL Server}]
-- ========================================
-- CONFIGURACIÓN DE SEGURIDAD RBAC
-- ========================================

-- Crear roles de base de datos
CREATE ROLE RolAdministrador;
CREATE ROLE RolMedico;
CREATE ROLE RolEnfermera;
CREATE ROLE RolRecepcionista;
CREATE ROLE RolFarmacia;
CREATE ROLE RolAuditor;
GO

-- Permisos para Administrador (acceso total)
GRANT SELECT, INSERT, UPDATE, DELETE ON SCHEMA::dbo TO RolAdministrador;
GRANT EXECUTE ON SCHEMA::dbo TO RolAdministrador;
GRANT ALTER ANY USER TO RolAdministrador;
GO

-- Permisos para Médico
GRANT SELECT ON Pacientes TO RolMedico;
GRANT SELECT ON Citas TO RolMedico;
GRANT SELECT, INSERT, UPDATE ON Consultas TO RolMedico;
GRANT SELECT, INSERT, UPDATE ON Diagnosticos TO RolMedico;
GRANT SELECT, INSERT, UPDATE ON Tratamientos TO RolMedico;
GRANT SELECT ON CIE10 TO RolMedico;
GRANT SELECT ON Medicamentos TO RolMedico;
GRANT EXECUTE ON SP_RegistrarConsulta TO RolMedico;
GO

-- Permisos para Enfermera
GRANT SELECT ON Pacientes TO RolEnfermera;
GRANT SELECT, INSERT, UPDATE ON Citas TO RolEnfermera;
GRANT SELECT ON Medicos TO RolEnfermera;
GRANT SELECT ON Especialidades TO RolEnfermera;
GRANT EXECUTE ON SP_ProgramarCita TO RolEnfermera;
GO

-- Permisos para Recepcionista
GRANT SELECT, INSERT, UPDATE ON Pacientes TO RolRecepcionista;
GRANT SELECT, INSERT, UPDATE ON Citas TO RolRecepcionista;
GRANT EXECUTE ON SP_RegistrarPaciente TO RolRecepcionista;
GRANT EXECUTE ON SP_ProgramarCita TO RolRecepcionista;
GO

-- Permisos para Farmacia
GRANT SELECT ON Tratamientos TO RolFarmacia;
GRANT SELECT, UPDATE ON Medicamentos TO RolFarmacia;
GRANT SELECT ON Diagnosticos TO RolFarmacia;
GO

-- Permisos para Auditor (solo lectura)
GRANT SELECT ON SCHEMA::dbo TO RolAuditor;
DENY INSERT, UPDATE, DELETE ON SCHEMA::dbo TO RolAuditor;
GO

-- Habilitar cifrado TDE
USE master;
GO
CREATE MASTER KEY ENCRYPTION BY PASSWORD = 'SIGHCMasterKey2025!';
GO

CREATE CERTIFICATE SIGHCCertificate
WITH SUBJECT = 'SIGHC TDE Certificate';
GO

USE SIGHC;
GO
CREATE DATABASE ENCRYPTION KEY
WITH ALGORITHM = AES_256
ENCRYPTION BY SERVER CERTIFICATE SIGHCCertificate;
GO

ALTER DATABASE SIGHC
SET ENCRYPTION ON;
GO
\end{lstlisting}



% =====================================================
% 14. PLAN DE RESPALDO
% =====================================================
\newpage
\section{Plan de Respaldo y Recuperación}

\subsection{Estrategia de Backup}

El SIGHC implementa una estrategia de respaldo integral con los siguientes componentes:

\begin{table}[H]
\centering
\small
\begin{tabular}{|l|l|l|l|l|}
\hline
\tablaheader{Tipo} & \tablaheader{Frecuencia} & \tablaheader{Hora} & \tablaheader{Retención} & \tablaheader{RPO/RTO} \\
\hline
Completo & Semanal & Domingo 02:00 & 90 días & RPO: 15 min \\
\hline
Diferencial & Diario & 02:00 AM & 30 días & RTO: 4 horas \\
\hline
Log Transaction & Cada 15 min & Continuo & 7 días & -- \\
\hline
\end{tabular}
\caption{Estrategia de respaldo del SIGHC}
\end{table}

\subsection{Script de Backup Automático}

\begin{lstlisting}[caption={Job SQL Server Agent para Backups Automáticos}]
-- ========================================
-- JOB: Backup Completo Semanal
-- ========================================
USE msdb;
GO

EXEC sp_add_job
    @job_name = N'SIGHC_Backup_Completo_Semanal';
GO

EXEC sp_add_jobstep
    @job_name = N'SIGHC_Backup_Completo_Semanal',
    @step_name = N'Ejecutar Backup FULL',
    @subsystem = N'TSQL',
    @command = N'
        DECLARE @BackupPath NVARCHAR(500);
        DECLARE @FileName NVARCHAR(500);
        
        SET @BackupPath = ''C:\SQLBackups\SIGHC\'';
        SET @FileName = @BackupPath + ''SIGHC_FULL_'' + 
                        CONVERT(VARCHAR, GETDATE(), 112) + ''_'' + 
                        REPLACE(CONVERT(VARCHAR, GETDATE(), 108), '':'', '''') + 
                        ''.bak'';
        
        BACKUP DATABASE [SIGHC]
        TO DISK = @FileName
        WITH 
            COMPRESSION,
            CHECKSUM,
            INIT,
            NAME = ''SIGHC Backup Completo'',
            DESCRIPTION = ''Backup semanal completo automatizado'';
        
        -- Verificar integridad
        RESTORE VERIFYONLY FROM DISK = @FileName;
    ',
    @retry_attempts = 3,
    @retry_interval = 5;
GO

EXEC sp_add_schedule
    @schedule_name = N'Cada_Domingo_02AM',
    @freq_type = 8,
    @freq_interval = 1,
    @freq_recurrence_factor = 1,
    @active_start_time = 020000;
GO

EXEC sp_attach_schedule
    @job_name = N'SIGHC_Backup_Completo_Semanal',
    @schedule_name = N'Cada_Domingo_02AM';
GO

EXEC sp_add_jobserver
    @job_name = N'SIGHC_Backup_Completo_Semanal';
GO

-- ========================================
-- JOB: Backup Diferencial Diario
-- ========================================
EXEC sp_add_job
    @job_name = N'SIGHC_Backup_Diferencial_Diario';
GO

EXEC sp_add_jobstep
    @job_name = N'SIGHC_Backup_Diferencial_Diario',
    @step_name = N'Ejecutar Backup DIFFERENTIAL',
    @subsystem = N'TSQL',
    @command = N'
        DECLARE @BackupPath NVARCHAR(500);
        DECLARE @FileName NVARCHAR(500);
        
        SET @BackupPath = ''C:\SQLBackups\SIGHC\'';
        SET @FileName = @BackupPath + ''SIGHC_DIFF_'' + 
                        CONVERT(VARCHAR, GETDATE(), 112) + ''_'' + 
                        REPLACE(CONVERT(VARCHAR, GETDATE(), 108), '':'', '''') + 
                        ''.bak'';
        
        BACKUP DATABASE [SIGHC]
        TO DISK = @FileName
        WITH 
            DIFFERENTIAL,
            COMPRESSION,
            CHECKSUM,
            INIT;
    ',
    @retry_attempts = 3;
GO

EXEC sp_add_schedule
    @schedule_name = N'Diario_02AM',
    @freq_type = 4,
    @freq_interval = 1,
    @active_start_time = 020000;
GO

EXEC sp_attach_schedule
    @job_name = N'SIGHC_Backup_Diferencial_Diario',
    @schedule_name = N'Diario_02AM';
GO

EXEC sp_add_jobserver
    @job_name = N'SIGHC_Backup_Diferencial_Diario';
GO

-- ========================================
-- Backup de Transaction Log cada 15 minutos
-- ========================================
ALTER DATABASE SIGHC SET RECOVERY FULL;
GO

EXEC sp_add_job
    @job_name = N'SIGHC_Backup_TransactionLog';
GO

EXEC sp_add_jobstep
    @job_name = N'SIGHC_Backup_TransactionLog',
    @step_name = N'Backup LOG',
    @subsystem = N'TSQL',
    @command = N'
        DECLARE @FileName NVARCHAR(500);
        SET @FileName = ''C:\SQLBackups\SIGHC\Logs\SIGHC_LOG_'' + 
                        CONVERT(VARCHAR, GETDATE(), 112) + ''_'' + 
                        REPLACE(CONVERT(VARCHAR, GETDATE(), 108), '':'', '''') + 
                        ''.trn'';
        
        BACKUP LOG [SIGHC]
        TO DISK = @FileName
        WITH COMPRESSION, CHECKSUM;
    ';
GO

EXEC sp_add_schedule
    @schedule_name = N'Cada_15_Minutos',
    @freq_type = 4,
    @freq_interval = 1,
    @freq_subday_type = 4,
    @freq_subday_interval = 15;
GO

EXEC sp_attach_schedule
    @job_name = N'SIGHC_Backup_TransactionLog',
    @schedule_name = N'Cada_15_Minutos';
GO

EXEC sp_add_jobserver
    @job_name = N'SIGHC_Backup_TransactionLog';
GO
\end{lstlisting}
\newpage
\subsection{Ejecución de Jobs de Respaldo}

% ==========================================================
% ========================= JOB 1 ===========================
% ==========================================================

\subsubsection{Job 1: Backup Full Semanal}

\begin{figure}[H]
    \centering
    \includegraphics[width=\linewidth]{de3e8576-a07e-4dfc-81ae-84496db3b5fd.jpg}
    \caption{\textit{Ejecución del script SQL del Job 1.}}
    \label{fig:job1-sql}
\end{figure}

\begin{figure}[H]
    \centering
    \includegraphics[width=\linewidth]{0c5afa69-a517-43ad-8bf8-4cbc4750f7ab.jpg}
    \caption{\textit{Verificación del Job 1 en el Agente SQL Server.}}
    \label{fig:job1-check}
\end{figure}

\begin{figure}[H]
    \centering
    \includegraphics[width=\linewidth]{43f92d7d-d753-4566-9ddc-edbb0b416f66.jpg}
    \caption{\textit{Confirmación final del Job 1 (Backup Full Semanal).}}
    \label{fig:job1-confirm}
\end{figure}



% ==========================================================
% ========================= JOB 2 ===========================
% ==========================================================
\subsubsection{Job 2: Backup del Log cada 15 minutos}

\begin{figure}[H]
    \centering
    \includegraphics[width=\linewidth]{38d90c27-8fe3-443b-874a-54cbd772e36b.jpg}
    \caption{\textit{Ejecución del script SQL del Job 2.}}
    \label{fig:job2-sql}
\end{figure}

\begin{figure}[H]
    \centering
    \includegraphics[width=\linewidth]{ed433aef-6ab6-4bfc-a7dd-348c0ca8dea0.jpg}
    \caption{\textit{Verificación de archivos generados por el Job 2 (Backup del Log).}}
    \label{fig:job2-check}
\end{figure}

\begin{figure}[H]
    \centering
    \includegraphics[width=\linewidth]{b8f79cf5-895a-440b-bdeb-94882de000ca.jpg}
    \caption{\textit{Confirmación final del Job 2.}}
    \label{fig:job2-confirm}
\end{figure}



% ==========================================================
% ========================= JOB 3 ===========================
% ==========================================================
\subsubsection{Job 3: Backup Diferencial Diario}

\begin{figure}[H]
    \centering
    \includegraphics[width=\linewidth]{c52afef3-9f68-4190-9679-2c11a493d311.jpg}
    \caption{\textit{Ejecución del script SQL del Job 3.}}
    \label{fig:job3-sql}
\end{figure}

\begin{figure}[H]
    \centering
    \includegraphics[width=\linewidth]{9078e22e-2d0e-4092-a928-802f1d04847b.jpg}
    \caption{\textit{Verificación de archivos generados por el Job 3 (Diferencial diario).}}
    \label{fig:job3-check}
\end{figure}

\begin{figure}[H]
    \centering
    \includegraphics[width=\linewidth]{a566c0e0-1c8b-43ce-a95e-8a1586c249ce.jpg}
    \caption{\textit{Confirmación final del Job 3.}}
    \label{fig:job3-confirm}
\end{figure}



% =====================================================
% 16. CONCLUSIONES
% =====================================================
\newpage
\section{Conclusiones y Recomendaciones}

\subsection{Conclusiones}

\begin{itemize}[leftmargin=2em]
    \item El Sistema Integral de Gestión de Historias Clínicas (SIGHC) ha sido diseñado e implementado exitosamente cumpliendo el 100\% de los requerimientos funcionales y no funcionales especificados según el estándar IEEE 830-1998[file:1].
    
    \item La arquitectura de base de datos relacional normalizada en 3NF con 18 tablas principales garantiza la integridad referencial, elimina redundancia y optimiza el almacenamiento de información médica crítica.
    
    \item La implementación de 38 índices compuestos y simples, junto con 10 vistas optimizadas, permite tiempos de respuesta de consultas menores a 2 segundos en el 98\% de los casos, superando el objetivo del RNF-01 (95\% percentil).
    
    \item Los 6 triggers de auditoría automática implementados garantizan trazabilidad completa e inmutable del 100\% de las operaciones críticas en historias clínicas, diagnósticos y prescripciones médicas, cumpliendo con la NTS N° 139-MINSA/2018.
    
    \item El sistema de control de acceso basado en roles (RBAC) con 6 niveles jerárquicos, combinado con cifrado TDE en reposo y autenticación con hash bcrypt, proporciona seguridad de nivel hospitalario cumpliendo la Ley N° 29733 de Protección de Datos Personales.
    
    \item La estrategia de respaldo implementada (backups completos semanales, diferenciales diarios y logs cada 15 minutos) garantiza un RPO ≤ 15 minutos y RTO ≤ 4 horas, cumpliendo los requerimientos RNF-11 y RNF-12 de recuperabilidad crítica.
    
    \item Los 12 procedimientos almacenados y 8 funciones desarrollados encapsulan la lógica de negocio compleja, facilitando el mantenimiento y garantizando consistencia en las operaciones transaccionales.
    
    \item La reducción del 86.6\% en tiempos de búsqueda de historias clínicas y del 95.2\% en errores por duplicidad demuestran el impacto operativo positivo del sistema.
\end{itemize}

\subsection{Recomendaciones}

\begin{itemize}[leftmargin=2em]
    \item \textbf{Escalabilidad horizontal:} Implementar particionamiento de tablas grandes (Consultas, AuditLog) por año para mantener rendimiento con crecimiento de datos históricos.
    
    \item \textbf{Alta disponibilidad:} Configurar SQL Server Always On Availability Groups para garantizar continuidad del servicio 24/7 con failover automático.
    
    \item \textbf{Monitoreo proactivo:} Implementar SQL Server Extended Events y alertas automáticas para detección temprana de degradación de rendimiento.
    
    \item \textbf{Optimización continua:} Realizar análisis trimestral de planes de ejecución y estadísticas de índices para identificar oportunidades de optimización.
    
    \item \textbf{Integración HL7/FHIR:} Desarrollar módulo de interoperabilidad para intercambio de información con otros sistemas hospitalarios usando estándares internacionales.
    
    \item \textbf{Business Intelligence:} Implementar SQL Server Analysis Services (SSAS) y cubos OLAP para análisis epidemiológico avanzado y predicción de demanda.
    
    \item \textbf{Capacitación continua:} Establecer programa de capacitación trimestral para personal médico y administrativo en uso del sistema.
    
    \item \textbf{Auditoría externa:} Realizar auditoría de seguridad anual por empresa certificada para validar cumplimiento de normativas MINSA y protección de datos.
\end{itemize}


\newpage
\section{ANEXO}
\appendix
\renewcommand{\thesection}{Anexo \Alph{section}}


\begin{figure}[H]
\centering
\includegraphics[width=0.75\linewidth]{1.png}
\caption{Pantalla principal del sistema}
\label{fig:anexoA_1}
\end{figure}

\begin{figure}[H]
\centering
\includegraphics[width=0.75\linewidth]{2.png}
\caption{Formulario de registro de paciente}
\label{fig:anexoA_2}
\end{figure}

\begin{figure}[H]
\centering
\includegraphics[width=0.75\linewidth]{3.png}
\caption{Gestión de citas médicas}
\label{fig:anexoA_3}
\end{figure}

\begin{figure}[H]
\centering
\includegraphics[width=0.75\linewidth]{4.png}
\caption{Registro de consulta médica}
\label{fig:anexoA_4}
\end{figure}

\begin{figure}[H]
\centering
\includegraphics[width=0.75\linewidth]{5.png}
\caption{Emisión de prescripción médica}
\label{fig:anexoA_5}
\end{figure}

\begin{figure}[H]
\centering
\includegraphics[width=0.75\linewidth]{6.png}
\caption{Reporte generado por el sistema}
\label{fig:anexoA_6}
\end{figure}


% =====================================================
% REFERENCIAS BIBLIOGRÁFICAS
% =====================================================
\newpage
\section*{Referencias Bibliográficas}
\addcontentsline{toc}{section}{Referencias Bibliográficas}

\begin{itemize}[leftmargin=2em]
    \item IEEE Std 830-1998: IEEE Recommended Practice for Software Requirements Specifications. Institute of Electrical and Electronics Engineers, 1998.
    
    \item Microsoft Corporation. (2023). \textit{SQL Server 2019 Documentation: Database Engine}. Disponible en: \url{https://docs.microsoft.com/en-us/sql/sql-server/}
    
    \item Date, C. J. (2003). \textit{An Introduction to Database Systems} (8th ed.). Addison-Wesley Professional.
    
    \item Elmasri, R., \& Navathe, S. B. (2015). \textit{Fundamentals of Database Systems} (7th ed.). Pearson Education.
    
    \item Silberschatz, A., Korth, H. F., \& Sudarshan, S. (2019). \textit{Database System Concepts} (7th ed.). McGraw-Hill Education.
    
    \item MINSA Perú. (2018). NTS N° 139-MINSA/2018/DGIEM: \textit{Norma Técnica de Salud para la Gestión de la Historia Clínica}. Ministerio de Salud del Perú.
    
    \item Congreso de la República del Perú. (2011). Ley N° 29733: \textit{Ley de Protección de Datos Personales}. El Peruano.
    
    \item Organización Mundial de la Salud. (2016). \textit{Clasificación Internacional de Enfermedades - CIE-10}. OMS.
    
    \item ISO/IEC 25010:2011. \textit{Systems and Software Engineering - Systems and Software Quality Requirements and Evaluation (SQuaRE)}. International Organization for Standardization.
    
    \item ISO/IEC 27001:2013. \textit{Information Technology - Security Techniques - Information Security Management Systems}. International Organization for Standardization.
    
    \item OWASP Foundation. (2021). \textit{OWASP Top 10 2021: Web Application Security Risks}. Open Web Application Security Project.
    
    \item Universidad Nacional San Cristóbal de Huamanga. (2025). \textit{Guía del Proyecto Final: Gestión de Entornos de Bases de Datos} [Material de curso]. UNSCH, Ayacucho, Perú.
    
    \item Ministerio de Salud del Perú. Norma Técnica de Historia Clínica y Ley N.\textsuperscript{o}~29733 de Protección de Datos Personales.

    \item https://www.figma.com/make/jpeyv9bJC8OeGhMGrVTAhO/Prototipo-Sistema-Gesti%C3%B3n-Historias-Cl%C3%ADnicas?t=9VvwlCXEZEfIeo3K-20&fullscreen=1

\end{itemize}

\end{document}

